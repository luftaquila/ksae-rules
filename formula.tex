\documentclass[final,a4paper,10pt]{report}
% packages
\usepackage{tikz}
\usepackage{float}
\usepackage{kotex}
\usepackage{amsmath}
\usepackage{amssymb}
\usepackage{caption}
\usepackage{enumitem}
\usepackage{etoolbox}
\usepackage{fancyhdr}
\usepackage{fontspec}
\usepackage{graphicx}
\usepackage{multirow}
\usepackage{titlesec}
\usepackage{tabularray}
\usepackage{indentfirst}
\usepackage[unicode,
            pdfencoding=auto,
            bookmarks=true,
            bookmarksnumbered=true,
            bookmarksopen=true,
            bookmarkstype=toc]{hyperref}
\usepackage[nameinlink,noabbrev]{cleveref}
\usepackage{bookmark}

% margins
\usepackage[margin=2cm,bottom=3.5cm]{geometry}

% fonts
\setmainfont{NanumMyeongjo}[
  UprightFont     = {NanumMyeongjo},
  BoldFont        = {NanumMyeongjo Bold},
  ExtraBoldFont   = {NanumMyeongjo ExtraBold},
  ItalicFont      = {NanumMyeongjo},
  ItalicFeatures  = {FakeSlant=0.2},
  LetterSpace=3
]

\linespread{1.3}\selectfont

\newfontfamily\textb{NanumMyeongjo Bold}
\newfontfamily\texteb{NanumMyeongjo ExtraBold}
\newfontfamily\pretendard{Pretendard}
\newfontfamily\pretendardb{Pretendard Bold}
\newfontfamily\pretendardeb{Pretendard ExtraBold}
\newfontfamily\hanja{Noto Sans CJK TC}[Script=CJK]

\makeatletter
  \spaceskip=1.5\fontdimen2\font
             plus 0.3\fontdimen3\font
             minus 0.2\fontdimen4\font
\makeatother

% borders
\usetikzlibrary{calc}
\AddToHook{shipout/background}{
  \begin{tikzpicture}[remember picture,overlay]
    \draw[line width=0.5pt]
      ($(current page.north west)+(1cm,-1cm)$) rectangle
      ($(current page.south east)+(-1cm,1.8cm)$);
  \end{tikzpicture}
}

% graphics
\pagestyle{fancy}
\fancyhf{}

\renewcommand{\headrulewidth}{0pt}
\renewcommand{\footrulewidth}{0pt}

\newlength{\borderoffset}
\setlength{\borderoffset}{1cm}
\newlength{\margin}
\setlength{\margin}{2cm}

\makeatletter
\newcommand{\setupfancy}{
  \fancyhfoffset[L]{\dimexpr\margin-\borderoffset\relax}
  \fancyhfoffset[R]{\dimexpr\margin-\borderoffset\relax}
  \lfoot{\pretendardb{자동차공학은 한국의 힘!}}
  \cfoot{\pretendardb{\thepage}}
  \rfoot{\pretendardb{사단법인 한국자동차공학회}}
}
\makeatother
\setupfancy
\fancypagestyle{plain}{\setupfancy}

\setlength{\headheight}{45pt}

\makeatletter
\newcommand{\SidePaddedBox}[5]{
  \fcolorbox{black}{gray}{
    \vbox{
      \vskip #1
      \hbox{
        \hskip #2
        \color{white}\textbf{\small #5}
        \hskip #3
      }
      \vskip #4
    }
  }
}
\makeatother

\fancypagestyle{firstpage}{
  \fancyheadoffset[L]{\dimexpr\margin-\borderoffset+0.5cm\relax}
  \fancyfootoffset[L]{\dimexpr\margin-\borderoffset\relax}
  \lhead{\SidePaddedBox{1pt}{10pt}{40pt}{1pt}{\pretendard\fontsize{10}{12}\selectfont 한국자동차공학회 규정}}
  \renewcommand{\headrulewidth}{0pt}
}

\newcounter{starnote}
\fancypagestyle{star}{
  \fancyheadoffset[L]{\dimexpr\margin-\borderoffset-0.5cm\relax}
  \refstepcounter{starnote}
  \lhead{[별표~\arabic{starnote}]}
}

% chapters
\makeatletter
\patchcmd{\chapter}
  {\if@openright\cleardoublepage\else\clearpage\fi}
  {}{}{}
\patchcmd{\chapter}
  {\thispagestyle{plain}}
  {}{}{}
\makeatother

\renewcommand{\chaptermark}[1]{}
\titleformat{\chapter}[hang]
  {\pretendardb\fontsize{12}{16}\selectfont}
  {제\arabic{chapter}장}
  {0.5em}
  {}
\titlespacing*{\chapter}{0pt}{3ex}{1ex}
\pagestyle{plain}

% sections
\counterwithout{section}{chapter}
\renewcommand\thesection{\arabic{section}}
\newcommand{\lawformat}[1]{\mbox{제\thesection{}조 (#1)}}
\titleformat{\section}[block]
  {\textb}
  {}
  {0pt}
  {\lawformat}
\titlespacing*{\section}{0pt}{3ex}{1ex}

% items
\DeclareRobustCommand*{\circlenum}{
  \ifcase\value{enumi}\relax
  \or ①\or ②\or ③\or ④\or ⑤\or ⑥\or ⑦\or ⑧\or ⑨\or ⑩
  \or ⑪\or ⑫\or ⑬\or ⑭\or ⑮\or ⑯\or ⑰\or ⑱\or ⑲\or ⑳
  \else
    \arabic{enumi}
  \fi
}
\renewcommand{\theenumi}{\circlenum}
\renewcommand{\labelenumi}{\normalfont\circlenum}

\DeclareRobustCommand{\koreannum}[1]{
  \ifcase#1 \or
    가\or 나\or 다\or 라\or 마\or 바\or 사\or 아\or 자\or 차\or 카\or 타\or 파\or 하
  \else ?\fi
}

\setlist[enumerate,1]{ref={제\thesection{}조~\arabic*항},labelsep=0.5em,itemsep=0pt,leftmargin=3em}
\setlist[enumerate,2]{ref={제\thesection{}조~\arabic{enumi}항~\arabic*번},label=\arabic*.,labelsep=0.5em,leftmargin=2.5em}
\setlist[enumerate,3]{label={\koreannum{\arabic*}.},labelsep=0.5em,leftmargin=2.5em}
\setlist[enumerate,4]{label=\alph*.,labelsep=0.5em,leftmargin=2.5em}
\setlist[itemize,1]{label=\textbullet,leftmargin=2em,itemsep=0.5ex,before=\small}

% texts
\sloppy
\setlength{\leftskip}{1.5em}
\setlength{\parindent}{0pt}

% hyperlinks
\let\origcref\cref
\renewcommand{\cref}[1]{\textit{\origcref{#1}}}

\creflabelformat{chapter}{#2제#1장#3}
\creflabelformat{section}{#2제#1조#3}
\crefname{chapter}{}{}
\crefname{section}{}{}
\crefname{figure}{그림}{그림}
\crefname{enumi}{}{}

% bookmarks
\makeatletter
\pdfstringdefDisableCommands{
  \def\thechapter{제\arabic{chapter}장}
  \def\thesection{제\arabic{section}조}
}
\makeatother

% figures
\graphicspath{{assets/}}
\captionsetup[figure]{
  name=그림,
  labelsep=period,
  figurewithin=none
}
\captionsetup[figure]{labelfont=bf, textfont=bf}

\newcommand{\fig}[3]{
  \begin{figure}[H]
    \centering
    \includegraphics[width=#3\linewidth]{#2/#1.jpg}
    \caption{#1}\label{fig:#1}
  \end{figure}
}

\begin{document}
\thispagestyle{firstpage}

\begin{center}
  {\fontsize{22}{25}\selectfont\pretendardb 대학생 자작자동차대회}\\[2.5ex]
  {\fontsize{22}{25}\selectfont\pretendardb Formula Student Korea 차량기술규정}\\[2.5ex]
  [ 시행 2024.11.21. 이사회 ]
\end{center}

\chapter{목적 및 일반사항}

\section{목적}
본 규정은 KSAE 대학생 자작자동차대회 대회운영규정(이하 "대회운영규정"이라 한다) 제10조 제4항에 따라 Formula 차량기술에 관한 사항을 규정함을 목적으로 한다.

\section{일반사항}
Formula 차량기술규정의 일반적인 사항들은 대회 운영 규정에 따른다.

\chapter{Formula 참가 차량의 조건}

\section{참가 차량의 정의}
\begin{enumerate}
  \item 1인승 차량으로, 4바퀴(앞바퀴2, 뒷바퀴2) 이상을 가진 차량으로 반드시 운전석과 휠이 개방된 포뮬러(Formula) 스타일이어야 한다.
  \item 외장은 드라이버를 보호하기 위해 차량의 앞(벌크헤드)에서부터 방화벽 또는 메인 롤 후프(Main Roll Hoop)까지 운전석을 제외하고는 개방된 공간이 없어야 한다.
  \item 참가 차량은 내연기관 또는 모터를 사용한 차량으로 규정에 맞게 대학생들이 설계하고 제작한 차량이어야 한다.
  \item 참가 차량은 대학생들이 직접 설계하고 제작한 710cc 이하의 내연기관이나 600VDC 이하의 전기 동력원을 사용하는 차량이어야 한다.
  \item 동일한 프레임을 사용한 차량은 2년까지 대회를 참가할 수 있다. 
\end{enumerate}

\section{참가 차량의 휠베이스 및 최저 지상고}
\begin{enumerate}
  \item 휠베이스 (Wheel Base)\\
    휠베이스는 앞, 뒤 타이어의 접촉면의 중심 사이의 거리를 측정하여 최소한 1,500mm 이상이 되어야 한다.
  \item 최저 지상고 (Ground Clearance)\\
    타이어 이외의 차량의 일부분이 지면과 접촉이 없어야 하며, 경기 중 노면과의 지속적인 접촉 및 부품의 탈락 등으로 안전상 문제가 발생할 우려가 있다고 판단될 경우 경기 참가를 제한한다.
\end{enumerate}

\section{전륜과 후륜의 트레드(Tread) 비율}
전륜 및 후륜 중 좁은 트레드는 넓은 트레드 대비 75\% 이상의 폭을 가져야 한다.

\chapter{차체 기본규정}

\section{휠 - Wheel}
\begin{enumerate}
  \item 하나의 고정너트를 사용하는 휠 마운트는 반드시 너트 풀림 방지 장치를 해야 하며, 이중너트(잼너트)는 풀림방지장치로 인정되지 않는다.
  \item \label{item:휠 볼트 토크} 휠 볼트, 너트는 98Nm 이상의 토크로 체결될 수 있어야 한다.
  \item 알루미늄 휠 너트는 사용될 수 있으나, 경질 애노다이징 처리되어 있어야 하며 변형이 없어야 한다.
  \item 상용 휠 볼트, 너트 미사용 시 \cref{item:휠 볼트 토크}을 만족해야 한다.
\end{enumerate}

\section{타이어 - Tire}
\begin{enumerate}
  \item 타이어의 트레드 패턴이나 홈은 제조사에 의해 만들어졌거나, 개조된 형태만 사용 가능하며 임의의 트레드 패턴이나 홈의 재가공 및 개조는 금지한다.
  \item 일반적이지 않은 패턴이나 홈을 가진 레인타이어의 경우 검차 시 패턴에 대한 증빙자료가 요구 될 수 있다.
  
  \item 참가팀은 노면의 조건에 따라 아래와 같이 타이어를 선택하여 사용할 수 있다.
    \begin{enumerate}
      \item 드라이 타이어 (Dry Tire)\\
        경주용 레인타이어를 제외한 타이어를 자유롭게 사용할 수 있다.
      \item 레인 타이어 (Rain Tire)\\
        경주용 슬릭 타이어를 제외한 트래드의 깊이가 최소 2.4mm 이상인 타이어를 사용할 수 있다.
    \end{enumerate}
    
  \item 검차를 통과한 차량은 타이어 및 휠의 사이즈나 타입의 변경은 불가하며, 타이어워머(Tire Warmer) 등의 사용은 금지한다.
\end{enumerate}

\section{현가장치 - Suspension System}
\begin{enumerate}
  \item 차량은 드라이버 탑승 상태에서 바퀴의 움직임이 50mm 이상 작동하고 (상, 하 각 25mm) 쇽업소버(Shock Absorber)를 포함한 4바퀴에 모두 작용되는 현가장치를 갖추어야 한다. 현가장치는 작동범위 내에서 상호 간섭되는 부분이 없어야 한다.
  \item 모든 현가장치의 고정부는 외부로 노출되어 있거나 검차 시 확인이 가능해야 한다.
\end{enumerate}

\section{조향장치 - Steering System}
\begin{enumerate}
  \item 조향장치는 반드시 조향 잠김 현상을 방지할 수 있는 조향 제한장치가 있어야 한다. 조향 제한장치는 업라이트(허브, 너클)나 랙\&피니언의 랙에 장착될 수 있고 타이어가 현가장치나 바디 또는 프레임에 닿지 않아야 한다.
  \item 조향휠의 유격은 회전방향 7°이하, 축방향 10mm 이하로 제한한다.
  \item 조향휠은 퀵릴리즈를 통해 조향 칼럼에 연결되어야 하며, 드라이버가 일반적인 운전 자세에서 장갑을 낀 상태로 분리할 수 있어야 한다.
  \item 조향휠은 연결된 상태의 원형이나 타원형에 가까운 형태로 폐곡선(일정한 직선구간을 가진 경우도 허용)을 가져야 하며 H자 형태 또는 단절된 부분이 있는 경우는 사용을 금지한다.
  \item 어떠한 조향각에서도 조향휠은 전방 롤 후프 최상단부 보다 아래에 위치해야 하며, 조향휠을 잡은 손이 메인 롤 후프의 최상부와 전방 롤 후프 최상부에 접하는 연장선 안에 있어야 한다. (\cref{fig:메인 롤 후프 및 전방 롤 후프 일반 규정} 참고)
\end{enumerate}

\section{제동장치 - Brake System}
\begin{enumerate}
  \item 차량은 하나의 페달로 네 바퀴에 모두 작동하는 제동장치를 장착해야 한다.
  \item 두개의 독립적인 유압 회로로 구성되어 누출이나 작동불능 상태일 때 최소한 2개의 바퀴가 제동되도록 해야 한다. 각 유압회로는 각각의 분리된 오일 저장 용기가 있어야 한다.
  \item LSD(Limited Slip Differential)의 사용 또는 라이브엑슬(Live Axle: 차동장치가 없는 독립현가방식) 타입의 사용 시에만 구동축에 한 개의 제동장치 장착이 허용된다.
  \item 제동장치는 추돌로 인한 충격이나 동력전달장치의 파편들로부터 보호되어야 한다.
  \item 보호장치가 없는 플라스틱 브레이크 라인은 금지한다.
  \item 기계식이 아닌 전자식 제동장치는 사용할 수 없다.
  
  \item 제동장치 미작동 감지장치 (Brake Over travel Switch, BOTS)
    \begin{enumerate}
      \item 제동장치가 작동불능상태가 되어 브레이크 페달이 평소 운동거리를 넘어설 때 이를 감지하는 스위치가 작동하여 엔진을 멈출 수 있게 해야 한다. (\cref{fig:제동장치 미작동 감지장치 예시} 참고)
      
      \item 제동장치 미작동 감지장치는 작동 시
        \begin{description}
          \item \string[C-Formula만 해당\string] 제동 등을 제외한 모든 전기 및 전자장치는 전원이 차단되어야 하며 반드시 엔진을 멈출 수 있어야 한다.
          \item \string[E-Formula만 해당\string] \cref{section:차단 회로} 차단 회로(SDC)를 개방해야하며 반드시 모터를 멈출 수 있어야 한다.
        \end{description}
        
      \item 이 스위치는 제동 페달을 놓았을 때나 다시 밟았을 때 엔진이 다시 시동되지 않거나 모터로 전원이 공급되지 않아야 한다.
      \item 제동장치 미작동 감지장치가 정상적으로 작동하는지 확인할 수 있는 사진 자료 등을 검차 시 제출하여야 한다.
    \end{enumerate}
    
    \fig{제동장치 미작동 감지장치 예시}{formula}{0.8}
    
  \item 제동등 (Brake Light)
    \begin{enumerate}
      \item 차량은 태양빛 아래에서도 상태를 식별할 수 있는 제동등을 반드시 장착해야 한다. 제동등의 위치는 뒤에서 보았을 때 차량의 중심선과 드라이버의 어깨선과 뒷바퀴 축 사이와의 교차지점에 있어야 한다. (\cref{fig:제동등 위치} 참고)
      \item 제동등의 면적은 15㎠ 이상이어야 한다. LED의 경우 150mm*30mm 이상으로 제작해야 하며, 100㎟ 당 1개 이상의 LED를 포함하여야 한다.
      \item 제동등은 주 비상정지스위치를 제외한 전원스위치가 OFF된 상태에서도 작동되어야 한다.
      \item 제동등은 작동 시 점멸이 되지 않아야 되며, 제동 스위치에 의해서만 작동되어야 한다.
      \item 어떠한 부품도 제동등을 가리지 않아야 한다.
    \end{enumerate}
    
    \fig{제동등 위치}{formula}{0.6}
\end{enumerate}

\section{잭 지지점 - Jacking Point}
\begin{enumerate}
  \item 퀵 잭(Quick Jack)은 차량이 코스 내에서 주행 불가 상태가 된 경우 차량을 신속히 코스 밖으로 옮기기 위함으로, 퀵 잭에는 차량을 이동시킬 수 있는 바퀴가 장착되어 있어야 하며, 퀵 잭으로 차량이동 시 차량과 지면 간 접촉이 없어야 한다.
  
  \item 차의 무게를 지지하고 퀵 잭을 할 수 있게 하는 잭 지지점(Jacking Point)은 차의 후면에 위치하여야 하며 다음과 같이 요구된다.
    \begin{enumerate}
      \item 차의 중심라인에서 수평과 수직 방향으로 장착되어 차량이 들려졌을 때 안정되어야 한다.
      \item 알루미늄이나 강철 튜브 재질로 25mm 바깥지름으로 둥글게 만든다.
      \item 잭 지지점의 파이프가 최소한 280mm 길이로 장애물 없이 부착되어 있어야 한다.
      \item 잭 지지부는 오랜지색으로 도색되어야 한다.
    \end{enumerate}
    
  \item 잭 지지점 높이는 다음과 같이 요구된다.
    \begin{enumerate}
      \item 지면으로부터 잭 지지점 파이프의 아랫부분 사이가 75mm 이상 100mm 이하이어야 한다.
      \item 잭 지지점의 하단과 지면과의 거리가 200mm일 때, 차의 바퀴는 지면에 닿지 않아야 한다.
    \end{enumerate}
  \item 검차 또는 패독에서 정비 시 에어로 부품의 손상을 막기 위해 별도의 퀵 잭을 사용할 수 있다.
\end{enumerate}

\chapter{드라이버 보호구조}

\section{드라이버 보호구조 - Driver’s Cell} \label{section:드라이버 보호구조}
\begin{enumerate}
  \item 드라이버는 차량의 전복과 추돌로부터 보호되어야 한다. 드라이버 보호를 위해 드라이버 공간을 확보하여야 하며, 드라이버 공간은 벌크헤드로부터 메인 롤 후프 또는 방화벽, 좌우 주 구조물 안쪽을 말한다.
  \item 사용된 재료는 \cref{section:재료의 최소 요구조건}를 만족해야 한다.
  \item A-클래스 참가 팀이 조직위원회에서 허용한 프레임을 사용할 경우 \cref{section:드라이버 보호구조}, \cref{section:재료의 최소 요구조건}는 만족한 것으로 인정한다.
  
  \fig{드라이버 보호구조의 정의}{formula}{0.8}
  
  \item 드라이버 공간의 프레임은 측면도 상의 각 부재의 연결점을 기준으로 트러스(Truss) 구조 만족을 통해 외부 하중을 견딜 수 있어야 한다. 연결된 부재의 중심선은 하나의 점에서 만나야 한다. (\cref{fig:드라이버 보호구조의 트러스 구조} 참고)
  
  \fig{드라이버 보호구조의 트러스 구조}{formula}{0.8}
  
  \item 프레임 주 구조물(메인 롤 후프, 전방 롤 후프, 측면 보호 구조, 롤 후프 지지대, 벌크헤드 등)에는 검사구멍을 제외한 어떠한 구멍도 허용되지 않는다. 바닥판, 방화벽 등 격리용 판 부재 및 부품 장착 시 별도의 브래킷(Bracket)을 사용해야 한다.
  \item 운전석 공간 확보를 위해 \cref{fig:운전석 공간 확인용 측정도구}와 같은 측정도구를 사용한다. 측정도구는 메인 롤 후프 최상단부에서부터 수직방향으로 상단 부재의 하단까지 통과되어야 한다.
  
  \fig{운전석 공간 확인용 측정도구}{formula}{0.5}
  
  \item 드라이버 다리 공간 확보를 위해 \cref{fig:운전석 단면 확인용 측정도구}과 같은 측정도구를 사용한다. 측정도구는 수직으로 세워진 상태로 전방 롤 후프에서부터 페달의 가장 뒷면에서 100mm 떨어진 지점까지 수평으로 통과되어야 한다. 이때, 위치조정이 가능한 페달 시스템의 경우 차량의 가장 앞쪽에 위치한 상태로 측정한다.
  
  \fig{운전석 단면 확인용 측정도구}{formula}{0.5}
\end{enumerate}

\section{안전 구조 대응물 - Safety Structure Equivalency} \label{section:안전 구조 대응물}
\begin{enumerate}
  \item \cref{item:기본 철강 재료}에 제시된 기본 재료와 치수 이외의 재료를 사용하는 경우 기준 철강 재료와 동일하거나 높은 안전율을 증명하는 검증된 자료를 제출해야 하며, 최종 사용 허가는 조직위원회에서 결정된다.
  \item 재료 및 구조에 대한 검토 자료는 부록에 있는 구조 대응물 양식(Structural Equivalency Form: SEF)에 맞게 작성하여 제출해야 한다.
  \item 대체 재료의 최소 요구조건 부합 여부 확인을 위해 별도의 자료를 요청할 수 있다.
\end{enumerate}

\section{재료의 최소 요구조건 - Minimum Material Requirements} \label{section:재료의 최소 요구조건}
프레임을 구성하는 재료의 최소 요구조건은 아래와 같다. 

\begin{enumerate}
  \item 기본 철강 재료 (Baseline Steel Material) \label{item:기본 철강 재료}
    \begin{enumerate}
      \item 프레임 주 구조물 즉, 메인 롤 후프, 전방 롤 후프, 측면 보호 구조, 롤 후프 지지대, 벌크헤드 등은 다음의 요구사항을 만족하여야 한다.
      \item 원형의 일반 또는 탄소강으로 최소 0.1\% 이상의 탄소를 함유하여야 하며, 규격은 아래 표에 제시된 규격이상을 사용하여야 하며, 사용된 기본 철강 재료의 물성치는 다음에 제시된 값 보다 커야 한다.
      \[항복\ 강도(\sigma_{y}) = 305\ \mathrm{MPa},\ 인장\ 강도(\sigma_{u}) = 365\ \mathrm{MPa}\]
      \item 검차 시 증빙자료(성분표, 시험성적서)는 구조 대응물 양식과 함께 제출한다.

      \begin{table}[H]
        \centering
        \begin{tblr}{
          width = 0.9\linewidth,
          colspec = {|X[c,m]|Q[c,m]|Q[c,m]|},
          rows = {ht=1.0\baselineskip},
          row{1} = {bg=gray!30, font=\bfseries},
          hlines,
          vlines
        }
          사용 위치
            & 원형-외경×두께
            & 각형-가로×세로×두께 \\
          메인 롤 후프 \& 전방 롤 후프, 어깨벨트 마운트
            & 25mm × 1.8mm
            & 불가 \\
          측면, 전방 충격 보호 구조, 롤 후프 지지대
            & 25mm × 1.6mm
            & 불가 \\
          나머지 프레임 재료
            & 20mm × 1.2mm
            & 25 × 25 × 1.6mm \\
        \end{tblr}
      \end{table}
    \end{enumerate}

    \begin{itemize}
      \item 나머지 프레임: 프레임 주 구조물을 제외한 운전석의 구조를 형성하는 프레임으로 하중을 받거나 전달하는 프레임.
      \item 주의: 파이프의 재료를 합금강이나 다른 재료를 사용할 경우에도 파이프 두께를 얇은 것을 사용할 수 없다.
    \end{itemize}
    
  \item 대체 재료 \label{item:대체 재료}\\
    합금강 재료, 알루미늄합금, 복합재료를 사용할 경우에는 \cref{chapter:드라이버 보호구조 대체재료}에서 제시한 기준에 부합하여야 한다.
\end{enumerate}

\section{롤 후프 - Roll Hoops}
차량의 전복 시 드라이버를 보호하는 역할을 한다.

\begin{enumerate}
  \item 메인 롤 후프 및 전방 롤 후프(Main and Front Hoops) 일반 규정
    \begin{enumerate}
      \item 가장 키가 큰 드라이버와 일반인 평균의 상위 95\%(운전석 윗면에서 헬멧을 쓴 머리까지의 거리는 1,000mm로 가정)에 해당하는 키의 드라이버가 평상시대로 운전석에 앉아 안전벨트 및 보호장비를 착용했을 때, 메인 후프의 최상부와 전방 롤 후프 최상부에 또는 메인 후프 지지대 하단부의 연장선에서부터 드라이버의 헬멧 사이의 간격이 50mm 이상이어야 한다. (\cref{fig:메인 롤 후프 및 전방 롤 후프 일반 규정} 참고)
        \begin{enumerate}
          \item 일반인 평균 상위 95퍼센트 드라이버 모델 (\cref{fig:일반인 평균 상위 95퍼센트 드라이버 모델} 참고)
          \begin{enumerate}
            \item 운전석을 가장 뒤쪽으로 위치시킨다.
            \item 페달을 가장 앞 쪽(차량전방)으로 위치시킨다.
            \item 하단 직경 200mm원의 중심에서 페달까지는 최소 915mm 이상이 되어야 한다.
            \item 상단 직경 200mm원은 좌석 등받이에 위치시킨다.
            \item 상단 직경 300mm원은 헤드레스트와 최대 25mm까지 이격시킬 수 있다.
          \end{enumerate}
        \end{enumerate}
        
      \item 파이프 벤딩의 최소 반경은 파이프의 중심선에서부터 측정하였을 때 파이프 외경의 3배를 넘어야 한다.
      \item 벤딩 부위에 벤딩으로 인한 주름이 없어야 하며 벤딩에 의해 최소 외경이 벤딩 전 원래 외경보다 15\% 이상 줄어들지 않아야 한다.
      \item 메인 롤 후프 및 전방 롤 후프를 프레임 주 구조물에 확실히 부착하기 위해 적절한 보강판과 삼각형 구조의 보강재 사용을 권장한다.
      
      \fig{메인 롤 후프 및 전방 롤 후프 일반 규정}{formula}{0.8}
      
      \begin{itemize}
        \item 어떠한 조향각에서도 조향휠은 전방 롤 후프 최상단부 보다 아래에 위치해야 하며, 조향휠을\\
        잡은 손이 메인 롤 후프의 최상부와 전방 롤 후프 최상부에 접하는 연장선 안에 있어야 한다.
      \end{itemize}
      
      \fig{일반인 평균 상위 95퍼센트 드라이버 모델}{formula}{0.8}
      
      \item 메인 롤 후프 및 전방 롤 후프의 직선부 내에서 상단 바깥쪽에 파이프의 두께를 검사하기 위한 직경 5mm의 검사구멍을 아래와 같이 표시된 곳에 뚫어 놓아야 한다. (\cref{fig:메인 롤 후프 및 전방 롤 후프 검사구멍 위치} 참고)
      \fig{메인 롤 후프 및 전방 롤 후프 검사구멍 위치}{formula}{0.8}
    \end{enumerate}
    
  \item 메인 롤 후프 (Main Roll Hoop) \label{item:메인 롤 후프}
    \begin{enumerate}
      \item 메인 롤 후프는 \cref{item:기본 철강 재료}과 \cref{item:대체 재료}을 만족하는 재료로 제작해야 한다.
      \item 메인 롤 후프는 하나의 끊어지지 않은 원형 파이프로 한쪽(왼쪽) 바닥프레임에서 드라이버 위를 지나 반대쪽(오른쪽) 바닥 프레임까지 이어져 있어야 한다.
      \item 어떠한 복합소재도 메인 롤 후프의 재료로 사용할 수 없다.
      \item 측면에서 볼 때, 메인 롤 후프의 프레임 주구조물 윗부분 기울기는 전방 그리고 후방으로 수직축과 10° 이내이어야 한다. (\cref{fig:메인 롤 후프 기울기 및 폭} 참고)
      
      \fig{메인 롤 후프 기울기 및 폭}{formula}{0.8}
      
      \item 메인 롤 후프가 프레임 주 구조물과 만나는 좌, 우 지점의 폭(파이프와 파이프의 안쪽을 측정)은 380mm 이상 떨어져 있어야 한다.
      
      \item 모노코크 구조의 경우에도 메인 롤 후프는 하나의 끊어지지 않은 원형 파이프로 한쪽(왼쪽) 밑바닥에서 드라이버 위를 지나 반대쪽(오른쪽) 바닥면까지 이어져 있어야 한다. \label{item:모노코크 메인 롤 후프}
        \begin{enumerate}
          \item 볼트-너트 같은 기계적 체결용 요소를 사용하여 메인 롤 후프를 확실히 부착시켜야 한다. 사용된 볼트의 직경은 최소 8mm, 강도구분 8.8 이상의 볼트 2개 이상을 사용한다.
          \item \cref{section:재료의 최소 요구조건}에 해당하는 원형 파이프에 부착되어 있을 때 동일한 강도를 가지는 것을 증명하는 구조 대응물 양식을 제출하여야 한다.
          \item 메인 롤 후프가 부착될 고정용 판은 최소한 3.0mm 이상의 두께를 갖는 철강 판(또는 동일한 강도의 알루미늄) 위에 용접해야 한다. 또한 같은 두께의 철강 판으로 모노코크 구조의 반대편을 덧대야 한다. 이는 하중분산과 모노코크 구조의 찌그러짐을 방지하기 위한 것이다. 위의 구조가 \cref{section:재료의 최소 요구조건}에 해당하는 원형 파이프에 결합되어 있을 때 동일한 강도를 가지는 것을 증명하는 구조 대응물 양식을 제출하여야 한다.
        \end{enumerate}
    \end{enumerate}
    
  \item 전방 롤 후프 (Front Roll Hoop) \label{item:전방 롤 후프}
    \begin{enumerate}
      \item 전방 롤 후프는 \cref{item:기본 철강 재료}과 \cref{item:대체 재료}을 만족하는 재료로 제작해야 한다.
      \item 전방 롤 후프는 하나의 끊어지지 않은 원형 파이프로 한쪽(왼쪽) 바닥프레임에서 조향축 위를 지나 반대쪽(오른쪽) 바닥 프레임까지 이어져 있어야 한다.
      \item 어떠한 복합소재도 전방 롤 후프의 재료로 허용하지 않는다.
    \end{enumerate}
\end{enumerate}

\section{롤 후프 지지대 - Roll Hoop Bracing}
\begin{enumerate}
  \item 메인 롤 후프 지지대 (Main Roll Hoop Bracing) \label{item:메인 롤 후프 지지대}
    \begin{enumerate}
      \item 메인 롤 후프 지지대는 \cref{item:기본 철강 재료}과 \cref{item:대체 재료}을 만족하는 재료로 제작해야 한다.
      \item 메인 롤 후프 지지대는 두 개 이상의 원형파이프로 좌, 우 양쪽에서 메인 롤 후프를 지지해야 한다.
      \item 메인 롤 후프가 뒤쪽으로 기울어졌다면 뒤쪽에 메인 롤 후프 지지대가 있어야 하며 메인 롤 후프가 앞쪽으로 기울어졌다면 앞쪽에 메인 롤 후프 지지대가 있어야 한다.
      \item 메인 롤 후프 지지대는 메인 롤 후프의 최상부에 부착되어야 하나 만약 그것이 힘들다면 메인 롤 후프의 최상부에서 160mm 이내의 위치에 메인 롤 후프와 30° 이상의 각도로 부착해야 한다.
      \item 메인 롤 후프 지지대는 구부러지지 않은 직선의 파이프이어야 한다.
    \end{enumerate}
    
  \item 전방 롤 후프 지지대 (Front Roll Hoop Bracing) \label{item:전방 롤 후프 지지대}
    \begin{enumerate}
      \item 전방 롤 후프 지지대는 \cref{item:기본 철강 재료}과 \cref{item:대체 재료}을 만족하는 재료로 제작해야 한다.
      \item 전방 롤 후프 지지대는 두 개 이상의 원형파이프로 좌, 우 양쪽에서 전방 롤 후프를 지지해야 한다.
      \item 전방 롤 후프 지지대는 전방 롤 후프로부터 벌크헤드까지 연장되어 드라이버의 다리를 보호할 수 있어야 한다.(벤딩불가) 모노코크 구조가 전방 롤 후프 지지대를 대신할 경우 동일한 강도를 가진다는 것을 증명할 수 있는 구조 대응물 양식을 제출한다.
      \item 전방 롤 후프 지지대는 전방 롤 후프의 최상부에 부착되어야 하나 만약 그것이 힘들다면 전방 롤 후프의 최상부에서 50mm 이내의 위치에 부착되어야 한다.
      \item 전방 롤 후프가 후방으로 수직축과 10° 이상 기울어진다면 전방 롤 후프 뒤쪽에 또 다른 지지대를 설치해야 하며, 지지대는 \cref{item:기본 철강 재료}과 \cref{item:대체 재료}을 만족하는 재료로 제작해야 한다.
    \end{enumerate}
    
  \item 다른 방식의 지지대 사용 시 요구조건
    \begin{enumerate}
      \item 모노코크 구조 사용 시, 메인 롤 후프 지지대나 전방 롤 후프 지지대는 3mm 이상의 두께를 갖는 철강 판(또는 동일한 강도의 알루미늄) 위에 용접해야 하며 같은 두께의 철강 판으로 모노코크 구조의 반대편을 덧대어 직경 8mm, 강도구분 8.8 이상의 지름을 갖는 볼트로 체결하여야 한다. (\cref{fig:모노코크 구조의 메인 롤 후프 지지대 고정 방법} 참고) \label{item:모노코크 지지대 접합부}
      
      \fig{모노코크 구조의 메인 롤 후프 지지대 고정 방법}{formula}{0.3}
      
      \item 탈착식 롤 후프 지지대의 사용
        \begin{enumerate}
            \item 마운트는 2개 이상의 판으로 이루어진 Double-Lug방식 (\cref{fig:탈착식 롤후프 지지대 - Double-Lug 방식} 참고) 혹은 Sleeved butt joint (\cref{fig:탈착식 롤후프 지지대 - Sleeved butt joint 방식} 참고)를 사용하여야 한다. 마운트 판은 4.5mm 이상의 두께를 갖는 철강 판이어야 하며 폭은 25mm 이상, 길이는 최대한 짧아야 한다.
            \fig{탈착식 롤후프 지지대 - Double-Lug 방식}{formula}{0.7}
            \fig{탈착식 롤후프 지지대 - Sleeved butt joint 방식}{formula}{0.4}
            \item 모든 마운트는 Capping Plate 또는 Doubler 방식으로 보강해야 한다. 만약 Doubler 방식이 사용된다면 1.6mm 이상의 두께를 가져야 하며 둘레의 2/3(120°) 이상을 감싸야 한다.
            \item 사용된 핀 또는 볼트는 직경 10mm 강도구분 10.9 이상이어야 한다.
            \item 원형 구가 들어간 로드엔드(볼 엔드)의 사용은 금지한다.
            \item Sleeved butt joint(인접한 접합 부분을 슬리브(축을 끼우는 관)에 끼우는 방법)를 사용하는 경우 슬리브는 최소한 80mm 이상의 길이를 갖고, 접합된 면으로부터 양쪽으로 40mm 이상 관의 둘레를 모두 감싸야 한다. 슬리브의 재질은 철강재질로 두께는 최소한 관의 두께와 같아야 한다. 볼트는 반드시 6mm, 강도구분 10.9 이상이어야 한다. 슬리브와 관의 구멍들은 반드시 볼트에 의해 확실히 고정해야 한다.
        \end{enumerate}
    \end{enumerate}
\end{enumerate}

\section{전방 충돌 보호 구조 - Impact Attenuator}
\begin{enumerate}
  \item 벌크헤드 (Bulkhead)\\ \label{item:벌크헤드}
    벌크헤드란 프레임의 가장 앞부분의 면을 지칭하는 말로서 페달 주위의 프레임 구조, 프레임 주 구조물 가장 앞부분을 말한다. 
    \begin{enumerate}
      \item 벌크헤드는 \cref{item:기본 철강 재료}을 만족하는 재료로 제작해야 한다.
      \item 사각형이나 원형 등의 폐쇄된 형태를 가져야 하고 프레임 주 구조물에 확실히 부착(용접 등)되어야 한다.
      \item 벌크헤드는 프레임 주 구조물로 좌, 우측 각각 3개 이상의 프레임 부재(전방 롤 후프 지지대, 프레임 하단 부재, 대각선 부재 등)로 연결되어야 한다.
      \item 위치조절이 가능한 페달 시스템은 최대한 앞으로 위치해 놓고 검사한다.
      \item 벌크헤드 최상부에서 50mm 이내에 프레임 주 구조물이 부착되어야 하며, 충돌 시 충격완화장치의 밀림을 방지하기 위해 벌크헤드 모서리 두 점을 연결하는 대각선 부재를 가지고 있어야 한다.
      \item 모든 모노코크 구조는 \cref{section:재료의 최소 요구조건}에 명시된 파이프로 만든 트러스형 프레임과 동등한 강도를 가져야 한다. 각 팀은 \cref{section:재료의 최소 요구조건}를 만족하는 항복-인장강도를 가진다는 것을 증명하기 위해 \cref{section:안전 구조 대응물}의 구조 대응물 양식에 맞게 작성하여 제출한다. \label{item:모노코크 벌크헤드}
    \end{enumerate}
    
  \item 충격완화장치 구조의 요구조건 \label{item:충격완화장치}
    \begin{enumerate}
      \item 재료의 조건과 부착방법
        \begin{enumerate}
          \item 어떤 한계를 가지고 차량의 속도를 확실히 줄일 수 있어야 한다. (완전히 변형되었을 때 차량이 정지하여야 한다.) 충격완화장치의 성능을 증명하기 위해 실험을 권장하며, 실험이 불가능할 시 유한요소해석을 통한 충격완화장치의 안전성을 입증해야 하며, 이는 \cref{section:안전 구조 대응물}의 안전 구조 대응물 양식에 첨부하여야 한다.
          \item 충격완화장치는 총중량이 3,000N인 차량의 전방에 부착되었다고 가정하고 7m/s의 속도로 단단한 벽으로 돌진, 충돌할 때 평균 감속이 20g를 초과하지 않고 최대 감속이 40g를 초과하지 않으며 전체 에너지 흡수량이 7,350J를 초과해야 한다. (테스트 데이터를 권장하나 유한요소해석을 통한 데이터 제출도 가능하다.) (\cref{fig:유한요소해석 예제} 참고)
          \item \cref{item:충격완화장치 예시}을 충족하는 충격완화장치를 제작, 장착하였을 경우에는 \cref{item:충격완화장치}을 증빙하는 데이터 제출은 하지 않아도 된다. (\cref{fig:충격완화장치 예시} 참고)
          \item 해석이나 시험자료는 대회 개최년도를 포함하여 3년 이내의 자료만 허용하며, 증빙자료를 제출해야 한다.
          
          \fig{유한요소해석 예제}{formula}{0.5}
        \end{enumerate}
      \item 충격완화장치 앞면은 카울을 제외한 어떠한 물체도 있어서는 안 된다.
    \end{enumerate}
    
  \item 충격완화장치 예시 \label{item:충격완화장치 예시}\\
    충격완화장치는 두개의 면으로 구성되어진다. 이 두면은 앞 차축과 뒤 차축에 평행하고 지면에 수직한 면이다.
    \begin{enumerate}
      \item 앞면: 충격 완화장치 앞면은 가로 200mm 이상, 세로 100mm 이상의 직사각형이어야 한다.
      \item 뒷면: 충격완화장치의 뒷면은 벌크헤드의 앞면과 동일한 면으로 사용할 수 있다. 충격완화장치의 뒷면이 벌크헤드의 앞면(파이프의 외곽선기준)보다 작을 경우 벌크헤드의 앞면을 2t의 철제판 또는 4t의 알루미늄판으로 막고 사용해야 한다.
      \item 앞면과 뒷면 사이의 간격: 앞면과 뒷면 사이의 간격은 200mm 이상 떨어져 있어야 한다.
      \item 재질은 2mm 이상의 철제판 또는 4mm 이상의 알루미늄판으로 제작해야 한다.
      \item 충격완화장치의 각 모서리 이음새는 용접 등으로 연결되어야 한다.
      \item 충격완화장치는 설계의도 외의 변형이 생긴 경우 사용할 수 없다.
      
      \fig{충격완화장치 예시}{formula}{0.4}
    \end{enumerate}
    
  \item 보호되어야 하는 부품의 배치 (Non-Crushable Objects)
    \begin{enumerate}
      \item 모든 보호되어야 하는 부품(축전지, 마스터 실린더, 브레이크 오일 리져버 탱크, 페달 등)은 벌크헤드보다 뒤쪽에 있어야 한다.
      \item 충격완화장치 안에 보호되어야 하는 부품이 있어서는 안 된다. 
    \end{enumerate}
\end{enumerate}

\section{측면 충돌 보호 구조} \label{section:측면 충돌 보호 구조}
드라이버는 운전석에 정상적으로 앉아있는 상태에서 측면 충돌로부터 보호되어야 한다. 측면 충돌 보호 구조물에 사용된 재료의 조건은 \cref{section:재료의 최소 요구조건}를 만족하여야 한다.
\begin{enumerate}
  \item 트러스형 파이프 프레임 (Tube Frame)
    \begin{enumerate}
      \item 좌, 우측 각각 최소한 3개 이상의 파이프를 사용하여 측면 충돌 보호 구조를 만들어야 한다.
      \item 드라이버가 보통의 자세로 앉았을 때 드라이버의 양쪽에 위치해야 한다.
      
      \item 다음에 명시된 3개의 부재는 \cref{section:재료의 최소 요구조건}에 명시된 재료의 조건을 만족해야 한다.
        \begin{enumerate}
          \item 상단 부재(Upper Member): 메인 롤 후프와 전방 롤 후프를 연결해야 하며, 드라이버가 탑승한 상태에서 지면에서부터 부재의 중심선이 300 \string~ 350mm 사이의 높이에 있어야 한다. (\cref{fig:측면 충돌 보호 구조물 정의 및 조건} 참고)
          \item 대각선 부재(Diagonal Member)  상단 부재와 하단 부재를 연결하고 메인 롤 후프와 전방 롤 후프와도 연결되어야 한다. 다수의 부재로 구성된 경우, 대각선 부재의 한 쪽 끝단이 연결되어 있어 트러스구조를 이루어야 한다.
          \item 하단 부재(Lower Member): 하단 부재는 메인 롤 후프의 최하단부와 전방 롤 후프의 최하단부를 연결해야 한다. 하단 부재는 보통 바닥 프레임 구조물의 일부이다. 하단 부재는 측면 보호 구조물의 일부이므로 \cref{section:재료의 최소 요구조건}에 명시된 재료를 사용해야 한다. 그렇지 않을 경우는 동일하거나 더 높은 강도를 가진다는 것을 증명할 수 있는 자료를 \cref{section:재료의 최소 요구조건}의 구조 대응물 양식에 맞게 작성하여 제출하여야 한다. 하단부재의 경우 벤딩하여 사용할 수 있으나 절단 후 용접 등으로 연결하여 사용할 수는 없다.
          
          \fig{측면 충돌 보호 구조물 정의 및 조건}{formula}{0.8}
        \end{enumerate}
    \end{enumerate}
\end{enumerate}

\section{머리충격 흡수패드}
\begin{enumerate}
  \item 머리충격 흡수패드는 드라이버의 머리를 보호하기 위해 필수 장착되어야 한다. 모든 드라이버가 운전자세로 앉았을 때 헬멧의 뒷부분 중앙이 머리충격 흡수패드의 중앙에 와야 한다.
  \item 머리충격 흡수패드는 스티로폼(Soft), 스펀지 등 탄력을 지닌 재료로 최소 240㎠의 면적과 40mm 이상의 두께를 가져야 하고 헬멧으로부터 25mm 이하의 거리를 유지하되 헬멧이 패드에 닿아 패드가 압축된 상태에 놓이면 안 된다.
  \item 흡수패드는 차량 후방 충격을 견딜 수 있도록 견고하게 부착되어야 하며 장착된 상태에서 흔들림이 없어야 한다.
\end{enumerate}

\section{프레임 패딩}
\begin{enumerate}
  \item 메인 롤 후프, 메인 롤 후프 지지대 또는 프레임 주 구조물의 어떠한 부위든 드라이버의 헬멧과 닿는다면 최소 10mm 두께의 스티로폼(Soft), 스펀지 등 탄력을 지닌 재료로 둘러싸야 한다.
\end{enumerate}

\section{날카로운 부위 처리}
\begin{enumerate}
  \item 드라이버, 팀원, 경기진행요원, 심사위원 등의 안전을 위협할 수 있는 날카로운 부분은 금지되며, 반드시 안전하게 마무리 처리를 해야 하며 차량 바디 노즈의 반경은 35mm 이상으로 한다.
\end{enumerate}

\chapter{드라이버 보호구조 대체재료} \label{chapter:드라이버 보호구조 대체재료}

\section{대체 재료의 사용 - Alternative Tubing and Material}
\begin{enumerate}
  \item 다음 사항을 만족할 때 \cref{section:재료의 최소 요구조건}에 제시된 파이프의 치수와 재료를 대체하여 사용할 수 있다. \label{item:대체 재료 조건}
    \begin{enumerate}
      \item 메인 롤 후프, 메인 롤 후프 지지대
        \begin{enumerate}
          \item 부하 하중: Fx(길이방향) = 6.0kN, Fy(측면방향) = 5.0kN, Fz(상하방향) = -9.0kN
          \item 하중부하점: 메인 롤 후프 최고점
          \item 최대 허용 변형: 25mm
          \item 하중부과 후 어떠한 구조물도 파괴가 일어나지 않아야 한다.
        \end{enumerate}
      \item 전방 롤 후프
        \begin{enumerate}
          \item 부하 하중: Fx = 6.0kN, Fy = 5.0kN, Fz = -9.0kN
          \item 하중부하점: 전방 롤 후프 최고점
          \item 최대 허용 변형: 25mm
          \item 하중부과 후 어떠한 구조물도 파괴가 일어나지 않아야 한다.
        \end{enumerate}
      \item 측면 보호 구조
        \begin{enumerate}
          \item 부하 하중: Fx = 0kN, Fy = 7.0kN, Fz = 0kN
          \item 하중부하점: 측면 보호 구조물의 센터를 기준으로 최대 지름 250mm의 원으로 하중 부과
          \item 최대 허용 변형: 25mm
          \item 하중부과 후 어떠한 구조물도 파괴가 일어나지 않아야 한다.
        \end{enumerate}
      \item 벌크헤드 및 벌크헤드 지지 구조물
        \begin{enumerate}
          \item 부하 하중: Fx = 120kN, Fy = 0kN, Fz = 0kN
          \item 하중부하점: 전방 충격완화장치와 벌크헤드 사이
          \item 최대 허용 변형: 25mm
          \item 하중부과 후 어떠한 구조물도 파괴가 일어나지 않아야 한다.
        \end{enumerate}
    \end{enumerate}

  \item 대체 재료에 대한 구조 대응물 양식은 \cref{section:안전 구조 대응물}에 맞게 작성하고 \cref{item:대체 재료 조건}을 만족한다는 것을 증명하기 위해 구조해석, 실험 등을 통한 자료를 첨부하여 제출해야 한다.
    \begin{enumerate}
      \item 재료 유형 (구매영수증, 선적서류 또는 기증서) 및 재료 특성에 대한 문서를 제출해야 한다.
      \item 복합재 레이업 기술 및 사용된 구조 재료 (섬유 종류, 무게 및 수지 종류, 층 수, 코어 재료 및 금속인 경우 표면 재료)에 대한 세부 정보를 제출해야 한다.
      \item \cref{item:기본 철강 재료}의 표에서 명시된 최소 요구 사항에 대한 해당하는 항목과 대체 재료의 동등성을 입증하는 계산을 제출한다. 에너지 소산, 항복 및 굽힘, 좌굴 및 장력의 궁극적 강도에 대한 동등성 계산을 제출해야 한다.
    \end{enumerate}

  \item 준 등방성 레이업\\
    레이업 평면의 모든 방향을 따라 동일한 섬유 강도와 강성을 가진 레이업을 말한다. 
    \begin{enumerate}
      \item 레이업이 0 / 90 / +45 / -45 방향으로 동일한 섬유 특성과 질량을 갖는 경우 레이업은 준 등방성으로 간주될 수 있다.
    \end{enumerate}

  \item 복합재 테스트 \label{item:복합재 테스트}
    \begin{enumerate}
      \item 주요 구조 적층 테스트\\
      팀은 차량기술 규정집에서 요구하는 섀시 부분에 대해 복합재와 동일한 시험편을 제작하여 3점 굽힘 테스트를 수행해야 한다.
        \begin{enumerate}
          \item 시험편(테스트 패널)은 다음을 충족해야 한다.
            \begin{itemize}
              \item 275mm x 500mm 측정
              \item 400mm의 스팬 거리로 지지
              \item 윗면과 아랫면의 표면적이 동일해야 한다.
              \item 적층된 층이 표면 외 별도의 내부 코어층을 가지고 있어야 한다.
            \end{itemize}
            
          \item 안전 구조 대응물 서류에는 다음이 포함되어야 한다.
            \begin{itemize}
              \item 3점 굽힘 테스트의 데이터
              \item 시험 샘플 사진
              \item 안전 구조 대응물 서류에서 스팬 거리를 기록한 측정값을 보여주는 테스트 샘플 및 테스트 설정 (\cref{fig:복합재 테스트} 참고)
            \end{itemize}
            
            \fig{복합재 테스트}{formula}{0.8}
            
          \item 섀시의 1차 구조 영역에 해당하는 복합재 패널 동등성을 계산할 목적으로 안전 구조 대응물 서류 공식에 의해 강성, 항복 강도, 극한 강도 및 흡수 에너지 특성을 도출하기 위해 테스트 패널 결과를 사용해야 한다.
          \item 측면 충격 복합재에 대한 테스트 패널 결과는 좌굴 탄성률, 극한강도, 에너지 흡수에 대해 아래 비교 시험에 따라 테스트 된 2개의 측면 충격 강관과 동등한 안전 구조 대응물 서류 공식을 사용하여 계산으로 표시해야 한다.
        \end{enumerate}

      \item 비교 시험
        \begin{enumerate}
          \item 팀은 2개의 측면 충격 보호 프레임에 해당하는 기본 철강 재료(\cref{item:기본 철강 재료})로 동등한 시험을 실시하여 시험 장비의 모든 규정 준수를 고려하고 기준 튜브의 흡수 에너지값을 설정해야 한다.
          \item 스틸 튜브는 최소 19mm의 변위로 테스트해야 한다.
          \item 흡수된 에너지의 계산은 하중의 시작에서 19mm의 변위까지의 힘 대 변위의 적분을 사용한다.
        \end{enumerate}

      \item 시험 실시
        \begin{enumerate}
          \item \cref{item:복합재 테스트}에 따라 패널 / 튜브를 테스트하는 데 사용되는 하중 부하 장치는 다음과 같아야 한다.
            \begin{itemize}
                \item 금속
                \item 반경 50mm
            \end{itemize}
            
          \item 하중 부하 장치는 가장자리에 하중이 부여되는 것을 방지하기 위해 시험편의 폭 바깥으로 돌출되어야 한다. (\cref{fig:복합재 테스트} 참고)
          \item 하중 부하 장치와 테스트할 시험편 사이에 다른 재료를 놓아서는 안 된다.
        \end{enumerate}

      \item 주변부 전단 시험
        \begin{enumerate}
          \item 편평한 복합재 시험편을 통해 25mm 직경의 단면이 평면인 펀치를 밀거나 당기는 데 필요한 힘을 측정하여 주변부 전단 테스트를 완료해야 한다.
          
          \item 시험편은 다음을 충족해야 한다.
            \begin{itemize}
                \item 최소 100mm x 100mm 측정
                \item 실제 적용에 사용된 것과 동일한 코어 및 외피 두께
                \item 동일한 재료와 공정을 사용하여 제조
            \end{itemize}
            
          \item 고정 장치는 펀치와 축 정렬된 32mm 구멍을 제외하고 전체 시험편을 지지해야 한다.
          \item 시험편을 고정 장치에 고정해서는 안 된다.
          \item 고정구의 펀치 및 구멍의 가장자리에는 최대 반경 1mm까지의 선택적 필렛이 포함될 수 있다.
          \item 시험 설정의 힘 및 변위 데이터와 사진이 구조 대응물 양식의 첨부서류에 포함되어야 한다.
          \item 하중-변형 곡선의 첫 번째 최고점은 표면 전단 강도로 사용한다.
          \item 프론트 벌크헤드의 전단 강도는 직경이 25mm인 단면에 대해 4kN 이상이어야 한다.
          \item 측면 보호 구조물의 전단 강도는 25mm인 단면의 경우 7.5kN 이상이어야 한다.
        \end{enumerate}

      \item 추가 시험\\
        복합재 패널이 준 등방성 레이업이 아닌 경우: 
        \begin{enumerate}
          \item 3점 굽힘 시험의 결과가 0° 레이업 방향으로 할당한다.
          \item 모노코크는 구조 대응물 양식에서 동등성에 부합하는 단면에 수직인 방향으로 시험한다.
          \item 가장 약한 방향의 재료의 물성치는 구조 대응물 양식의 첨부서류에 포함된 가장 강한 방향 물성치의 50\% 이상이어야 한다.
        \end{enumerate}

      \item 랩 조인트 테스트
        \begin{enumerate}
          \item 랩 조인트 테스트는 서로 부착된 두 개의 복합재 시험편으로 구성된 조인트를 분리하는 데 필요한 힘을 측정한다.
            \begin{itemize}
                \item 접착된 부분에 순수전단이 작용할 수 있도록 평행한 방향으로 당긴다.
            \end{itemize}
            
          \item 시험편은 다음을 충족해야 한다.
            \begin{itemize}
                \item 밀착 방향과 평행한 접착면을 갖는다.
                \item 실제 모노코크에 사용된 것과 동일한 외피(섬유적층부분) 두께를 가짐
                \item 동일한 재료와 공정을 사용하여 제조
            \end{itemize}

          \item 시험 설정의 힘 및 변위 데이터와 사진이 구조 대응물 양식의 첨부서류에 포함되어야 한다.
          \item 접착부분의 전단 강도는 외피(섬유적층부분, 스킨)부분의 인장강도보다 커야 한다.
        \end{enumerate}
    \end{enumerate}

  \item 좌굴 계수 – 동등한 평판 계산
    \begin{enumerate}
      \item 지정된 경우, 섀시의 EI는 복합재의 중립 축에 대한 섀시와 동일한 평판 패널의 EI로 계산한다.
      \item 패널의 곡률과 섀시의 기하학적 단면은 계산에서 무시한다.
      \item \cref{item:복합재 테스트}을 참고하지 않고 실제 형상을 고려한 EI를 계산할 수 있다. 
    \end{enumerate}

  \item 메인 롤 후프와 메인 롤 후프 지지대, 전방 롤 후프는 복합소재를 사용할 수 없다.
\end{enumerate}

\chapter{안전 규정}

\section{안전벨트 - Safety Belt} \label{section:안전벨트}
  모든 드라이버는 아래 사항에 맞는 안전벨트를 착용하여야 한다. 팔 안전벨트 또한 구비해야 한다. 드라이버의 완전한 구속을 위해 안전벨트는 항상 꽉 조여진 상태로 착용될 수 있어야 한다.
  \begin{enumerate}
    \item 안전벨트의 요구조건
      \begin{enumerate}
        \item 6점식 이상의 안전벨트를 사용하여야 한다.
        \item 드라이버의 상반신이 지면에서 60˚ 이하 기울었을 때 다리 사이 벨트를 반드시 추가해야 한다.
        \item 허리벨트와 어깨벨트는 하나의 풀림장치를 공유하며, 이는 금속과 금속으로 연결되는 퀵 릴리스 타입의 걸쇠이어야 한다.
        \item 공식 인증 제품으로 손상이 없어야 한다.
        
        \item 인증 규격은 하기의 규격 혹은 그 이후의 인증 규격을 인정한다.
          \begin{itemize}
            \item SFI Specification 16.1
            \item SFI Specification 16.5
            \item FIA specification 8853/98
            \item FIA specification 8853/2016
          \end{itemize}
      \end{enumerate}
    \item 벨트와 벨트 마운트
      \begin{enumerate}
        \item 허리벨트, 어깨벨트와 다리사이벨트(Anti-Submarine Strap)는 차량의 프레임 주 구조물에 부착되어야 한다.
        \item 프레임 주 구조물은 \cref{section:재료의 최소 요구조건}의 요구사항을 만족하는 구조물이다. 함석판, 알루미늄 판 등으로 만든 바닥판이나 등판에 볼트로 고정된 벨트 마운트는 허가되지 않는다.
        \item 벨트를 조절했을 때 벨트의 어떠한 부분도 운전석 영역 밖으로 돌출되어서는 안 되며, 섀시의 회전 부분이나 동력전달계통 등과 접촉되어서는 안 된다.
        \item 안전벨트의 차체 연결부는 두께 2mm 이상, 폭 25mm 이상의 철강재료의 브라켓 형태로, \cref{fig:벨트 마운트 브라켓}의 A \string~ D 중 최소 폭이 홀 직경의 150\%를 초과하여야 한다.\\
        (파이프에 직접 감아 마운팅하는 경우 별도의 브라켓을 사용하지 않아도 된다.)
        
        \fig{벨트 마운트 브라켓}{formula}{0.4}
        
        \item 허리벨트 및 어깨벨트 고정을 위해 사용하는 볼트는 직경 8mm 이상, 강도 8.8 이상을 사용해야 한다.
        \item 강도 8.8 미만의 아이볼트 사용을 금지한다.
        \item 안전벨트 고정을 위하여 볼트(아이볼트/아이너트) 또는 너트를 용접하여 사용하는 것은 금지한다.
        \item 모노코크 구조의 경우는 \cref{section:안전 구조 대응물}에 제시된 구조 대응물 양식을 제출한다. \label{item:모노코크 안전벨트 접합부}
        \item 벨트는 방화벽을 기준으로 운전석 쪽에 위치해야 한다.
        \item 벨트 브라켓과 볼트는 아래와 같이 인장하중 및 전단하중을 견딜 수 있도록 설치하여야 한다. (\cref{fig:벨트 설치 예} 참고)
        
        \fig{벨트 설치 예}{formula}{0.5}
      \end{enumerate}
    \item 허리벨트 요구조건
      \begin{enumerate}
        \item 허리벨트는 골반보다 약간 아래 엉덩이뼈 주위를 지나가야 한다. 어떠한 상황에서도 복부 부근에 벨트가 지나가서는 안 된다. (\cref{fig:허리벨트 조건} 참고)
        \item 허리벨트는 바닥 프레임에서부터 드라이버 시트를 지나 드라이버의 엉덩이 아래에서부터 엉덩이뼈를 완전히 감싸고 다시 운전석 반대쪽을 관통하여 반대쪽 바닥 프레임에 부착되어야 한다.
        \item 허리벨트 브래킷은 최소 25mm x 1.8mm(외경 x 두께)의 파이프에 고정되어야 한다.
        
        \item 벨트 착용 시 측면에서 바라볼 때, 허리벨트 각도와 고정점의 위치는 아래와 같다. (\cref{fig:허리벨트 조건} 참고)
          \begin{enumerate}
            \item 허리벨트 고정점은 시트 뒤끝 지점에서 0 \string~ 75mm 사이에 있어야 한다.
            \item 허리벨트는 지면과 45° \string~ 65°를 이루어야 한다. (\cref{fig:허리벨트 조건} 좌측 참고)
            \item 드라이버의 상반신이 지면에서 60˚ 이하인 경우 벨트는 지면과 60° \string~ 80°를 이루어야 한다. (\cref{fig:허리벨트 조건} 우측 참고)
          \end{enumerate}
        
        \fig{허리벨트 조건}{formula}{0.8}
      \end{enumerate}
    \item 어깨벨트 요구조건
      \begin{enumerate}
        \item 어깨벨트는 높은 감속력에서 어깨뼈 부상을 최소화하기 위해 어깨벨트의 각도는 드라이버 어깨 뒤쪽의 수평선을 기준으로 위 방향으로 약 10° 아래 방향으로 20° 되는 직선 사이에 있어야 한다. (\cref{fig:어깨벨트 조건} 참고)
        \item 벨트의 차체 고정부의 수직 높이는 바닥에서 측정했을 때 어깨선 미만에 위치해 있어야 한다.
        \item 분리된 어깨벨트만이 허용된다. (Y형, H형 어깨벨트는 허용되지 않는다.) 어깨벨트는 벨트 고리가 있어 길이를 조절할 수 있게 해야 한다. 어깨벨트 마운트는 최소 180mm에서 최대 230mm 이내의 간격을 두고 장착하여야 한다. (\cref{fig:어깨벨트 조건} 참고)
        \item 어깨벨트 브래킷은 최소 25mm x 1.8mm(외경 x 두께)의 파이프에 고정되어야 한다.
        
        \fig{어깨벨트 조건}{formula}{0.8}
      \end{enumerate}
    \item 다리사이벨트 (Anti-Submarine Strap)
      \begin{enumerate}
        \item 6점식 안전벨트 시스템의 다리사이벨트는 프레임 부재에 부착되어야 하며, 다리벨트, 어깨벨트와 함께 한 번에 분리할 수 있는 금속 걸쇠에 연결되어야 한다. (6점식 안전벨트의 6점은 차체 마운팅 포인트 수를 기준으로 한다.)
        \item 다리사이벨트의 마운트는 운전자 허벅지의 바깥쪽에 위치하여 벨트가 허벅지를 감싸며 고정되어야 한다. \cref{fig:다리사이 벨트 조건}의 형태를 만족할 경우 마운트의 간격은 최소 100mm 이상을 만족하여야 하며, 아닐 경우 최소 200mm의 폭을 가져야 한다.
        \item 5점식 안전벨트 사용을 금지한다.
        
        \fig{다리사이 벨트 조건}{formula}{0.8}
      \end{enumerate}
  \end{enumerate}

\section{드라이버 안전 장비 - Drivers Equipment} \label{section:드라이버 안전 장비}
아래의 안전장비들은 운전 시 이외에도 정지 상태에서 드라이버가 탄 상태에서 엔진 시동 중에는 반드시 착용하고 있어야 하며, 인증 규격은 하기의 규격 혹은 그 이후의 인증 규격을 인정한다.
\begin{enumerate}
  \item 헬멧
    \begin{enumerate}
      \item 쉴드가 포함된 풀 페이스 헬멧(Full Face Helmet)만 허용한다. 턱과 얼굴 안면이 노출되는 헬멧의 사용은 금지한다.
      \item 헬멧을 착용할 때는 항상 턱걸이 끈을 고정하여야 한다.
      \item 헬멧 쉴드를 제외한 헬멧의 앞부분이 오픈되는 구조의 헬멧의 사용은 금지한다.
      
      \item 공식인증 용품을 사용해야 한다.
        \begin{itemize}
          \item Snell K2005, K2010, K2015, M2005, M2010, M2015, SA2005, SA2010, SAH2010, SA2015, EA2016
          \item SFI Specs 31.1/2005, 31.1/2010, 31.1/2015, 41.1/2005, 41.1/2010, 41.1/2015
          \item FIA Standards FIA 8860-2004, FIA 8860-2010, FIA 8860-2018, FIA 8859-2015
        \end{itemize}
      
      \fig{풀페이스형 헬멧}{formula}{0.7}
    \end{enumerate}
    
  \item 레이싱복, 방화복
    \begin{enumerate}
      \item 방염 소재로 된 손목 끝까지 덮는 긴 팔 상의와 발목 끝까지 덮는 바지의 착용을 의무화한다.
      
      \item 공식 인증된 용품을 사용해야 한다.
        \begin{itemize}
          \item SFI 3.2A/5 (or higher ex: /10, /15, /20)
          \item FIA Standard 1986
          \item FIA Standard 8856-200,8856-2000 이상
        \end{itemize}
    \end{enumerate}
    
  \item 장갑
    \begin{enumerate}
      \item 방염 소재의 장갑을 착용하여야 하며 구멍이 있는 장갑은 금지된다.
      
      \item 공식 인증된 용품을 사용해야 한다.
        \begin{itemize}
          \item SFI Spec 3.3/5
          \item FIA Standard 8856-200,8856-2000 이상
        \end{itemize}
    \end{enumerate}
    
  \item 눈 보호 장비 (헬멧 쉴드)\\
    눈 보호 장비는 충격에 강한 소재로 만들어져 있어야 하며 경기 중에는 항상 닫힌 상태에 있어야 한다.
    
  \item 신발
    \begin{enumerate}
      \item 방염 소재의 신발을 착용하여야 하며 샌들과 같은 구멍이 있는 신발의 착용은 금지한다.
      \item 신발은 끈이 외부로 노출되어 안전사고를 유발할 위험이 있어서는 안 된다.
      \item 공식 인증된 용품을 사용해야 한다.
        \begin{itemize}
          \item SFI Spec 3.3
          \item FIA Standard 8856-200,8856-2000 이상
        \end{itemize}
    \end{enumerate}
    
  \item 팔 안전벨트
    \begin{enumerate}
      \item 팔 안전벨트는 차량이 어떠한 상황에 있더라도 드라이버의 팔이 운전석 공간 내에 있도록 하여야 한다.
      \item 팔 안전벨트는 안전벨트의 풀림 장치에 장착되어 안전벨트를 풀었을 때 풀려야 한다.
      \item 탈출 시에 팔 안전벨트가 드라이버 손목에 장착되어 있어도 무관하다.
      \item SFI Spec 3.3 인증 또는 이와 동일한 요구사항을 만족해야 한다.
    \end{enumerate}
  \item 바라클라바
    \begin{enumerate}
      \item 방염 소재의 바라클라바를 착용해야 한다.
      
      \item 공식 인증된 용품을 사용해야 한다.
        \begin{itemize}
          \item SFI Spec 3.3
          \item FIA Standard 8856-200,8856-2000 이상
        \end{itemize}
    \end{enumerate}
\end{enumerate}

\section{드라이버 시야} \label{section:드라이버 시야}
드라이버가 정상적으로 앉았을 때 머리를 돌리거나 거울을 사용하여 좌우로 총 200도 이상의 시야가 확보되어야 한다.

\section{드라이버 탈출} \label{section:드라이버 탈출}
\begin{enumerate}
  \item 모든 드라이버는 5초 이내에 차량의 옆으로 탈출할 수 있어야 한다.
  \item 탈출시간은 드라이버가 완전히 앉아 손은 조향휠을 잡고 모든 안전 장비를 착용한 상태에서 시작한다. 탈출시간은 드라이버의 양쪽 발이 땅에 닿았을 때까지의 소요시간을 측정한다.
\end{enumerate}

\section{드라이버 공간의 폐쇄} \label{section:드라이버 공간 폐쇄}
\begin{enumerate}
  \item 모든 차량은 드라이버와 지면을 완전히 분리시키는 하나 또는 하나 이상의 판을 가져야 한다. 만일 여러 장의 판이 사용된다면 판과 판 사이의 틈이 3mm를 넘지 않아야 한다.
  \item 판은 발에서부터 방화벽까지 연결되어야 하며, 도로의 파편으로부터 다리와 몸통을 보호해야 한다.
  \item 바닥판은 반드시 볼트, 리벳 등 기계적인 결합을 통해 견고하게 장착되어야 하며, 케이블 타이나 와이어를 이용한 고정, 피스의 사용은 허용하지 않는다.
  \item 드라이버 공간 내의 작동되는 부품의 경우 최대 20mm 이하의 여유 공간만 허용된다.
\end{enumerate}

\section{화재 보호 장치 - Fire Extinguishers} \label{section:화재 보호 장치}
\begin{enumerate}
  \item 방화벽 (Firewall) \label{item:방화벽}
    \begin{enumerate}
      \item 방화벽은 동력장치, 연료장치, 윤활장치, 냉각장치, 축전지로부터 드라이버를 완전히 격리하여 드라이버를 보호할 수 있어야 한다.
      \item 방화벽은 두께 1mm 이상의 금속판으로 설치하여야 한다.\\
      (단, 수냉각장치 주변의 방화벽은 끓는물에 변형되지 않는 비금속 재질도 허용한다.)
      \item 전선이나 케이블을 통과시키기 위해 구멍을 뚫었다면 구멍의 남은 틈새를 완전히 메워야 한다. 또한 방화벽이 다수의 판으로 연결되어도 무방하지만 드라이버 보호를 위해 틈새가 있으면 안 되며 테이프에 의한 판끼리의 연결은 금지한다.
      \item 연료통 및 연료장치와 배기시스템 사이에 별도의 방화벽을 두어 연료통을 보호하여야 한다.
      \item 압력탱크와 배기장치 사이에 별도의 방화벽을 두어야 한다.
      \item 방화벽은 시트로 사용할 수 없다.
      \item 연료통이 엔진 또는 배기장치 위쪽에 위치할 경우 별도의 연료통 드립팬을 설치하여야 한다.
      \item \string[E-Formula\string] 방화벽은 모든 고전압 시스템 및 배선과 휠에 위치하지 않은 모든 구동 시스템(Tractive System)으로부터 드라이버를 완전히 격리하여 드라이버를 보호할 수 있어야 한다.
      \item \string[E-Formula\string] 방화벽은 두 개의 층으로 이루어져야 한다. 구동시스템을 바라보는 층은 1mm 이상의 알루미늄으로 이루어지고 차체에 접지되어야 한다. 드라이버를 바라보는 층은 \cref{section:난연성 재료}를 만족하는 난연성의 전기적 절연 재료로 이루어져야 하고 \cref{item:접지}을 만족해야 한다. 이 층은 CFRP로 이루어져서는 안 된다. \label{item:e-formula 방화벽}
    \end{enumerate}
    
  \item 소화기
    \begin{enumerate}
      \item 각 팀은 화재에 대비하여 소화기를 준비하여야 하고, 검차 시 확인받아야 한다.
        \begin{itemize}
          \item \string[C-Formula\string]: 1kg 이상 분말 ABC 소화기 최소 2개
          
          \item \string[E-Formula\string]: 리튬배터리용 소화기 (아래 소회기 품명) 최소 1개 및 1kg 이상 분말 ABC 소화기 최소 1개
            \begin{itemize}
              \item AMEREX: B270, B272
              \item DOKA: Model 6L 이상
              \item AFT: 10/01, 50/02 이상
            \end{itemize}
          \item 할론소화기의 사용은 금지한다.
        \end{itemize}
        
      \item 하나는 패독에 비치하고 다른 하나는 팀원이 소화기를 들고 차량 이동 시 동행하여야 한다. 또한 차량을 정비하거나 세팅할 시 차량의 앞·뒤 대각선으로 2개 이상의 소화기를 비치하여야 한다.
      \item 소화기에는 각 팀의 이름과 출전번호를 부착하여야 한다.
    \end{enumerate}
\end{enumerate}

\section{비상 정지 스위치} \label{section:비상 정지 스위치}
\string[ C-Formula만 해당, E-formula는 \cref{section:주 비상 정지 스위치}, \cref{section:보조 비상 정지 스위치}를 따른다. \string]

\begin{enumerate}
  \item 비상 정지 스위치는 기계적으로 작동하는 스위치여야 한다.
  
  \item 각 차량은 다음과 같은 비상 정지 스위치가 있어야 한다.
    \begin{enumerate}
      \item 엔진을 동력원으로 사용하는 차량
        \begin{enumerate}
          \item 비상 정지 스위치는 작동 시 엔진을 멈출 수 있어야 한다.
          \item 주 비상 정지 스위치 1개와 보조 비상 정지 스위치 1개가 있어야 한다.
        \end{enumerate}
    \end{enumerate}
    
  \item 주 비상 정지 스위치 \label{item:주 비상 정지 스위치}
    \begin{enumerate}
      \item 외부 인원이 쉽게 조작할 수 있도록 드라이버의 어깨 높이에 위치해야 한다.
      \item 드라이버의 오른쪽 어깨 높이에 1개의 주 비상 정지 스위치가 있어야 한다.
      \item 주 비상 정지 스위치 작동 시 모든 전기·전자 장치의 전원을 차단해야 한다.
    \end{enumerate}
    
  \item 보조 비상 정지 스위치 \label{item:보조 비상 정지 스위치}
    \begin{enumerate}
      \item 드라이버가 쉽게 조작할 수 있는 위치에 견고하게 부착되어야 한다.\\
        (Push-Pull 또는 Push-Rotate 방식의 스위치 권장)
      \item 보조 비상 정지 스위치 작동 시 제동등을 제외한 모든 전기·전자장치의 전원은 차단되어야 한다.
    \end{enumerate}
    
  \item 가로 25mm, 세로 45mm 이상의 흰색 바탕의 사각형 안에 파란색 삼각형에 빨간색 불꽃으로 구성된 스티커를 비상 정지 스위치와 근접하게 부착하여야 한다. (\cref{fig:주 비상 정지 스위치} 참고)
  \item 주 비상 정지 스위치는 작동 시 레버가 분리되는 스위치를 사용해야 한다. (\cref{fig:주 비상 정지 스위치} 참고)
  
  \fig{주 비상 정지 스위치}{formula}{0.5}
\end{enumerate}

\section{저전압 축전지 – Low Voltage Batteries, LV Batteries} \label{section:저전압 축전지}
\begin{enumerate}
  \item 모든 축전지는 차체에 안전하게 고정되어야 하며, 차량 프레임 내부에 위치시키고 진동이나 충격에 전후, 좌우 및 상하로 이동이 되지 않도록 충분히 보호되어야 한다.
  \item 모든 축전지 단자 및 +단자의 경우 반드시 절연 처리를 하여야 한다. (ex. 스타트모터, 스타트모터 릴레이 및 메인롤후프 파이프와 근접한 킬스위치 등)
  \item 리튬 기반의 저전압 축전지는 다음의 조건을 충족해야 한다.
    \begin{enumerate}
      \item 견고하고 \cref{section:난연성 재료}를 만족하는 난연성인 케이스에 내장되어야 한다. \label{item:저전압 축전지 케이스}
      \item 방화벽으로 드라이버와 분리되어야 한다.(\cref{item:방화벽} 참고)
      \item 보호회로가 장착되어 있어야 한다. \label{item:저전압 축전지 보호회로}
      \item 국내에서 정식으로 판매되는 상용품을 사용할 경우 \cref{item:저전압 축전지 케이스}, \cref{item:저전압 축전지 보호회로}을 만족한 것으로 인정한다.
    \end{enumerate}
\end{enumerate}

\chapter{동력장치}

\section{동력장치 - Powertrain} \label{section:동력장치}
\begin{enumerate}
  \item 동력원의 허용 범위
    \begin{enumerate}
      \item 내연기관 동력원: 4행정 710cc 이하의 가솔린 엔진
      \item 전기장치 동력원 (E-Formula): 80kW 이하의 모터
    \end{enumerate}
    
  \item 내연기관 제한 요소
    \begin{enumerate}
      \item 흡기와 배기 계통 등 모든 요소의 개조는 허용하되 반드시 조직위원회에 신고하여야 한다.
      \item 경기위원회의 결정에 의해 참가 차량의 엔진을 분해하여 개조 여부를 검사 할 수 있다.
    \end{enumerate}
    
  \item 내연기관의 흡기 제한 장치 (Restrictor)
    \begin{enumerate}
      \item 1개 이상의 실린더를 가지고 있는 엔진의 경우라도 모든 흡입구는 하나의 구멍을 지난 뒤 나뉘어야 하며, 자연흡기로 사용하는 한국산 엔진을 제외하고는 하나의 스로틀 바디만 허용한다.
      \item 흡기장치에는 흡기 제한을 위한 하나의 최소 직경을 가진 흡기 제한 장치(Restrictor)를 가져야 한다.
      \item 한국산 엔진을 자연흡기로 사용할 경우 흡기제한장치 장착을 하지 않아도 되나, 과급기를 사용할 경우 외산엔진의 흡기제한장치 규정을 따라 장착하여야 한다.
      \item 흡기제한장치의 내경은 완전한 원형이어야 하며, 위치는 반드시 스로틀바디 또는 기화기 이후에 위치하여야 한다. (\cref{fig:흡기제한장치 위치} 참고) 검차 시 흡기제한장치의 내경 확인이 가능하도록 스로틀바디 및 흡기필터 등을 제거 후에 검차에 임하여야 한다.
      \item 흡기제한장치는 300cc 이하의 엔진을 사용할 경우 최대 직경은 23mm$\varnothing$를 넘을 수 없다.
      \item 흡기제한장치는 300cc 초과의 엔진을 사용할 경우 최대 직경은 20mm$\varnothing$를 넘을 수 없다.
      \item 과급기를 사용할 경우, 흡기제한장치의 위치를 과급기 앞에 위치해야 한다. (\cref{fig:과급기 사용 시 흡기제한장치 위치} 참고)
    \end{enumerate}

  \begin{table}[H]
    \centering
    \begin{tblr}{
      width = 0.9\linewidth,
      colsep = 12pt,
      colspec = {|Q[c,m]|X[c,m]|Q[c,m]|X[c,m]|},
      rows = {ht=1.3\baselineskip},
      row{1} = {bg=gray!30, font=\bfseries},
      hlines,
      vlines
    }
      배기량
        & 형식
        & 최소직경
        & 최소직경의 위치 \\
      \SetCell[r=3]{bg=gray!30,valign=t} 300 CC 이하
        & 한국산엔진 (자연흡기)
        & 제한 없음
        & 제한 없음 \\
        & 한국산엔진 (과급기사용)
        & \SetCell[r=2]{c} 23mmØ
        & \SetCell[r=2]{c} 스로틀바디 이후 설치 \\
        & 외산엔진
        & 
        &  \\
      \SetCell[r=3]{bg=gray!30,valign=t} 300 CC 초과
        & 한국산엔진 (자연흡기)
        & 제한 없음
        & 제한 없음 \\
        & 한국산엔진 (과급기사용)
        & \SetCell[r=2]{c} 20mmØ
        & \SetCell[r=2]{c} 스로틀바디 이후 설치 \\
        & 외산엔진
        & 
        &  \\
    \end{tblr}
  \end{table}

  \fig{흡기제한장치 위치}{formula}{0.5}
  \fig{과급기 사용 시 흡기제한장치 위치}{formula}{0.7}

  \item 전기장치 동력원 제한 요소
    \begin{enumerate}
      \item 모터의 사용 수량에는 제한이 없다.
      \item 최대 사용 전력은 80kW를 넘을 수 없다.
      \item 사용할 수 있는 축전지의 최대 전압은 DC 600V이다.
    \end{enumerate}
    
  \item 변속장치와 동력전달장치 (Transmission \& Drive)\\
    모든 방식의 동력전달 장치와 변속장치를 사용할 수 있다.
    
  \item 동력전달장치 보호판 (Drivetrain Shield)
    \begin{enumerate}
      \item 벨트, 체인, 스프라켓 등 고속 회전하는 부품들이 노출되어 있는 형태의 차량은 운행 중 그 부품들이 파손되어 파편이 튈 때 드라이버나 주위에 서 있는 사람, 연료라인, 브레이크 라인, 고전압 시스템(E-Formula) 등을 보호하기 위해 보호판을 장착해야 한다.
      \item 보호판은 구멍이 난 소재를 사용해서는 안 된다.
      \item 만약 프로펠러나 냉각용 팬이 사용된다면 팬과 신체의 접촉이 불가능 하도록 배치한 경우를 제외하고, 외부로 노출된 경우에는 손가락 보호망을 설치해야 한다. 단, 푸시바를 사용하여 차량의 동력전달장치가 주위 사람에게 위험하지 않다면 손가락 보호망의 미사용이 가능하다.
      
      \item 구동 장치 보호판
        \begin{enumerate}
          \item 보호판은 2mm 두께 이상의 철판을 반드시 사용해야 하며(다른 어떤 재료의 사용도 불가) 체인이나 벨트의 3배 이상의 폭을 가져야 하며, 최소 80mm 이상 되어야 한다.
          \item 구동 장치 보호판의 중심선은 체인의 중심에 위치하여야 하며, 체인이나 벨트가 회전하는 바깥쪽 부분에 설치되어야 한다. 구동 장치 보호판의 끝부분은 아래와 같이 구동장치의 최하단 수평선 아래 까지 위치해야 하며, 드라이버를 향해 열린 면이 없어야 한다. (\cref{fig:구동 장치 보호판의 끝부분} 참고)
        \end{enumerate}
        \fig{구동 장치 보호판의 끝부분}{formula}{0.6}
        
      \item 보호판 체결 방법\\
        모든 보호판은 직경 6mm, 강도구분 8.8 이상의 볼트를 사용하여야 한다. 보호판은 체인, 벨트와 항상 나란히 있도록 확실히 고정해야 한다.\\
        동력장치의 위치로 인해 구동장치 보호판이 하나의 판으로 구성하기가 힘들거나, 다른 부품의 간섭으로 체인이나 벨트의 파손이 외부로 빠져나가는 것을 막을 수 있다면 보호판의 연장으로 보거나 여러 개의 판으로 연결하여 구성할 수 있다.
    \end{enumerate}
    
  \item 액체 누출 방지
    \begin{enumerate}
      \item 엔진과 변속기는 액체 누출 방지를 위해 밀폐되어야 한다.
      \item 냉각장치와 윤활장치의 누출을 대비하기 위한 별도의 캐치 캔을 장착하여야 하며, 각각의 캐치 캔은 액체 전체 부피의 10\% 이상을 담을 수 있거나, 1리터 이상의 용량을 가져야 한다.
      \item 캐치 캔은 끓는 물에 변형되지 않아야 되며, 고정 시 캐치캔의 찌그러짐 등의 변형이 없어야 하며 움직임이 없도록 확실히 고정되어야 한다.
      \item 캐치 캔은 방화벽 뒤, 드라이버의 어깨높이보다 아래쪽에 위치 하여야 하며, 케이블 타이나 테이프에 의한 고정은 금지한다.
      \item 냉각수용 캐치 캔은 냉각장치로부터 유입되는 호스와 별도로 캐치 캔으로부터 누출되는 냉각수의 배출을 위해 최소내경 3mm 이상의 호스를 별도로 설치하여야 하며, 프레임의 가장 아래 부분까지 내려오게 고정시켜야 한다.
    \end{enumerate}
  \item 수냉각시스템 및 냉각수의 요구조건
    \begin{enumerate}
      \item 수냉각 시스템의 구성품은 냉각수의 최고온도 및 압력에서 충분히 견딜 수 있어야 한다.
      \item 수냉식 엔진은 반드시 불순물이 없는 순수한 물을 사용하여야 한다.
      \item 부동액과 워터펌프 윤활유 등의 첨가제는 사용을 금지한다
    \end{enumerate}
  \item 시동장치(Starter)\\
    내연기관 차량은 스타트 모터에 의한 자력 시동만 허용한다.
\end{enumerate}

\section{배기장치 - Muffler and Exhaust System} \label{section:배기장치}
\begin{enumerate}
  \item 배기장치\\
    차량은 배기를 확장시켜 배기압을 줄여주는 배기장치, 즉 머플러를 장착해야 한다.
    
  \item 배기장치의 위치
    \begin{enumerate}
      \item 배기구는 배기가스가 드라이버 방향으로 배출되지 않게 해야 한다.
      \item 배기 파이프는 운전석을 통과해서는 안 된다. 머플러의 배기구는 지면으로부터 600mm 이내의 높이에 있어야 하며 뒤축 중심으로부터 후측 방향으로 450mm 이내에 있어야 한다.
      \item 소음측정을 위해 배기구 끝의 방향이 바닥을 향해서는 안 된다.
      \item 메인 롤 후프의 앞쪽에 위치한 어떠한 배기장치도 사람이나 차량에 닿지 않도록 차량 바디의 옆면에서 돌출된 부분이 없어야 한다.
    \end{enumerate}
    
  \item 소음 제한
    \begin{enumerate}
      \item 110dBC 이하로 소음 제한을 둔다.
      \item 소음측정은 차량검사 시에 진행한다.
      \item 소음측정 시 엔진 회전수 검출을 위해 차량은 타코미터를 장착하여야 한다. 단, 엔진 회전수 검출장비를 휴대할 경우 타코미터를 장착하지 않아도 된다.
      \item 소음측정 장소는 개방되어 있으며 측정방법은 차량의 배기구 끝으로부터 0.5m, 45° 각도, 배기구에서 수평한 위치에서 측정한다.
      \item 하나 이상의 배기구가 있을 경우 각각의 배기구에서 소음측정 후 높은 결과를 사용한다.
      \item 소음측정을 위한 엔진의 회전속도는 중립상태에서 피스톤 속도가 차량 및 이륜차량에서 사용되는 엔진의 경우 910m/min, 범용엔진의 경우 730m/min을 기준으로 500 rpm단위로 반올림한 회전속도에서 측정된다. 엔진별 테스트 회전수는 아래의 식으로 산출된다.
      \[
        \text{테스트 회전수}
        =
        \frac{910\;(\text{or }730)\times1000}{2\times\text{행정 (mm)}}
        \;\mathrm{rpm}
      \]
      \item 소음이 기준치보다 초과된 차량은 수정하여 재검사를 받을 수 있으며, 통과하지 못할 시 경기에 참여할 수 없다.
    \end{enumerate}
\end{enumerate}

\chapter{연료장치}

\section{연료 - Fuel} \label{section:연료}
모든 경기에서 사용하는 연료는 조직위원회에서 인정하는 연료만 사용할 수 있다. 연료에는 어떠한 첨가제도 허용되지 않는다.

\begin{enumerate}
  \item 연료 온도 변경 금지\\
    연료 온도는 어떠한 이유에서든 변경되어서는 안 된다.
  \item 연료 첨가제 금지\\
    엔진 흡입구에는 공기와 연료 이외의 어떠한 물질도 흡입되어서는 안 된다.
  \item 연료의 주입\\
    참가팀은 정해진 장소에서만 연료를 주입할 수 있다.
\end{enumerate}

\section{연료 장치 - Fuel System} \label{section:연료장치}
\begin{enumerate}
  \item 연료 탱크
    \begin{enumerate}
      \item 연료 탱크의 용량은 제한이 없으나, 내구레이스 경기 도중에는 재급유를 할 수 없다.
      \item 연료 주입구 주위는 부품에 의해 가려져 연료 주입 시 방해되어서는 안 된다.
      \item 연료 주입구가 배기장치 등 열원에 근접해 있을 시 연료 주입구에 드립 팬을 설치하여야 하며 드립팬에 고인 연료를 바닥으로 배출할 수 있는 호스가 있어야 한다.
    \end{enumerate}
    
  \item 연료장치의 위치 및 고정
    \begin{enumerate}
      \item 연료통은 프레임 주 구조물과 메인 롤 후프 지지대 밖으로 노출되어서는 안 된다. 연료통은 드라이버 공간 안에 위치해서는 절대 안 된다.
      \item 연료주입구는 주 구조물의 외부, 메인 롤 후프 최상단과 4개 타이어의 바깥 모서리가 이루는 공간 안에 있어야 하며, 바디를 탈거하지 않고 주유 및 주유량 확인이 가능하여야 한다. (\cref{fig:연료통 주입 연료량 확인 조건} 참고)
      \item 연료통은 진동을 흡수할 수 있는 부시류(고무 등)를 사용하여 고정해야 한다.
    \end{enumerate}
    
  \item 주입 연료량 확인\\
    연료주입구 상단에는 내연료성을 가진 연료량 확인용 투명호스를 장착하여야 하며, \cref{fig:연료통 주입 연료량 확인 조건}과 같이 수직방향으로 125mm 이상의 길이를 가져야 하며 연료량 확인용 투명호스가 내연료성을 가지고 있음에 대한 증빙자료를 검차 시 제출하여야 한다.

  \fig{연료통 주입 연료량 확인 조건}{formula}{0.5}

  \item 연료통, 기화기 가스 배출관
    \begin{enumerate}
      \item 연료 탱크와 기화기의 가스 배출관은 심한 코너링이나 급가속 시에 연료가 새지 않도록 설계되어야 한다.
      \item 차량을 45° 기울였을 때 기화기나 연료 탱크에서의 연료 누출이 없어야 한다.
      \item 연료 및 가스 배출관은 탱크가 뒤집혀도 연료 누출을 막기 위한 체크 밸브를 갖추고 있어야 한다. 연료탱크와 기화기의 가스배출관은 캐치 탱크에 연결되면 안 되며 차체 밖으로 나와 있어야 한다.
    \end{enumerate}
    
  \item 연료라인, 라인 장착과 보호
    \begin{enumerate}
      \item 연료탱크와 엔진 간 연료라인(공급과 리턴)은 플라스틱 사용을 금지하며, 휘발유를 연료로 사용하는 차량, 오토바이 등 기타 동력장치 또는 기계장치에 사용된 연료호스를 사용해야 한다. 그 외 호스는 연료라인으로 사용을 금지한다. (\cref{fig:내연료성 연료호스} 참고)

      \fig{내연료성 연료호스}{formula}{0.5}

      \item 만약 고무 연료라인 또는 호스를 사용한다면 호스가 빠지지 않도록 끝이 둥근 고리형태나 가시모양의 피팅을 사용하여야 하며 연료라인 고정용으로 설계된 클램프를 사용하여야 한다. 클램프는 다음 3가지 중요한 조건을 만족하여야 한다. (\cref{fig:연료라인용 둥근고리 피팅과 가시모양 피팅 및 클램프} 참고)
        \begin{enumerate}
          \item 360° 전체를 감싸야 한다.
          \item 너트, 볼트를 사용하여 조여야 한다.
          \item 호스 가장자리는 조임으로 인해 절단, 파손되지 않아야 한다.
        \end{enumerate}

      \fig{연료라인용 둥근고리 피팅과 가시모양 피팅 및 클램프}{formula}{0.8}

      \item 웜기어 방식의 호스 클램프 사용 시에는 연료라인이 손상되지 않도록 연료라인 외부에 보호 플라스틱을 사용하고 그 겉에 웜기어 방식의 호스 클램프를 사용할 수 있다.
      \item 연료라인은 차량에 안전하게 고정되어야 한다. 모든 연료라인은 운전석을 통과해서는 안 되며 모든 충돌과 파손으로부터 보호되어야 한다.
    \end{enumerate}
    
  \item 저압 분사(Low Pressure Injection, LPI) 시스템\\
    저압 분사 시스템은 연료 압력 10Bar (145psi) 이하의 시스템을 말한다. 대부분의 포트연료분사 (Port Fuel Injection, PFI) 시스템은 저압 분사 시스템이다.
    
    \begin{enumerate}
      \item 연료라인(연료펌프 사용): 연료라인은 유연성이 있어야 하며, 아래 사항을 포함해야 한다.
        \begin{enumerate}
          \item 금속 매쉬호스를 사용하고 끝은 재사용이 가능한 나사산이 난 피팅으로 고정하여야 한다.
          \item 강화 고무호스를 사용할 경우 연료라인을 파손으로부터 보호할 수 있는 형태의 클램프와 같이 사용되어야 한다.\\
          ※ 주의: 금속 매쉬 호스는 클램프를 사용할 수 없다.
        \end{enumerate}
        
      \item 연료레일: 인젝션 시스템의 연료레일은 엔진실린더블록, 실린더헤드 또는 흡기 매니폴드, 흡기파이프 등에 기계적으로 안전하게 장착되어야 한다. (호스클램프, 플라스틱 타이, 안전와이어 제외)
      \item 흡기 매니폴드: 연료 인젝션 엔진의 흡기 매니폴드는 엔진블록 또는 실린더헤드에 안전하게 고정되어야 한다.
    \end{enumerate}
    
  \item 고압분사(High Pressure Injection, HPI) / 직분사(Direct Injection, DI) 시스템\\
    직분사 (Direct Injection, DI) 엔진의 경우 연료라인 내의 압력을 올려주기 위해 고압생성(Boost) 펌프가 사용된다. 이 경우 고압생성 펌프부터 인젝터까지를 고압 연료라인이라 한다.
    
    \begin{enumerate}
      \item 고압 연료라인
        \begin{enumerate}
          \item 유연성이 있는 스테인레스 스틸 재질의 매쉬호스를 사용해야 한다.
          \item 고압 연료라인 100mm마다 엔진 구조물(실린더 헤드, 실린더 블록 등)에 기계적인 구속을 통해 고정해야 한다.
        \end{enumerate}
        
      \item 저압 연료라인
        \begin{enumerate}
          \item 연료라인은 유연성이 있어야 하며, 금속 매쉬호스를 사용하고 끝은 재사용이 가능하도록 나사산이 난 피팅으로 고정하여야 한다.
          \item 강화 고무호스를 사용할 경우 연료라인을 파손으로부터 보호할 수 있는 형태의 클램프와 같이 사용되어야 한다.
        \end{enumerate}
        
      \item 고압 연료 펌프\\
        연료펌프는 엔진구조물(실린더 헤드, 실린더 블록 등)에 확실히 고정되어야 한다.
    \end{enumerate}
    
  \item 흡기, 연료 시스템 장착 요구조건\\
    냉각시스템을 제외한 모든 연료 저장, 공급에 관련된 시스템, 엔진 흡기, 연료 제어 시스템(에어클리너를 제외한 흡기 시스템)은 메인 롤 후프 최상단과 4개 타이어의 바깥 모서리가 이루는 공간 안에 있어야 한다. 또한 모든 연료탱크와 흡기시스템은 측면 충돌로부터 보호되어야 한다. (\cref{fig:연료 시스템과 엔진 흡기 시스템 위치조건} 참고)

  \fig{연료 시스템과 엔진 흡기 시스템 위치조건}{formula}{0.6}
\end{enumerate}

\section{스로틀과 스로틀 작동 방법 - Throttle, Throttle Actuation} \label{section:스로틀 작동}
\begin{enumerate}
  \item 기화기 / 스로틀 바디: 장착 필수\\
    어떤 종류의 기화기나 스로틀 바디를 사용해도 무방하다.
    
  \item 스로틀 작동 방법
    \begin{enumerate}
      \item 스로틀 밸브는 케이블 등을 사용한 기계적인 방법 또는 전자식 스로틀 제어 등 전자식 방법을 허용한다.
      \item 전자식 스로틀 제어를 사용할 경우 \cref{section:전자식 스로틀 제어}를 만족하여야 한다.
    \end{enumerate}
    
  \item 과급기 사용 (Turbochargers \& Superchargers)\\
    터보차저와 수퍼차저 등의 과급기 사용이 가능하다.
\end{enumerate}

\section{전자식 스로틀 제어 - Electronic Throttle Control, ETC} \label{section:전자식 스로틀 제어}
\string[ C-Formula 해당 \string]

\begin{enumerate}
  \item 일반
    \begin{enumerate}
      \item 전원이 차단되면 전자 스로틀이 자동으로 닫혀야 한다 (공회전 상태로 복귀).
      
      \item 전자 스로틀은 스로틀을 공회전 위치로 되돌릴 수 있는 최소 두 개의 장치를 사용해야 한다.
        \begin{enumerate}
          \item 장치 중 하나는 스로틀을 작동하는 장치(액츄에이터)도 포함된다.
          \item 다른 장치는 액츄에이터 동력 손실이 발생할 경우 스로틀을 공회전 위치로 되돌릴 수 있는 스로틀 리턴 스프링이어야 한다.
          \item TPS의 스프링은 스로틀 리턴 스프링으로 허용되지 않는다.
        \end{enumerate}
        
      \item ETC 시스템은 의도하지 않은 가속을 방지할 수 있음이 입증되면 저단으로의 변속 중에 스로틀을 개방할 수 있다. (쉬프트 다운 시 스로틀 블리핑)
      \item ETC 시스템의 기능 등에 관한 것은 \cref{item:ETC 시스템 보고서}에 따라 작성 제출하여야 한다.
    \end{enumerate}
    
  \item 상용 ETC 시스템
    \begin{enumerate}
      \item 규정을 준수하지 않는 상용 ETC 시스템은 대회 전에 승인된 경우 사용할 수 있다.
      
      \item 승인을 받으려면 다음 내용을 포함하는 문서를 제출하여야 한다.
        \begin{enumerate}
          \item 팀이 사용 승인을 받고자 하는 ETC 시스템
          \item 상용 시스템이 준수하지 못하는 ETC 규정
          \item 사용하고자 하는 상용 시스템의 허용을 결정하기 위해 규정 미준수 부분에 대한 충분한 기술적 세부 정보
        \end{enumerate}
    \end{enumerate}
    
  \item ETC 시스템 보고서 (ETC Systems Form, ETCSF) \label{item:ETC 시스템 보고서}
    \begin{enumerate}
      \item ETC를 사용하는 팀은 ETC 시스템에 대한 보고서를 제출하여 사전 검토와 사용 승인을 받아야 한다.
      
      \item ETC 시스템 보고서에는 다음 내용을 포함한다.
        \begin{enumerate}
          \item ETC를 사용하려는 목적 또는 사용해야 되는 이유
          \item ETC 시스템의 구성, 작동방법
          \item 구성품의 잠재적 고장에 대한 대처방법
          \item 구성품 고장에 대한 탐지방법과 작동방법
        \end{enumerate}
        
      \item 별도로 공지된 보고서 제출 기한 내에 제출하지 않은 경우와 사용 목적, 사유가 타당하지 않을 경우 조직위원회에서 사용을 허가하지 않을 수 있다.
    \end{enumerate}
    
  \item 스로틀 위치 센서 (Throttle Position Sensor, TPS)
    \begin{enumerate}
      \item 스로틀 위치 센서(TPS)는 스로틀 또는 스로틀 액츄에이터의 위치를 측정해야 한다. 스로틀 위치는 완전히 닫힌 상태에서 완전히 열린 상태까지의 이동 비율로 정의되며, 0\%는 완전히 닫힌 상태, 100\%는 완전히 열린 상태이다.
      \item 두 개 이상의 개별 센서를 TPS로 사용해야 한다. 각 TPS는 공급 및 기준 전압의 차이를 감지할 수 있는 경우에만 동일한 공급 및 기준 전압을 공유하여 사용할 수 있다.
      \item 각 센서의 스로틀 위치가 10\% 이상 차이가 나거나 \cref{section:전자식 스로틀 제어}에 정의된 기타 고장이 발생한 경우 센서 고장으로 정의한다. 10\%보다 큰 차이 값의 사용은 가능하나 ETC 시스템 양식으로 정당성을 입증해야 한다.
      \item 두 TPS의 값 사이에 불안정성이 발생하고 100ms 이상 지속되면 전자 스로틀의 전원을 즉시 차단해야 한다.
      \item 3개의 센서를 사용할 때 TPS 고장이 발생하는 경우 10\% 이내로 스로틀 위치 값이 일치하는 두 개의 센서 값으로 스로틀 위치 목표 값을 정의할 수 있으며 3번째 TPS는 무시될 수 있다.
      
      \item 각 TPS는 기술 검사 중에 다음 중 하나를 갖추어 확인할 수 있어야 한다.
        \begin{enumerate}
          \item ECU와 다른 연결 사이에 영향을 주지 않고 TPS 신호를 별도의 분리형 커넥터로 연결
          \item 다른 연결에 영향을 주지 않고 스위치 전환으로 각 TPS 신호를 ECU와 차단할 수 있는 장치 또는 박스 사용
        \end{enumerate}
        
      \item TPS 신호는 아날로그 신호를 사용하거나 CAN 또는 FlexRay와 같은 디지털 데이터 전송 버스를 통해 스로틀 컨트롤러로 직접 전송해야 한다. TPS 또는 TPS 배선의 모든 고장은 컨트롤러에서 감지할 수 있어야 하며 차량 운행 불가능 상태로 판단한다.
      \item 아날로그 신호를 사용하는 경우 개방 회로 또는 단락 상태로 정상 작동 범위(예: $<$ 0.5V 또는 $>$ 4.5V)를 벗어나게 되면 고장이 발생한 것으로 간주한다. 센서를 평가하는 데 사용되는 회로는 풀다운 또는 풀업 저항을 사용하여 개방 회로 신호로 인한 오류가 감지되도록 해야 한다.
      
      \item 모든 종류의 디지털 데이터 전송을 사용하여 TPS 신호를 전송하는 경우,
        \begin{enumerate}
          \item ETC 시스템 보고서에는 발생할 수 있는 모든 잠재적 고장 모드, 이러한 고장을 탐지하는데 사용하는 방법 및 탐지 방법이 정상적으로 작동하는지 입증하기 위해 수행된 테스트에 대한 자세한 설명이 포함되어야 한다.
          \item 고려해야 할 고장 사례는 다수 존재하나 TPS 고장, TPS 신호 범위 초과, 메시지 손상 및 메시지 손실 및 관련 시간 초과가 반드시 포함되어야 한다.
        \end{enumerate}
    \end{enumerate}
    
  \item 가속 페달 (Accelerator Pedal)\\
    가속 페달은 \cref{item:가속 페달}을 만족해야 한다.
    
  \item 가속 페달 위치 센서 (Accelerator Pedal Position Sensor, APPS)
    \begin{enumerate}
      \item 가속 페달은 APPS를 작동시켜야 한다.
        \begin{enumerate}
          \item 가속 페달을 0\% 위치(아이들 상태)로 되돌리기 위해 반드시 두 개의 스프링을 사용해야 한다.
          \item 각 스프링은 다른 하나의 스프링이 파손되었을 때 가속 페달을 0\% 위치(아이들 상태)로 되돌릴 수 있어야 한다. APPS의 스프링을 페달 리턴 스프링으로 사용할 수 없다.
        \end{enumerate}
        
      \item 두 개 이상의 전기적으로 분리된 센서를 APPS로 사용해야 한다. 단일 하우징에 두 개의 분리된 센서가 있는 OEM 제품의 APPS도 사용 가능하다.
    \end{enumerate}
    
  \item 브레이크 시스템 인코더 (Brake System Encoder, BSE)
    \begin{enumerate}
      \item 차량에는 브레이크 페달의 위치나 브레이크 압력을 측정할 수 있는 센서 또는 스위치가 있어야 한다.
      
      \item 차량검사가 진행되는 동안 BSE는 다음 사항 중 하나를 가지고 있음을 확인할 수 있어야 한다.
        \begin{enumerate}
          \item ECU와 다른 연결 사이에 영향을 주지 않고 BSE 신호를 별도의 분리형 커넥터로 연결
          \item 다른 연결에 영향을 주지 않고 스위치 전환으로 각 BSE 신호를 ECU와 차단할 수 있는 장치 또는 박스 사용
        \end{enumerate}
        
      \item BSE 또는 스위치 신호는 반드시 아날로그 신호나 CAN 또는 FlexRay와 같은 디지털 데이터 전환 버스를 사용하여 제어기에 직접 신호를 전송하여야 한다.
      \item BSE 또는 BSE 배선의 문제가 100ms 이상 지속된 경우를 제어기에서 감지할 경우 전자식 스로틀(ETS)의 전원이 즉시 차단되어 작동을 멈추어야 한다.
    \end{enumerate}
    
  \item 스로틀 타당성 검사 (Throttle Plausibility Checks) \label{item:스로틀 타당성 검사}
    \begin{enumerate}
      \item 브레이크 및 스로틀 위치
        \begin{enumerate}
          \item 기계식 브레이크가 작동되고 TPS가 스로틀이 1초 이상 허용된 양 이상 열려 있다는 신호를 보내는 경우 전자 스로틀의 전원을 차단해야 한다.
          \item 브레이크 작동 후 스로틀은 1초 이내에 닫혀야(아이들 상태로 복귀) 하며, 시간 내에 스로틀이 닫히지 않으면 연료 및 점화 시스템이 즉시 차단되어야 한다.
          \item BSE와 TPS 간의 허용 관계는 팀이 테이블을 사용하여 정의할 수 있다. 이 기능은 기술 검사에서 입증되어야 한다.
        \end{enumerate}
        
      \item 스로틀 위치값 vs 목표값
        \begin{enumerate}
          \item 스로틀 위치가 예상 목표 TPS 위치에서 1초 이상 차이가 나는 경우 전자 스로틀의 전원을 즉시 차단해야 한다.
          \item 차이가 10\% 미만으로 줄어들기 위해 1초의 시간이 허용되며, 허용된 시간 초과 시 연료 및 점화 시스템이 즉시 중단되어야 한다. \label{item:TPS limit}
          \item TPS 위치의 오류와 그에 따른 시스템 차단은 기술 검사에서 입증되어야 한다.\\
            팀은 위의 \cref{item:TPS limit} 에서 요구된 작업이 정상적으로 작동됨을 차량 검사 시 테스트 방법을 통해 입증할 수 있어야 한다. 캘리브레이션 소프트웨어를 사용하여 표시되는 시스템 상태에는 제어 시스템에 관하여 자세한 설명이 수반되어야 한다.
        \end{enumerate}

      \item TPS 신호가 스로틀에 전원이 공급되지 않는 기본 위치 또는 그 이하에서 1초 이상 위치하면 전자식 스로틀 및 연료 분사 \& 점화 시스템은 즉각 차단되어야 한다.
    \end{enumerate}
    
  \item 제동시스템 타당성 장치 (Brake System Plausibility Device, BSPD)
    \begin{enumerate}
      \item 전자 스로틀 제어를 모니터링하기 위해서는 반드시 독립형의 프로그래밍이 불가능한 회로를 사용해야 한다. BSPD는 반드시 \cref{item:스로틀 타당성 검사}의 검사에 포함되어야 한다.
      \item 모든 센서의 신호는 반드시 BSPD로 직접 전송되어야 한다. 다른 모듈의 출력은 센서 신호 대신 사용할 수 없다.
      
      \item BSPD는 다음 조건을 모니터링해야 한다.
        \begin{enumerate}
          \item 아래 두 가지 상태가 1초 이상 지속될 때:
            \begin{itemize}
              \item 급제동 상태 (BSE)
              \item 스로틀 10\% 이상 개방 (TPS)
            \end{itemize}
          \item 제동 센서에서의 신호 손실이 100ms 이상인 경우
          \item 스로틀 센서의 신호 손실이 100ms 이상인 경우
          \item BSPD 회로에 전원이 차단된 경우
        \end{enumerate}
        
      \item 위 조건 중 하나라도 발생하면 BSPD는 주 비상 정지 스위치(\cref{item:주 비상 정지 스위치})를 개방한 것과 동일하게 작동하여 다음 사항을 만족해야 한다.
        \begin{enumerate}
          \item 엔진 정지
          \item 연료펌프, 점화장치, 전자제어 스로틀(ETC) 전원 차단
        \end{enumerate}
        
      \item BSPD는 주 비상 정지 스위치에 의해서만 리셋 가능하다. (\cref{item:주 비상 정지 스위치})
      \item 운전석 스위치(보조 비상 정지 스위치)가 꺼져 있을 때 BSPD가 리셋되어서는 안 된다. (\cref{item:보조 비상 정지 스위치})
      
      \item BSPD 신호 및 기능은 다음 중 하나를 사용하여 기술 검사 중에 확인할 수 있어야 한다.
        \begin{enumerate}
          \item ECU와 다른 연결 사이에 영향을 주지 않고 브레이크 센서, 스로틀 센서 신호를 차단할 수 있는 커넥터 및 BSPD 전원만을 차단할 수 있는 분리 가능한 커넥터
          \item ECU와 다른 연결 사이에 영향을 주지 않고 브레이크 센서와 스로틀 센서의 신호를 개별적으로 차단하고 BSPD 장치에만 전원을 공급할 장치 또는 박스 사용
        \end{enumerate}
    \end{enumerate}
\end{enumerate}

\chapter{전기 시스템 문서 및 전기 기술 검사 (Documentation, Electrical Tech Inspection):\ \ E-Formula 해당}

\section{전기시스템 관리자 - Electrical System Officer, ESO}
\begin{enumerate}
  \item 전기시스템 관리자(ESO)는 대회 중 차량의 모든 전기 작업을 관리한다.
  \item 모든 참가팀은 팀원 중 한 명 이상의 전기시스템 관리자를 지정해야 한다. 전기시스템 관리자 중 적어도 한 명은 드라이버가 아니어야 한다.
  \item 전기시스템 관리자는 자동차의 고전압 시스템에 대해 적절한 교육을 받거나 인증을 받아야 한다. 전기시스템 관리자(ESO) 양식에 교육에 대한 세부 정보를 작성하여 제출해야 한다.
  
  \item 전기시스템 관리자는 아래 사항과 같은 의무를 수행한다.
    \begin{enumerate}
      \item 차량의 어떤 시스템에서 작업을 하더라도 그 이전에 차량의 전기적인 안전 상태를 진단하여 작업을 허가할 수 있는 유일한 인원이다.
      \item 대회장에서 차량이 활성화되거나 이동할 때 반드시 동행해야 한다.
      \item 대회 기간 동안 전기 기술 검사 및 축전지 충전 시에 반드시 동행해야 한다.
      \item 안전분석보고서(FMEA)를 숙지하여 오류 상황에 대처할 수 있어야 한다.
      \item 대회 기간 동안 항상 전화로 연락할 수 있어야 한다.
      \item 전기시스템 관리자(ESO)는 대회장 내에서 차량 정비 또는 이동 시 항상 HV 비상 정지 스위치 off 및 HVD 분리 상태와 축전지 전압 표시장치를 확인해야 한다.
      \item HV 비상 정지 스위치의 키는 차량에서 완전히 분리 가능해야 하며 대회 기간 동안 차량 정비 또는 대회장 내 이동 시 전기 시스템 관리자가 키를 분리하여 관리하여야 한다.
    \end{enumerate}
    
  \item 전기시스템 관리자(ESO) 양식은 대회 홈페이지에서 별도로 공지한다.
  \item 제출 기한은 참가팀 등록 마감일을 기준으로 한다.
\end{enumerate}

\section{안전분석보고서 - Failure Mode and Effect Analysis, FMEA}
\begin{enumerate}
  \item 모든 팀은 차량의 전기 시스템과 관련된 안전 분석 보고서를 주어진 양식에 맞추어 작성하여 제출해야 한다.
  \item 안전분석보고서(FMEA) 양식 및 제출 마감 일자는 별도로 공지한다.
\end{enumerate}

\section{전기시스템 보고서 - Electrical System Form, ESF}
\begin{enumerate}
  \item 모든 팀은 안전검사에 앞서 제어 및 구동시스템을 포함한 모든 전기시스템에 대해 명확하게 구성된 전기시스템 보고서(ESF)를 제출하여야 한다.
  
  \item 전기시스템 보고서에는 다음 사항 등이 포함되어야 한다.
    \begin{enumerate}
      \item HV(High Voltage), LV(Low Voltage) 전압
      \item topology (구동방식, 모터 배치 등)
      \item 전체 배선도(회로도)
      \item 축전지 구성
    \end{enumerate}
    
  \item 전기 시스템 보고서에는 축전지, 모터, 모터 제어기, 전선, 보호관, 과전류 보호 장치, 릴레이, 커넥터를 포함하는 구동시스템에 사용된 모든 부품에 대한 정격 사양이 표시되어야 하며 이 부품에 대한 데이터시트가 포함되어야 한다. 데이터시트 및 참고자료는 별도의 파일로 제출하지 않고 전기 시스템 보고서 내에 첨부하거나 링크로 표시한다.
  \item 전기시스템 보고서의 양식은 대회 홈페이지에서 별도로 공지한다.
  \item 전기시스템 보고서 제출 기한은 별도로 공지한다.
  \item 전기시스템 보고서를 기한 내에 제출하지 않으면 벌점이 부여되고 정해진 기한 내에 검토자로부터 최종 승인을 받지 못한 팀은 대회에 참가할 수 없다.
\end{enumerate}

\section{전기 기술 검사 - Electrical Tech Inspection}
\begin{enumerate}
  \item 전기 기술 검사는 차량검사 중 실시하며 전기 기술검사를 통과하지 못한 차량은 제동 검사 및 동적 경기에 참가할 수 없다.
  \item 전기 기술 검사를 통과하기 전까지 HV 비상 정지 스위치와 고전압 분리기(HVD)를 분리한 상태에서만 대회장에서 이동하거나 정적 이벤트에 참가할 수 있다.
  \item 전기 기술 검사는 전기시스템 검사, 절연 저항 측정 검사(IMT), 절연 감시장치 검사(IMDT), 우천 검사(Rain Test)를 포함한다.
  \item 틸팅검사를 먼저 진행한 후 우천 검사를 진행한다.
  \item 전기 기술 검사가 완료되고 구동시스템 및 기타 구성 요소의 전체 또는 일부가 밀봉될 수 있으며 대회 운영위원회의 허가 없이 밀봉이 제거된 경우 대회 참가 자격이 박탈된다.
\end{enumerate}

\section{전기시스템 검사 - Electrical System Inspection}
\begin{enumerate}
  \item 전기시스템 검사는 아래 사항을 포함한다.
    \begin{enumerate}
      \item 전기시스템 보고서(ESF) 제출
      \item HV, LV 측정
      \item 충전기 및 축전지박스 핸드카트 검사
      \item 축전지 검사
      \item 전원 차단 회로 검사
      \item 고전압 표시등(TSAL) 검사
      \item 그 외 EV 규정 확인
    \end{enumerate}
  \item 전기시스템 검사에 아래 품목을 지참하여야 한다.
    \begin{enumerate}
      \item 축전지 충전기
      \item 축전지박스 핸드카트
      \item 여분의 축전지박스 (해당될 경우만)
      \item ESF, FMEA 및 부품 데이터시트
      \item 전기시스템 사용에 적합한 절연 공구
      \item 멀티미터
      \item HV 절연장갑 2켤레
      \item 측면 보호물이 있는 보안경
    \end{enumerate}
\end{enumerate}

\section{절연 저항 측정 검사 - Insulation Resistance Measurement Test, IMT} \label{section:절연 저항 측정 검사}
\begin{enumerate}
  \item 전압 측정 포인트의 HV +/-와 GLVS Ground 사이의 절연 저항은 전기 기술 검사(Electrical Tech Inspection)에서 측정한다.
  \item 절연 저항 측정 검사(IMT)를 통과하려면 측정된 절연 저항이 최대 구동시스템 작동 전압에 관하여 500 Ω/V 이상이어야 한다.
  \item 절연 저항 측정을 위한 테스트 전압은 구동 시스템 최대전압이 250V 이하이면 250V를 인가하고, 구동 시스템 최대전압이 250V를 초과하면 500V를 인가한다.
\end{enumerate}

\section{절연 감시장치 검사 - Insulation Monitoring Device Test, IMDT}
\begin{enumerate}
  \item 절연 감시장치 검사(IMDT)는 전기 기술 검사(Electrical Tech Inspection)에서 진행된다.
  \item 구동시스템이 동작하는 동안 전압 측정 포인트에 250 Ω/V의 저항을 연결하여 검사한다.
  \item 전압 측정 포인트의 HV +/-와 GLVS Ground에 검사 저항을 연결했을 때 IMD가 구동시스템을 30초 내에 차단시키면 절연 감시장치 검사를 통과한다.
  \item 절연 감시장치 검사(IMDT)는 경기 진행 중 수시로 진행할 수 있다.
  \item 차량이 처음 절연 감시장치 검사를 통과하면 구동시스템의 중요한 부분은 봉인된다. 만약 어떤 봉인이라도 파손되면 다시 절연 감시장치 검사를 통과하기 전까지는 그 어떤 경기도 참가할 수 없다.
\end{enumerate}

\section{우천 검사 - Rain Test}
\begin{enumerate}
  \item 우천 검사(Rain Test)는 전기 기술 검사(Electrical Tech Inspection)에서 진행된다.
  \item 절연 감시장치 검사(IMDT)를 통과한 차량만 우천 검사를 진행할 수 있으며 우천 검사 직전에 IMDT를 진행 후 우천 검사를 진행한다.
  \item 우천 검사를 진행하는 동안 구동시스템에 전원이 공급되어야 한다. 모든 구동 바퀴는 공중에 떠있는 상태이어야 하며 차량에는 아무도 탑승할 수 없다.
  \item 방향에 상관없이 2분(120초) 동안 비가 오는 상황과 유사하게 차량에 물을 분사한다. 물을 차량에 고압으로 분사하지는 않는다.
  \item 물 분사 시작 후 1분 지난 시점에 차량의 전원을 모두 껐다가 다시 켜는 동작을 1회 수행한다.
  \item 우천 테스트를 통과하기 위해서는 물을 뿌리는 동안 IMD가 반응하지 않아야 하고 물 분사가 멈춘 후 120초 이내에도 IMD는 동작하지 않아야 한다. 총 검사 시간은 4분(240초)이다.
  \item 우천 검사 실패 후 덮개 등 물줄기를 방호하는 방식으로 조치하는 것은 허용하지 않으며 경기 참가 시 우천 검사 통과와 동일한 상태로 참가하여야 한다.
  \item 인휠 모터 차량의 경우 휠을 탈거한 상태에서 우천 검사를 실시한다.
\end{enumerate}

\chapter{전기시스템 (Electric System): E-Formula 해당}

\section{정의 - Definitions}
\begin{enumerate}
  \item 저전압 시스템(Low Voltage System): 동작 전압이 60V DC 이하 또는 50V AC RMS 이하의 회로나 부품을 저전압 시스템(LV System)으로 정의한다. \label{item:저전압 시스템}
  \item 고전압 시스템(High Voltage System): 동작 전압이 60V DC 초과 또는 50V AC RMS를 초과하는 회로나 부품을 고전압 시스템(HV System)이라고 정의한다.
  \item 구동시스템(Tractive System, TS): 모터와 축전지에 전기적으로 연결된 모든 회로 및 부품으로 정의하고 다음과 같은 조건을 만족해야 한다. \label{item:구동시스템}
    \begin{enumerate}
      \item 차체(Chassis)를 포함한 차량의 모든 전도체로부터 전기적으로 완전히 격리되어야 한다.
      \item 구동시스템의 모든 부품의 정격전압은 구동시스템(축전지) 최대전압과 같거나 높아야 한다.
      \item 60V 이하의 전압을 가지더라도 구동시스템과 전기적으로 연결되어 있으면 구동시스템에 포함된다.
    \end{enumerate}
    
  \item GLVS(Grounded Low Voltage System): 구동시스템이 아닌 모든 전기 회로 및 부품으로 정의하고 다음과 같은 조건을 만족해야 한다.
    \begin{enumerate}
      \item \cref{item:저전압 시스템}에서 정의한 저전압 시스템(LV System)이다.
      \item 차체(Chassis)에 접지되어야 한다.
      \item 구동시스템이 활성화되기 전에 특정 절차를 통해 반드시 전원이 먼저 인가되어야 한다.
    \end{enumerate}
    
  \item 출력에 따른 사용 가능 최대 전압\\ \label{item:출력에 따른 사용 가능 최대 전압}
    구동시스템의 최대전압은 600V DC 이하이다.
    
  \item 전체 구동시스템(Tractive System)과 접지된 저전압 시스템(GLVS)은 전기적으로 완전히 분리되어야 한다. 구동시스템과 접지된 저전압 시스템의 경계는 전기적 절연(galvanic isolation)이다. 따라서 모터 제어기와 같은 몇몇 부품들은 두 시스템이 함께 존재할 수 있다.
  
  \item 전기적 절연(galvanic isolation)\\
    전기적 절연인 전기 회로들은 다음을 모두 만족해야 한다.
    \begin{enumerate}
      \item 두 회로의 저항은 구동시스템 최대 전압을 기준으로 500Ω/V 이상이어야 한다. 이는 \cref{section:절연 저항 측정 검사}를 바탕으로 검사한다.
      \item 절연판(isolation barrier)의 동작 전압(working voltage)이 데이터시트에 명시된 경우, 구동시스템의 최대 전압보다 높아야 한다. 또한 전기적 절연된 회로 사이를 연결하는 커패시터는 class-Y 만을 허용한다.
    \end{enumerate}
    
  \item 고전류 경로(High current path)\\ \label{item:고전류 경로}
    정격으로 1A 이상의 전류가 흐르는 모든 전기적 경로는 고전류 경로이다.
\end{enumerate}

\section{구성요소 - Component}
\begin{enumerate}
  \item 구동시스템 모터는 반드시 모터 제어기를 통해서 축전지에 연결되어야 한다. 모터를 축전지에 바로 연결하는 것은 금지한다.
  \item 차량이 후진하도록 모터에 전원을 공급하는 것은 금지한다.
  \item 최대 출력과 최대 전압은 \cref{item:출력에 따른 사용 가능 최대 전압}에 명시된 기준을 따른다. 출력과 전압은 에너지미터에 의해 확인되며 위반 사항이 발견될 경우 해당 경기의 차량은 DNF 처리 또는 페널티가 부과된다.
  
  \item 모터(Motors)
    \begin{enumerate}
      \item 모든 형태의 전기 모터를 사용할 수 있고 모터의 수는 제한이 없다.
      \item 모터의 회전자 부분은 모터 케이스 내에 포함되어야 한다.
      \item 모터 케이스의 최소 두께는 알루미늄 3mm, 철판 2mm이다. 회전자가 케이스 안쪽에 있는 상용품의 경우 별도로 케이스를 제작할 필요는 없다.
      \item 회전자가 모터 바깥쪽인 외륜형 모터의 경우 최소 1mm 두께의 알루미늄 또는 철판으로 만든 차폐물을 모터 주변에 설치해야 한다.
    \end{enumerate}
    
  \item 가속 페달 (Accelerator Pedal) \label{item:가속 페달}
    \begin{enumerate}
      \item 가속 페달은 파워트레인 출력을 제어해야 한다.
      \item 페달의 움직이는 거리는 초기 위치(밟지 않은 상태, 아이들 상태)에서 완전히 밟은(풀 스로틀) 위치까지의 백분율이다. 0\%는 초기 위치, 100\%는 완전히 밟은 상태이다.
      
      \item 가속 페달은 다음을 만족해야 한다.
        \begin{enumerate}
          \item 밟지 않았을 때 페달 위치를 0\%로 복귀
          \item 케이블, 액츄에이팅 시스템 및 센서 등이 손상되거나 과도한 응력(Stress)을 받지 않도록 스토퍼 적용
        \end{enumerate}
    \end{enumerate}
    
  \item 가속 페달 위치 센서 (Accelerator Pedal Position Sensor - APPS)
    \begin{enumerate}
      \item 가속 페달 위치 센서는 반드시 가속 페달에 의해 작동되어야 한다.
      \item 가속 페달 위치 센서는 가속 페달 작동에 의해 손상되지 않도록 설계되어야 한다.
      \item 가속 페달이 움직이는 부분에 배선 또는 다른 부품과의 간섭이 없어야 한다.
      \item 가속 페달은 작동하지 않을 때 반드시 원래의 위치까지 되돌아가야 한다. 반드시 2개의 리턴 스프링을 사용해야 하며 1개의 리턴 스프링으로도 원래 위치(0\%, 아이들 상태)까지 되돌아가야 한다. 센서 내부에 포함된 스프링은 리턴 스프링에 포함되지 않는다.
      \item 가속 페달 위치 센서 신호는 아날로그 또는 디지털 데이터 통신을 통해 모터 제어기로 직접 전송되어야 한다.
      \item 아날로그 신호를 사용할 경우 개방(Open) 또는 단락(Short)에 대비하여 동작 범위를 설정해야 한다. (예를 들면, 5V 전압 사용 시 동작 범위를 0.5V \string~ 4.5V로 설정하여 출력이 0.5V보다 작거나 4.5V보다 클 경우 동작 신호 미출력(Failure)으로 인식하도록 설정해야 한다.)
      \item 가속 페달 위치 센서 신호를 조작할 수 있는 알고리즘이나 전자 제어 장치는 (예를 들면, 트랙션 제어와 같은 차량 동적 제어 기능) 가속 페달 위치 센서로부터 입력된 토크보다 낮을 수 있지만 이를 절대로 증가시켜서는 안 된다.
    \end{enumerate}
  \item 저전압 축전지(LV Batteries)의 특정 요구사항은 \cref{section:저전압 축전지}에 명시된 기준을 따른다.
\end{enumerate}

\section{구동시스템과 접지된 저전압 시스템의 분리 - Separation of TS and GLVS}
\begin{enumerate}
  \item 차량의 프레임 또는 다른 전도성 표면과 구동시스템 회로의 어느 부분 사이에도 전기적 연결이 없어야 한다.
  \item 인터락(interlock) 회로 연결을 제외하고는 구동시스템(Tractive System)과 접지된 저전압 시스템(GLVS)은 동일한 도관 또는 커넥터를 통과하지 않도록 물리적으로 반드시 분리되어야 한다.
  \item 구동시스템과 접지된 저전압 시스템이 동일한 인클로저 내에 있을 때, 두 시스템의 부품과 배선은 내습성, 150℃ 이상의 내열성 절연 재료(전선피복 제외)에 의해 분리되거나 (예를 들면, Nomex 기반의 전기 절연)에 의해 분리되거나 전압 차이에 따라 다음과 같은 최소 절연 거리를 유지해야 한다. 이 절연 거리가 항상 유지되도록 모든 부품과 배선은 확실하게 고정되어 있어야 한다.\\
  ※ 전압 차이 U는 구동시스템에서의 최대 전압을, 저전압 시스템에서의 최소 전압을 기준으로 한다.
    \begin{enumerate}
      \item U < 100V DC : 10mm
      \item 100V DC < U < 200V DC : 20mm
      \item U > 200V DC : 30mm
    \end{enumerate}
    
  \item 구동시스템(TS)과 접지된 저전압 시스템(GLVS)이 동일한 회로 보드(PCB)에 있을 때, 두 시스템은 요구되는 필수 간격에 의해 분리되어야 하고 각 영역이 보드에 명확하게 표기되어야 한다. 또한, 동일한 회로 보드(PCB)에서 구동시스템(TS)들 사이도 필수 간격에 의해 분리되어야 한다.\\
    트레이스 및 보드에서 요구되는 필수 간격은 \cref{fig:절연}와 같은 조건표를 만족해야 하며 이때 전압은 두 도체 사이의 전위차이다. 공간거리와 연면거리(혹은 절연코팅시 연면거리)를 모두 만족해야 한다.\\
    구동시스템의 최대 전압에 정격이지만 필수 간격을 만족하지 못하는 photo-coupler와 같은 집적회로는 사용될 수 있으며 이때 필수 간격은 적용되지 않는다.
    
  \begin{table}[H]
    \centering
    \begin{tblr}{
      width = 0.8\linewidth,
      colsep = 6pt,
      colspec = {|X[c,m]|X[c,m]|X[c,m]|X[c,m]|},
      rows = {ht=1.0\baselineskip},
      row{1,2} = {bg=gray!30, font=\bfseries},
      hlines,
      vlines
    }
      \SetCell[r=2]{c} 전압
        & \SetCell[r=2]{c} 공간거리
        & \SetCell[c=2]{c} 연면거리
        & \\
        & 
        & 일반
        & 절연코팅 시 \\
      0--50V DC    & 1.0mm & 4mm  & 1mm  \\
      50--150V DC  & 1.0mm & 5mm  & 1mm  \\
      150--300V DC & 1.5mm & 10mm & 2mm  \\
      300--600V DC & 3.0mm & 20mm & 4mm  \\
    \end{tblr}
  \end{table}
  
  \fig{절연}{formula}{0.8}
  
  \item 연면거리를 만족하기 위한 구조물 혹은 공극은 1.5mm 이상의 폭을 가져야 한다. 또한, 위 표의 절연코팅(Conformal coating)은 PCB 기판 위의 절연 코팅을 뜻하며 솔더 레지스트(Solder resist)는 절연코팅으로 인정하지 않는다.
  \item 팀은 장비의 간격을 증명할 수 있어야 한다. 이에 대한 정보는 전기시스템 보고서(ESF)에 포함되어야 한다. 회로에 쉽게 접근할 수 없는 경우, 검사를 위해 여분의 보드 또는 적절한 사진을 준비해야 한다.
\end{enumerate}

\section{접지 - Grounding} \label{section:접지}
\begin{enumerate}
  \item 차량의 전기 전도성 부품은 (스틸, (아노다이징된) 알루미늄, 기타 금속 부품 등으로 시트 고정 마운트, 안전 벨트 마운트, 방화벽, 방화벽 마운트 지점 등을 말한다.) GLVS의 Ground에 대하여 100mΩ (1A 전류로 측정) 미만의 저항을 가져야 한다. \label{item:접지}
  \item 구동시스템의 부품으로부터 100mm 이내에 전기 전도성 부품(예를 들어, 완전히 코팅된 금속 부품 또는 탄소 섬유 부품 등)이 될 수 있는 차량 부품은 GLVS Ground와 100Ω 미만의 저항을 가져야 한다.
  \item 전기 검사에서 접지 저항 측정을 위해 금속 부품의 코팅을 일부 제거할 수 있다.
\end{enumerate}

\section{구동시스템 부품의 배치 - Positioning of Tractive System Parts}
\begin{enumerate}
  \item 전선과 배선을 포함한 모든 구동시스템의 부품은 충돌 또는 전복 시 손상되지 않도록 차량의 롤오버 보호 영역 (\cref{fig:롤오버 보호 영역} 참조) 내부에 위치해야 한다. \label{item:롤오버}
  \item 정면 또는 측면에서 보았을 때, 전선을 포함한 구동시스템의 어떤 부분도 프레임의 하부 표면 또는 모노코크 아래로 돌출되어서는 안 된다.
  
  \fig{롤오버 보호 영역}{formula}{0.8}
  
  \item 구동시스템의 부품이 지면에서 350mm 미만의 위치에 장착된다면 측면 충돌, 후면 충돌, 물체의 침입으로부터 보호되어야 한다.
  \item 휠 모터는 (모터, 배선이 \cref{item:롤오버}을 만족하지 않는 경우) 다음의 조건을 만족하는 경우에만 적용 가능하다.
    \begin{enumerate}
      \item 모터 배선이 손상되었을 경우 즉시 차단 회로를 개방하여 구동 시스템 릴레이를(AIR) 개방하는 인터락(interlock)이 추가되어야 한다.
      \item 모터 배선은 휠의 움직임에 의해 영향을 받거나 손상되지 않도록 휠의 움직임을 고려하여 고정해야 하며 배선 길이는 최소한으로 적용해야 한다.
    \end{enumerate}
\end{enumerate}

\section{구동시스템 절연 배선 및 연결 - Tractive System Insulation Wiring and Conduit}
\begin{enumerate}
  \item 구동시스템의 모든 부품, 특히 활선(live wire), 접점 등은 반드시 비전도성 절연 재료 또는 촉으로부터 보호할 수 있는 덮개로 격리되어야 한다. 배선은 120℃ 이상의 내열성 소재의 콜게이트 튜브 또는 유사한 방식으로 보호해야 하며 전기 접속부 혹은 축전지박스 인입부는 케이블 그랜드, 방수 커넥터 사용을 권장한다.
  \item 구동시스템 부품 및 축전지박스는 습기로부터 보호되어야 한다. 우천 검사(Rain Test)를 위해서 적어도 IP67 이상의 보호 등급을 권장한다.
  \item 절연재료는 예상되는 온도 변화에 적합한 내열성을 지녀야 하며, 최소 허용 온도 등급은 90℃이다. 절연테이프만을 이용한 절연은 금지된다.
  \item 구동시스템에 사용되는 모든 전선, 단자 및 기타 도체는 구동시스템의 연속 전류에 정격이어야 한다.
  \item 전선에는 직경, 온도 등급, 절연 전압 등급이 표시되어야 한다. 그렇지 않을 경우, 전선의 특성을 명확하게 알 수 있는 일련번호가 전선에 인쇄되어 있어야 한다. 구동시스템 전선의 최소 허용 온도 등급은 90℃이다. 사용된 모든 전선의 정보는 전기시스템 보고서(ESF)에 포함되어야 한다.
  \item 인클로저 외부에 존재하는 구동시스템 전선은 반드시 주황색이어야 한다. 그렇지 않을 경우, 주황색의 비전도성 전선 덮개(cover)로 감싸야 한다. 전선 덮개는 움직이지 않도록 잘 고정되어야 한다. 전선을 도색하는 행위는 금지한다.
  \item 인클로저 외부에 존재하는 구동시스템 커넥터는 차단회로를 개방하는 인터락(interlock) 기능이 포함되어야 한다. 커넥터만을 위한 인클로저는 허용하지 않는다.
  \item 구동시스템의 배선은 풀리거나 기구적 스트레스를 받지 않도록 단단히 고정되어야 한다. 구동시스템 배선은 회전 또는 움직이는 부품에 의해 손상되지 않도록 100mm 이상 이격하거나 금속 재질의 프로텍터로 보호되어야 한다. 차량의 가장 낮은 차체보다 낮은 위치에 배선이 위치하는 것은 금지한다.
  \item 구동시스템이 아닌 부분의 배선에 주황색 전선이나 주황색 전선 덮개 사용을 금지한다.
  \item 모터 케이스를 제외한 구동시스템의 일부를 포함하는 모든 하우징 또는 인클로저에는 ISO 7010-W012 (노란색 바탕에 검은색 번개가 있는 삼각형) 심볼 스티커를 부착해야 하고 스티커에는 "High Voltage" 또는 "고전압"의 글자가 포함되어야 한다.
  
  \item 모든 구동시스템의 전기적 연결부는 아래 사항을 만족하여야 한다.
    \begin{enumerate}
      \item 전류 경로는 구리 또는 알루미늄과 같은 도체를 통과하도록 설계되어야 한다.
      \item 강철 볼트는 도체로 인정하지 않는다.
      \item 구동시스템의 전기적 연결부에는 플라스틱과 같은 비금속 재료의 사용을 금지한다. (진동 등에 의한 단자 간 접촉 불량 방지)
    \end{enumerate}
    
  \item 외부 비절연 히트 싱크를 사용하는 경우 GLVS에 접지해야 하며, \cref{section:접지}를 만족해야 한다.
  \item \cref{item:고전류 경로}에 해당하는 모든 고전류 경로에 있는 볼트, 너트 및 기타 고정 장치를 포함한 모든 전기적 연결부는 고온에 적합한 풀림 방지 체결방식을 사용해야 한다. 풀림 방지 체결방식은 \cref{section:체결장치}를 따른다.
  
  \item 구동 시스템의 고전류 경로에 납땜을 이용한 연결은 다음 3가지의 항목을 모두 만족하는 경우 이외에는 허용하지 않는다.
    \begin{enumerate}
      \item PCB 상에서 연결될 경우
      \item 셀이나 와이어가 아닌 경우
      \item 풀림방지를 위한 추가적인 기계적 고정 장치가 있는 경우
    \end{enumerate}
\end{enumerate}

\section{고전압 분리기 - High Voltage Disconnect, HVD}
\begin{enumerate}
  \item 차량은 구동 시스템 축전지의 하나 이상의 극을 분리할 수 있는 고전압 분리기를(HVD) 가지고 있어야 한다. 고전압 분리기는 방해하는 요소 없이 직접 접근할 수 있어야 하고 다음과 같은 조건을 만족해야 한다.
    \begin{enumerate}
      \item 고전압 분리기는 차량 바디를 탈거하지 않고 분리할 수 있어야 한다.
      \item 고전압 분리기는 드라이버가 탑승한 상태에서 지상에서 350mm 이상이어야 하며 차량 뒤에서 쉽게 인식할 수 있어야 한다.
      \item 긴 손잡이, 로프 또는 와이어를 통한 고전압 분리기의 원격 작동은 허용되지 않는다. 고전압 분리기의 분리를 위한 도구는 허용되지 않는다.
    \end{enumerate}
    
  \item 누구나 경기 준비 상태에서 10초 이내에 고전압 분리기를 제거할 수 있어야 한다.
  \item 고전압 분리기 분리 후 시스템의 절연을 위해 더미 커넥터 혹은 이와 유사한 장치가 있어야 한다.
  \item 고전압 분리기에 "HVD"라고 명확하게 표시되어야 한다.
  \item 고전압 분리기가 탈거될 때, 고전압 분리기 인터락은 (interlock) 차단 회로를 작동시켜야 한다.\\
    ※ 인터락: 커넥터의 탈거 여부를 판단할 수 있는 장치 (\cref{fig:인터락} 참고)
\end{enumerate}

\fig{인터락}{formula}{0.8}

\section{초기충전회로와 방전회로 - Pre-Charge Circuit and Discharge Circuit}
\begin{enumerate}
  \item 두 개 이상의 HV 릴레이(AIR) 중 마지막 HV 릴레이(AIR)가 닫히기 전에 구동시스템 전압이 축전지 전압의 90\%까지 확보되도록 초기충전 회로를(Pre-charge Circuit) 구현해야 한다. (구동시스템에 축전지 전압이 그대로 인가될 경우 돌입 전류에 의해 인버터가 손상될 수 있음)
  \item 초기충전회로는 HV 릴레이가 닫힌 후에는 작동을 멈춰야 하며, 차단회로가 개방되었을 경우 초기충전은 불가능해야 한다.
  \item 초기충전회로의 전원은 HV 비상 정지 스위치에서 직접 공급되어야 한다.
  \item 두 개 이상의 HV 릴레이(AIR) 중 마지막 HV 릴레이가 닫히기 전에, 1초 이상의 시간 동안 구동시스템을 초기충전해야 한다. 구동시스템의 순간 전압을 측정하는 피드백은 요구되지 않는다.
  \item 방전회로는 차단 회로가 개방되거나 차량에서 LV 전원이 차단되었을 때 항상 활성화되도록 설계되어야 한다. 또한, 고전압 분리기(HVD)가 분리되었을 때에도 방전될 수 있도록 설계되어야 한다.
  \item \cref{item:구동시스템 비활성화}을 준수하기 위한 방전회로는 최소 15초 동안 최대 방전 전류를 처리하도록 설계해야 한다.
  \item 초기충전회로 및 방전회로는 퓨징을 금지한다.
\end{enumerate}

\section{구동시스템 전압 측정 포인트 - Tractive System Measuring Points, TSMP} \label{section:TSMP}
\begin{enumerate}
  \item 구동시스템 전압 측정 포인트(TSMP)는 전기 기술 검사 동안 구동시스템이 차단되는지 확인하기 위해, 정비 시 구동시스템의 절연 여부를 보장하기 위해 사용된다.
  \item 구동시스템 전압 측정 포인트는 인버터의 +/- 파워라인에 연결되어야 하며 주 비상 정지 스위치 바로 옆에 장착되어야 한다. 구동시스템 전압 측정포인트의 전도체 부분은 반드시 차체와 전기적으로 격리되어야 한다.
  \item 구동시스템 전압 측정 포인트는 공구 없이 열 수 있는 절연 케이스로 보호되어야 한다. 또한 케이스가 열려있을 때 맨손으로 닿지 않도록 보호되어야 한다.
  \item 구동시스템 전압 측정 포인트는 적색의 4mm Shrouded Banana Jack을 사용해야 하고 "HV+", "HV-"라고 명확하게 표시되어야 한다.
  
  \item 각 구동시스템 전압 측정 포인트는 (+/-) 다음과 같은 전류 제한 저항으로 보호되어야 한다. \label{item:TSMP 전류 제한 저항}
    \begin{enumerate}
      \item Vmax ≤ 200 V DC: 5kΩ
      \item 200V DC < Vmax ≤ 400V DC: 10kΩ
      \item 400V DC < Vmax ≤ 600V DC: 15kΩ
    \end{enumerate}

    \fig{전압측정 포인트용 Banana Jack}{formula}{0.4}
    
  \item 구동시스템 전압 측정 포인트 옆에 접지된 저전압 시스템의(GLVS) 접지 측정 포인트가 장착되어야 한다. 이 측정 포인트는 GLVS의 Ground와 연결되어야 한다.
  \item 접지 측정 포인트는 흑색의 4mm Shrouded Banana Jack을 사용해야 하고 "GND"라고 명확하게 표시되어야 한다.
  \item 전압 측정 포인트를 지지하는 구조물은 최소 3점으로 지지해야 한다. 세 지지점이 모두 동일 직선상에 있을 수 없다.
  \item 모든 전압 측정 포인트는 직접 연결되어야 하며 퓨징을 금지한다.
\end{enumerate}

\section{고전압 표시등 - Tractive System Active Light, TSAL}
\begin{enumerate}
  \item 차량은 구동시스템이 활성화되었다는 것을 명확하게 보여줄 수 있는 고전압 표시등(TSAL)을 장착해야 한다. 고전압 표시등은 다음과 같은 조건을 만족하여야 한다.
    \begin{enumerate}
      \item 고전압 표시등은 GLVS로부터 전원을 공급받아 작동하여야 한다.
      \item 고전압 표시등은 구동시스템의 전압에 의해 직접 제어해야 한다. 소프트웨어 제어 또는 HV 릴레이를 닫는 제어 신호로 고전압 표시등을 활성화시키는 것은 허용되지 않는다.
      \item 다른 기능을 하면 안 된다.
    \end{enumerate}
    
  \item 고전압 표시등은 축전지박스 외부의 전압이 60V DC 또는 50V AC RMS를 초과하는 전압일 때 적색으로 2 \string~ 5Hz의 주파수로 연속으로 깜빡여야 한다.
  
  \item 고전압 표시등은 다음 세 가지 경우를 모두 만족할 때 항상 녹색으로 켜져 있어야 한다. \label{item:고전압 표시등 녹색}
    \begin{enumerate}
      \item 모든 AIR가 개방
      \item 초기 충전 릴레이 개방
      \item 축전지박스 내부 AIR의 차량 쪽 전압이 60V DC 또는 50V AC RMS 이하의 전압
    \end{enumerate}
    
  \item AIR와 초기 충전 릴레이의 개방 및 단락은 실제 기구적인 상태를 감지해야 하며 릴레이 제어신호를 사용할 수 없다. 실제 기구적인 상태를 감시하기 위해 회로를 이용하는 경우 \cref{item:AIR 개방}을 만족해야한다.
  \item 적색등을 위한 전압 감지 회로, 녹색등을 위한 전압 감지 회로, 각 릴레이 상태 감지 회로는 각각 독립적으로 작동해야 한다. 적색등과 녹색등 사이의 연동 동작을 허용하지 않으며 적색등과 녹색등이 동시에 켜지는 경우도 존재할 수 있다.
  
  \item 고전압 표시등은 반드시 다음 조건에서 명확하게 인식 가능하여야 한다.
    \begin{enumerate}
      \item 발광면은 메인 롤 후프에 의해 가려지는 각도를 제외한 모든 방향에서 확실히 보여야 한다.
      \item 지면으로부터 높이 1.6m, 고전압 표시등(TSAL)으로부터 반경 3m 이내에서 확실히 보여야 한다.
      \item 고전압 표시등이 켜졌을 때 가로길이는 양 끝 광원의 중심을 기준으로 75mm 이상이어야 한다. 또한 인접한 광원 간의 가로방향 거리는 광원의 중심을 기준으로 15mm 이하여야 한다. \label{item:TSAL 폭}
      \item 고전압 표시등은 무색의 투명한 케이스에 내장되어야 한다.
      \item \cref{item:TSAL 폭}은 적색, 녹색으로 켜졌을 때 각각 측정한다.
    \end{enumerate}
    
  \item 고전압 표시등은 반드시 다음 위치에 장착되어야 한다.
    \begin{enumerate}
      \item 차량의 가장 높은 메인 롤 후프에 가까운 곳
      \item 메인 롤 후프의 가장 높은 지점보다 75mm 이내로 낮은 곳
      \item 드라이버의 헬멧에 닿지 않는 곳
      \item 다른 표시등과 가깝지 않은 곳
    \end{enumerate}
    
  \item 고전압 표시등은 움직이지 않도록 프레임에 단단하게 고정되어야 한다.
  \item 드라이버가 쉽게 볼수 있는 곳에 "TS Off" 마크와 함께 녹색의 TS Off 표시등을 장착해야 하며 TS Off 표시등은 \cref{item:고전압 표시등 녹색}에 의해 동작하는 녹색등과 동일하게 동작되어야 한다. 이 TS Off 표시등은 밝은 햇빛에서도 선명하게 볼 수 있어야 한다.
\end{enumerate}

\section{운전 준비 완료 상태 - Ready-to-drive mode, RTD mode}
\begin{enumerate}
  \item 가속페달 위치 센서(APPS)의 입력에 모터가 응답할 준비가 되어있을 때 차량은 운전 준비 완료 상태(RTD mode)라고 한다.
  
  \item 운전 준비 완료 상태가 되기 위한 방법
    \begin{enumerate}
      \item 먼저 구동시스템을 활성화 시킨다.
      \item 드라이버가 브레이크 페달을 밟은 상태로 운전 준비 버튼을 눌러야 운전 준비 완료 상태가 된다.
      \item 드라이버가 브레이크 페달을 밟지 않은 상태에서 운전 준비 버튼을 누르더라도 운전 준비 상태가 완료되어서는 안된다.
      \item 차단 회로를 닫는 것만으로 차량이 운전 준비 완료 상태가 되어서는 안 된다.
      \item 차단 회로가 개방되었을 경우, 운전 준비 완료 상태는 초기화되어야 한다.
    \end{enumerate}
    
  \item 차량이 운전 준비 완료 상태가 되었을 때 특정한 소리를 내야 하고 이를 운전 준비 완료 상태음(Ready-to-drive Sound, RTDS)이라고 한다. 운전 준비 완료 상태음은 다음과 같은 조건을 만족하여야 한다.
    \begin{enumerate}
      \item 최소 1초, 최대 3초 동안 소리가 나야 한다.
      \item 최소 80dBA, 최대 90dBA 범위의 음향 레벨을 가져야 한다. 음향 레벨은 음원을 중심으로 동등 높이, 반경 2m 임의 위치에서 위 범위를 만족해야 한다.
      \item 쉽게 인식할 수 있어야 한다. 공격적으로 들리는 동물의 소리, 노래의 일부는 허용되지 않는다. 차량의 실제 주행음과 구분될 수 있어야 한다.
      \item 운전 준비 완료 상태가 되지 않은 상태에서 운전 준비 완료 상태음이 발생해서는 안된다.
      \item 운전 준비 완료 상태음이 발생하는 중에는 동력원이 구동되어서는 안 된다.
    \end{enumerate}
\end{enumerate}

\chapter{차단 회로와 퓨징 (Shutdown Circuit and Fusing): E-Formula 해당}

\section{구동시스템 활성화 및 차단 회로 - Tractive System Activation and Shutdown Circuit, SDC} \label{section:차단 회로}
\begin{enumerate}
  \item 드라이버는 BMS, IMD, BSPD, BOTS가 구동시스템을 비활성화시킨 상황을 제외하고는 외부 도움 없이 운전석 내에서 구동시스템을 활성화할 수 있어야 한다.
  \item HV 릴레이를 (AIR) 작동시키는 전류와 초기 충전 회로를 작동시키는 전류는 차단 회로를 통해 직접 이동한다. \label{item:차단 회로 전류}
  \item 차단 회로는 최소 2개의 주 비상 정지 스위치, 3개의 보조 비상 정지 스위치, 제동장치 미작동 감지장치(BOTS), BMS, IMD, BSPD, 필요한 모든 인터락(interlock)으로 구성되어야 한다.
  \item 접지된 저전압 시스템의(GLVS) 전원이 차단되면 구동시스템은 즉시 비활성화되어야 한다. \label{item:구동시스템 비활성화}
  
  \item 차단 회로가 열리거나 중단된 경우에 다음의 조건을 만족해야 한다. \label{item:차단 회로 작동}
    \begin{enumerate}
      \item 즉시 모든 HV 릴레이를 개방하여 구동시스템을 비활성화해야 한다.
      \item \cref{section:AIR}의 AIR와 초기 충전 회로를 포함하여 즉시 모든 축전지 전류 흐름이 중단되어야 한다.
      \item 차단 회로를 개방한 후 5초 이내에 구동시스템의 전압이 저전압 기준 미만으로 낮아져야 한다.
    \end{enumerate}
    
  \item BMS, IMD, BSPD에 의해 차단 회로가 개방된 경우 \label{item:차단 회로 개방}
    \begin{enumerate}
      \item 구동시스템은 수동으로 재설정(RESET) 하기 전까지 비활성화 상태를 유지해야 한다.
      \item 드라이버는 차량 내부에서 구동시스템을 재활성화할 수 없어야 한다.
      \item 보조 비상정지 스위치 또는 주 비상정지 스위치의 작동만으로 차단 회로의 재설정이 이루어져서는 안 된다. (별도의 재설정(RESET) 버튼 필요)
      \item 구동시스템은 차량 외부의 인원에 의해서만 재설정(RESET) 되어야 한다.
      \item 재설정 버튼은 차량 외부에 위치하며, 드라이버가 작동할 수 없는 위치에 있어야 한다.
    \end{enumerate}

  \item 차단 회로에 포함된 모든 회로는 전원이 차단된 상태에서 개방되도록 설계해서 각 회로가 HV 릴레이(AIR)를 제어하는 전류를 제거하도록 한다. \label{item:차단 회로 전원 차단}
  \item 차단 회로의 모든 기능이 올바르게 작동함을 입증할 수 있어야 한다. \label{item:차단 회로 입증}
\end{enumerate}

\section{주 비상 정지 스위치 - Master Switch} \label{section:주 비상 정지 스위치}
\begin{enumerate}
  \item 모든 차량은 2개의 주 비상 정지 스위치(LV 비상 정지 스위치, HV 비상 정지 스위치)를 반드시 장착해야 한다.
  
  \item LV 비상 정지 스위치는 다음 조건을 만족해야 한다.
    \begin{enumerate}
      \item 메인 비상 스위치로써 모든 전기시스템의 전원을 차단할 수 있어야 한다.
      \item 어떤 릴레이나 회로를 통하지 않고 직접 작동해야 한다.
      \item "LV"라고 명확하게 표시되어야 한다.
    \end{enumerate}
    
  \item HV 비상 정지 스위치는 다음 조건을 만족해야 한다.
    \begin{enumerate}
      \item 차단 회로를 개방한다.
      \item 어떤 릴레이나 회로를 통하지 않고 직접 작동해야 한다. 인터락(interlock)과 초기충전 회로를 제외하고 HV 릴레이(AIR) 이전의 마지막 스위치이다. HV 비상 정지 스위치와 HV 릴레이 사이의 인터락(interlock)은 반드시 HV 릴레이 코일의 +단자에서 연결되어야 한다.
      \item "HV"라고 명확하게 표시되어야 하고 ISO 7010-W012 (노란색 바탕에 검은색 번개가 있는 삼각형) 심볼을 붙여야 한다. (\cref{fig:고전압주의 스티커 (예)} 참조)
      \item 실수로 작동되는 것을 방지하기 위해 잠금 기능이 있어야 한다.
    \end{enumerate}
    
  \item 두 스위치는 다음 조건을 만족해야 한다.
    \begin{enumerate}
      \item 차량의 오른쪽, 메인 롤 후프 근처, 드라이버의 어깨 높이에 위치해야 한다.
      \item 외부에서 쉽게 작동시킬 수 있어야 한다.
      \item 회전 타입, 붉은색, "OFF" 포지션에서만 탈착이 가능한 키 방식이어야 한다. (\cref{fig:주 비상 정지 스위치} 참조)
      \item 쉽게 분리되지 않도록 메인 프레임에 견고하게 고정되어야 한다.
      \item "ON" 포지션은 수평 방향, "OFF" 포지션은 수직 방향이어야 하고 글자가 명확하게 표시되어야 한다.
    \end{enumerate}
    
  \item 전기시스템 관리자는(ESO) 차량이 정비 중일 때 HV 비상 정지 스위치가 잠겨 있는지 반드시 확인해야 한다.
\end{enumerate}

\section{보조 비상 정지 스위치 - Shutdown Button} \label{section:보조 비상 정지 스위치}
\begin{enumerate}
  \item 모든 차량은 3개의 보조 비상 정지 스위치를 반드시 장착해야 한다.
  \item 보조 비상 정지 스위치 중 어느 하나라도 작동하면 차단 회로를 개방해야 하며, \cref{item:차단 회로 작동}을 만족해야 한다.
  \item 보조 비상 정지 스위치는 Push-Pull 또는 Push-Rotate 비상 스위치 방식이어야 하며 버튼을 눌렀을 때 구동시스템을 비활성화 시켜야 한다.
  \item 2개의 보조 비상 정지 스위치는 드라이버의 머리 높이, 드라이버 공간의 뒤쪽으로 차량 양옆에 각각 장착한다. 스위치 버튼의 최소 직경은 40mm이며 버튼 가까운 곳에 국제 표준의 전기 심볼을 붙여야 하며 외부에서 손으로 조작하기 쉬운 위치에 장착해야 한다.
  \item 하나의 보조 비상 정지 스위치는 드라이버가 조작할 수 있는 운전석에 장착한다. 스위치 버튼의 최소 직경은 24mm이며 버튼 가까운 곳에 국제 표준의 전기 심볼을 붙여야 한다. 이 버튼은 드라이버가 벨트를 맨 상태에서도 쉽게 누를 수 있고 조향 휠 또는 차량의 다른 부품에 간섭을 받지 않는 위치에 장착되어야 한다.
\end{enumerate}

\fig{차단 회로}{formula}{1.0}

\section{절연 감시 장치 - Insulation Monitoring Device. IMD}
\begin{enumerate}
  \item 모든 차량은 구동시스템에 반드시 절연 감시장치를(IMD) 장착해야 한다.
  \item 절연 감시장치는 자동차용으로 승인된 Bender사({\hanja 社})의 A-ISOMETER ® iso-F1 IR155-3203 혹은 –3204 제품 또는 동등한 사용 가능함이 확인된 IMD 장치만 사용 가능하다. 동등한 사용 가능함의 확인은 진동에 대한 강건성, 동작 온도 범위, IP등급, 출력 신호의 가용성, 셀프테스트 능력에 의해 판단하며 이 내용은 전기시스템 보고서(ESF)에 기재되어야 한다.
  \item 절연 파괴, IMD 오류 등이 발생했을 때 절연 감시 장치는 즉시 차단 회로를 개방하여 구동시스템을 비활성화시켜야 한다. 절연 파괴 이후 구동시스템의 재활성화와 관련된 부분은 \cref{item:차단 회로 개방}을 참고해야 한다.
  \item IMD 응답값은 최대 구동 시스템전압 기준으로 500Ω/V 이상이어야 한다.
  \item IMD의 HV+와 HV-는 축전지박스 내부 AIR의 차량쪽에 연결되어야 한다.
  \item IMD에 GLVS Ground 이외에 두 개의 독립적인 차체 접지를 연결해야 하며 그 중 하나는 접지된 배터리박스 케이스 또는 접지된 IMD 케이스에 연결하고 다른 하나는 메인 롤 후프의 드라이버 어깨 지점에 연결한다. 이때 연결을 위한 도체 또는 전선은 구동시스템 최대 전압 이상의 정격전압을 가져야 한다. 접지가 연결되지 않은 경우 절연 감시장치가 작동되어 차단 회로를 개방해야 한다.
  \item 모든 차량은 절연 감시장치가 구동시스템을 비활성화시킬 경우 작동하는 절연 감시장치 표시등(IMD Indicator)을 장착해야 한다.
  \item 절연 감시장치 표시등은 밝은 햇빛에서도 잘 보이는 적색이어야 하고, 드라이버가 잘 볼 수 있는 위치에 장착되어야 한다. 표시등 주변에는 "IMD"라고 명확하게 표시되어야 한다.
  \item 절연 감시장치가 작동하여 절연 감시장치 표시등이 점등된 경우 RESET 되기 전까지 점등된 상태를 유지해야 한다.
\end{enumerate}

\section{제동시스템 타당성 장치 - Brake System Plausibility Device, BSPD}
\begin{enumerate}
  \item 모든 차량은 모터 제어기에서 양의 전류가 전달될 때 (차량을 전진시키는 전류), 동시에 차량에 강한 제동을 할 때 HV 릴레이(AIR)가 개방되도록 하는 제동시스템 타당성 장치(BSPD)를 장착해야 한다. 이 회로는 관련된 PCB가 다른 기능을 하지 않는 독립형 회로로 구성되어야 하며 소프트웨어 제어로 이루어져서는 안 된다. 제동시스템 타당성 장치는 다음과 같은 조건을 만족해야 한다. \label{item:BSPD 조건}
    \begin{enumerate}
      \item 회로를 작동하는 전류 제한은 공칭 축전지 전압을 기준으로 5kW 초과의 전력이 모터에 전달되는 것을 기준으로 설정한다.
      \item \cref{item:BSPD 조건}의 상황이 0.5초 이상 지속되면 HV 릴레이를 개방해야 한다.
      \item BSPD 시스템의 이상 상태 감지 기능을 포함하여 차단 회로를 개방할 수 있어야 한다.
      \item 전기안전검사를 위해 압력센서와 전류센서의 신호선은 커넥터 등을 이용해 각각 분리할 수 있어야 한다.
    \end{enumerate}
    
  \item 제동시스템 타당성 장치(BSPD) 작동 후 10초 이상의 시간 동안 \cref{item:BSPD 조건}의 상황이 해제된 상태가 유지되면 그 후 구동시스템을 회로가 스스로 재활성화할 수 있다. 그렇지 않은 경우 \cref{item:차단 회로 개방}을 참고하여 재활성화해야 한다.
  \item 전기 기술 검사(Electrical Tech Inspection) 동안 제동시스템 타당성 장치를 입증하기 위한 방법을 (차량에 강한 제동을 달성함과 동시에, 5kW를 달성하기 위해 적절한 신호를 보내는 방법) 고안해야 한다.
  \item 제동 상황을 감지하기 위해 브레이크 압력센서를 반드시 사용해야 하며 제동시스템 타당성 장치 작동을 위한 압력센서의 기준값은 잠기는 휠이 없고 브레이크 압력이 30bar 미만인 값으로 설정해야 한다.
  \item 모든 차량은 제동시스템 타당성 장치가 구동시스템을 비활성화 시킬 경우 작동하는 제동시스템 타당성 장치 표시등(BSPD Indicator)을 장착해야 한다.
  \item 제동시스템 타당성 장치가 작동하여 제동시스템 타당성 장치 표시등이 점등된 경우 RESET 되기 전까지 점등된 상태를 유지해야 한다.
  \item 제동시스템 타당성 장치에 사용되는 센서들은 축전지 박스 내에 존재할 수 없다.
\end{enumerate}

\section{과전류 보호 장치 – Overcurrent Protection}
\begin{enumerate}
  \item 모든 전기시스템(LV, HV 모두 해당)은 반드시 적절한 과전류 보호 장치를 장착해야 한다. 과전류 보호 장치는 시스템이 차단할 수 있는 전류를 초과하는 전류가 흘렀을 때 다른 장치를 안전하게 보호하기 위해 시스템의 전류를 차단시켜야 한다.\\
    Note: 퓨즈(Fuse)는 과전류 보호 장치의 가장 일반적 형태이다.
  \item 모든 과전류 보호 장치는 그 장치가 보호하려는 시스템의 가장 높은 전압보다 높은 정격 전압을 가져야 한다. DC에 사용되는 과전류 보호 장치는 DC에 대한 정격을 가져야 하고 시스템 전압 이상의 DC 정격 전압을 가져야 한다.
  \item 모든 과전류 보호 장치는 시스템의 이론적인 단락(short) 전류보다 높은 차단용량(interrupt rating)의 전류를 가져야 한다. (차단 용량이 낮은 과전류 보호 장치에 높은 단락 전류가 흐르면 과전류 보호 장치가 끊어진 후에도 아크 방전으로 인해 폭발하거나 화재가 발생할 수 있음)
\end{enumerate}

\chapter{축전지(Accumulator): E-Formula 해당}

\section{축전지 – Battery, Accumulator containor} \label{section:축전지}
\begin{enumerate}
  \item 구동시스템에서 사용하는 전기 에너지를 저장하는 모든 셀을 축전지라고 한다. 용융염 전지나 열전지를 제외한 모든 유형의 축전지가 허용된다. 연료전지는 금지한다.
  \item 축전지의 가장 기본 단위는 셀이며, 셀을 여러 개 붙인 것이 세그먼트, 이 세그먼트를 여러 개 붙인 것을 축전지팩, 축전지팩을 보호하는 박스를 축전지박스라고 한다.
  \item 각 세그먼트의 최대 질량은 12kg을 넘을 수 없고, 최대전압은 120V를 넘을 수 없고, 최대 에너지는 6MJ을 넘을 수 없다. 최대 에너지는 셀의 정격용량과 각 세그먼트의 최대전압의 곱으로 계산한다. 축전지팩을 세그먼트로 분할한 의도는 축전지 정비와 관련된 위험을 줄이기 위함이다. \label{item:세그먼트 제한}
    \[
      ※\ \mathrm{Max\_energy\ [J]} = V_{\mathrm{cell,max}}\ [\mathrm{V}] \times C_{\mathrm{normal\ capacity}}\ [\mathrm{Ah}] \times N_{\mathrm{cell,segment}}\ [\mathrm{No.}] \times 3600\ [\mathrm{sec/hour}]
    \]
  \item 축전지 박스는 차량에서 분리할 수 있어야 한다.
  \item 여분의 축전지가 사용될 경우 반드시 크기, 무게, 형태가 동일해야 하고 전기시스템 검사에서 설명 및 확인을 받아야 한다.
  \item 전기시스템 검사에서 축전지박스 내부를 눈으로 쉽게 확인할 수 있어야 한다. 단, 상용품을 사용하거나 외부 제작에 의해 축전지박스에 접근하기 어려울 경우, 세부적인 내부 사진과 조립 과정의 사진을 반드시 제공해야 한다.
  
  \item BMS(Battery Management System)
    \begin{enumerate}
      \item 모든 축전지는 구동시스템이 활성화되거나 축전지가 충전기에 연결되어 있을 때 반드시 BMS에 의해 감시되어야 한다.
      \item BMS는 셀이 셀 데이터시트에 표시된 최소/최대전압 범위 내에 유지되도록 모든 셀의 전압을 지속적으로 감시해야 한다. 단일 셀들이 병렬로 직접 연결되어 있는 경우는 하나의 병렬 전압만 감시하면 된다.
      \item BMS는 셀을 셀 데이터시트에 표시된 최대 허용 온도와 60℃ 중 낮은 온도 이하로 유지하도록 축전지의 온도를 지속적으로 감시해야 한다.
      \item 셀 온도는 각 셀의 음극 단자에서 측정해야 하며 온도센서는 음극 단자나 버스바(Busbar)에 직접 접촉해야 한다. 센서가 버스바(Busbar) 위에 있을 경우 셀 단자로부터 10mm 이내에 위치해야 한다. \label{item:BMS 온도 센서}
      \item 리튬 기반 셀은 최소 20\% 셀의 온도가 BMS에 의해 감시되어야 하며 감시되는 셀은 축전지 박스 내에 균등하게 분포해야 한다. 센서에 의해 감지된 모든 셀에 대해 \cref{item:BMS 온도 센서}을 만족하는 경우, 하나의 센서로 다수의 셀을 감시하는 것은 허용된다.
      \item BMS는 셀 제조사의 데이터 시트상의 임계 전압, 온도가 감지되었을 때 반드시 HV 릴레이를 개방시켜 구동시스템을 비상 정지시켜야 한다.
      \item BMS는 셀의 전압 감지부에서 과전류가 감지되었을 경우, 구동시스템을 비상 정지시켜야 한다.
      \item BMS가 구동시스템을 비상 정지시켰을 때 드라이버가 잘 보이는 곳에 "BMS"라고 표시된 적색 LED가 켜져야 한다.
    \end{enumerate}
\end{enumerate}

\section{축전지 세그먼트의 연결 및 분리} \label{section:세그먼트 연결}
본 규정은 세그먼트간 연결뿐만 아니라 세그먼트와 다른 부품(AIR 등) 간의 연결에도 적용된다.
\begin{enumerate}
  \item 축전지 세그먼트는 커넥터에 의해 전기적으로 연결 및 분리되어야하며, 다음을 만족해야 한다.
    \begin{enumerate}
      \item 각 세그먼트는 \cref{item:세그먼트 제한}을 만족해야 한다.
      \item 각 세그먼트의 +/-에 모두 적용되어야 한다.
    \end{enumerate}
    
  \item 축전지 세그먼트 커넥터는 다음을 만족해야 한다.
    \begin{enumerate}
      \item 릴레이, 컨택터 또는 스위치는 세그먼트 커넥터로 인정되지 않는다.
      \item 설계된 설정으로만 연결이 가능하도록 구성해야 한다.\\
        (커넥터 방향, 전선길이 등을 이용하여 물리적으로 제한되도록 설계)
      \item 도구 사용 없이 연결 및 분리가 가능해야 한다.
      \item 커넥터는 진동 등에 의해 분리가 되지 않도록 하는 기능이 있어야 한다.
      \item 커넥터의 표면은 비전도성 재질로 구성되어야 한다.
    \end{enumerate}
    
  \item 축전지 박스가 열려있거나 세그먼트를 제거할 때, 세그먼트 커넥터를 분리해야 한다. 안전을 위해 모든 세그먼트 커넥터를 분리하는 것을 권장하며 일부 커넥터만을 분리하는 경우 연결된 세그먼트들의 최대 전압은 120V를 넘을 수 없고, 최대 에너지의 합은 6MJ을 넘을 수 없다.
  \item 세그먼트는 축전지박스에 기계적 결합을 통해 고정되어 하중을 전달해야 한다. 볼트와 너트를 사용하는 경우 \cref{item:체결장치의 풀림 방지}을 만족해야 한다.
\end{enumerate}

\section{축전지박스의 전기적 구성 – Accumulator Container Electrical System}
\begin{enumerate}
  \item 축전지 박스가 도전체로 만들어졌다면 축전지의 세그먼트와 셀의 극은 반드시 축전지박스의 외벽과 내벽, 바닥, 커버로부터 절연되어야 한다. 이때 절연 재료는 구동시스템의 최대전압에 적합해야 하며 \cref{section:난연성 재료}를 만족하는 난연성이어야 한다. 도전체로 만들어진 축전지 박스 외부의 전도성 표면은 LV System Ground와 접지되어야 한다.
  \item 모든 축전지박스는 하나 이상의 메인 과전류 보호 장치와 두 개 이상의 HV 릴레이(AIR)를 포함해야 한다. HV 릴레이 및 메인 과전류 보호 장치는 축전지박스의 전도성 표면과 \cref{section:난연성 재료}를 만족하는 난연성의 전기적 절연 재료를 이용하여 분리되어야 한다.
  \item 축전지 박스 내부에는 다음 항목을 제외한 그 어떤 접지된 저전압 시스템(GLVS)도 포함되지 않아야 한다. (예외항목: HV 릴레이(AIR), 초기충전 및 방전회로, HV DC/DC 컨버터, BMS, IMD, TSAL 구성 요소, 냉각용 팬).\\
    축전지 박스 내부에 포함된 모든 LV 배선의 전기 절연(galvanic isolation)에 대한 내용은 전기시스템 보고서(ESF)에 포함되어야 한다.
  \item 축전지 박스 내부의 각 세그먼트 사이와 세그먼트 위쪽 부분은, 정비 시 실수로 부품이나 공구가 떨어졌을 때 발생할 수 있는 사고를 방지하기 위하여 \cref{section:난연성 재료}를 만족하는 난연성의 견고한 절연 재료를 이용하여 전기적으로 절연시켜야 한다. 공극을 이용한 절연은 허용되지 않는다.
  \item 고전류가 흐르는 경로에서 납땜을 이용한 셀 간 접촉 및 연결은 금지된다. BMS의 전압 감지를 위해 셀에 전선을 납땜하는 것은 허용된다.
  \item 축전지 박스 내부에 사용되는 모든 전선은 구동시스템, 접지된 저전압 시스템의 구분없이 최대 구동 전압에 만족하는 정격전압을 가져야 한다.
  
  \item 축전지 전압 표시장치 (Accumulator voltage indicator)
    \begin{enumerate}
      \item 모든 축전지박스는 축전지박스 내부 AIR가 닫힌 상태에서 구동 시스템 전압의 차량쪽 전압이 60V DC 이상일 때, 적색으로 점등되는 LED나 눈에 띄는 표시장치가 있어야 한다.
      \item 소프트웨어 또는 AIR를 닫는 제어신호를 이용하여 축전지 전압 표시장치를 제어하는 것은 허용되지 않으며 하드웨어 장치를 이용하여 직접 제어해야 한다.
      \item 축전지 전압 표시장치는 축전지박스가 LV 시스템과 연결이 끊기거나 차량에서 분리가 되어도 항상 반드시 작동해야 한다.
      \item 축전지 전압 표시장치는 축전지박스가 차량에서 분리되었을 때 명확하게 볼 수 있는 곳에 위치해야 한다.
      \item 축전지 전압 표시장치에 "전압 표시장치" 혹은 "Voltage Indicator"라고 표시되어야 한다.
    \end{enumerate}
\end{enumerate}

\section{축전지 박스의 기계적 구성 – Accumulator Container Mechanical System}
\begin{enumerate}
  \item 모든 축전지 박스는 프레임 내부에 위치해야 한다.
  \item 축전지 박스는 기계적으로 강인한 재료를 사용해야 한다.
  \item 축전지 박스는 화재에 대한 내성이 있어야 한다.
  \item 모든 축전지 박스는 차량의 측면과 후면의 충돌로부터 보호되어야 한다.
  \item 축전지 박스의 바닥은 2mm 이상의 철판 또는 3mm 이상의 알루미늄을 사용해야 하며 수직 방향의 외벽과 커버는 1mm 이상의 철판 또는 2mm 이상의 알루미늄을 사용해야 한다.
  
  \item 축전지 박스 내부에는 셀 및/또는 세그먼트를 공간적으로 분리하는 내벽이 존재해야 하며 다음을 만족해야 한다.
    \begin{enumerate}
      \item 1mm 이상의 철판 또는 2mm 이상의 알루미늄을 사용해야 한다.
      \item 인접한 세그먼트를 모두 가릴 수 있을 만큼의 최소 높이와 폭을 가져야 한다.
      \item 제거가 가능한, 축전지 박스의 바닥과 수평한 내벽을 사용할 경우 각 외벽과 최소 한 개 이상의 체결지점을 가져야 한다.
    \end{enumerate}
    
  \item 다음 중 하나를 사용하여 축전지 박스를 차량에 부착해야 하며, \cref{item:축전지박스 마운트}을 만족해야 한다. \label{item:축전지박스 하중}
    \begin{enumerate}
      \item 축전지 부착 – 코너 고정
        \begin{enumerate}
          \item 모든 구성에는 최소한 8개의 고정점이 필요하다.  
            \begin{itemize}
              \item 다중 축전지 세그먼트의 직사각형 배열의 각 코너마다 하나의 고정점  
              \item 사각형이 아닌 배열에는 최소 개수 이상의 패스너가 필요하다.\\
                예: 돌출된 L모양은 10개의 볼록한 코너에 고정점이 필요하다. (L 내부의 코너는 볼록하지 않다.) 육각형에는 12개의 부착물이 필요하다.
            \end{itemize}
            
          \item 각 모서리의 기계적 연결은 다음과 같아야 한다.  
            \begin{itemize}
              \item 세그먼트 모서리에서 50mm 이내  
              \item 다른 모서리의 기계적 연결부와 최소 50mm 간격  
              \item 동일한 패스너 지름 및 재질  
            \end{itemize}
            
          \item 각 부착 지점은 다음의 하중을 모든 방향에 대해서 견딜 수 있어야 한다.  
            \begin{itemize}
              \item 시험 하중은 축전지 박스의 1/4 질량에서 40g 가속도가 발생했을때의 하중으로 한다.
            \end{itemize}
        \end{enumerate}
        
      \item 축전지 부착 – 하중 기반
        \begin{enumerate}
          \item 사용해야 하는 부착 점 수는 축전지 박스의 총 질량에 따라 아래를 만족해야 한다.  

          \begin{table}[H]
            \centering
            \begin{tblr}{
              width = 0.5\linewidth,
              rows = {ht=1.0\baselineskip},
              colspec = {|X[c,m]|X[c,m]|},
              row{1} = {bg=gray!30, font=\bfseries},
              hlines,
              vlines
            }
              축전지 박스 무게      & 최소 고정점 개수 \\
              20kgf 미만             & 4  \\
              20kgf \string~ 30kgf  & 6  \\
              30kgf \string~ 40kgf  & 8  \\
              40kgf 이상             & 10 \\
            \end{tblr}
          \end{table}

          \item 각 부착 지점은 마운트, 백 플레이트 및 백 플레이트 인서트를 포함하며, 모든 방향에서 20kN을 견딜 수 있어야 하고 이를 시험 하중으로 한다.
          \item 각 부착 지점이 별도의 부착 지점으로 계산되려면 50mm 이상 떨어져 있어야 한다.
        \end{enumerate}
    \end{enumerate}
    
  \item 축전지박스의 모든 부착지점과 섀시 마운트 또는 모노코크 마운트는 각각 다음을 만족해야 한다. \label{item:축전지박스 마운트}
    \begin{enumerate}
      \item 부착에 사용되는 모든 패스너는 직경이 6mm 이상이어야 한다. 또한 2개의 5mm 또는 3개의 4mm볼트로 6mm 볼트 1개를 대체할 수 있다. \label{item:축전지박스 패스너}
      \item 모든 패스너는 \cref{item:축전지박스 하중}의 시험 하중에 대한 순수 전단을 견뎌야 한다.
      
      \item 축전지 박스를 섀시에 부착하는 모든 마운트는 다음과 같아야 한다.  
        \begin{itemize}
          \item 최소 2mm 두께의 강철 또는 최소 4mm 두께의 알루미늄으로 제작  
          \item 굽힘 하중을 전달하는 보강재가 있어야 한다.\\  
            예시) 마운트가 부착되는 면을 아래 \cref{fig:굽힘하중 보강재}과 같이 XY 평면이라고 할 때, X와 Y축 굽힘 하중을 전달하는 보강재가 있어야 한다.
        \end{itemize}
        
        \fig{굽힘하중 보강재}{formula}{0.3}
        
      \item 모든 축전지 마운트, 섀시 마운트, 모노코크 마운트는 \cref{item:축전지박스 하중}의 시험 하중에 대한 굽힘, 순수 전단, 순수 인장을 견딜 수 있어야 한다. 용접 시 전단, 본드 접합 시 전단 및 인장에서 시험 하중을 견뎌야 한다.
    \end{enumerate}
    
  \item 외벽, 내벽, 바닥 및 커버를 결합하는 방식은 다음 중 하나를 사용해야 한다.
    \begin{enumerate}
      \item 용접
        \begin{enumerate}
          \item 용접은 연속(Continuous) 혹은 단속(Interrupted)이어야 한다.
          \item 단속인 경우 용접길이(Welding length)가 용접 사이 공간 길이(Space length)보다 커야 한다.
          \item 단속의 경우 각각의 용접길이는 25mm 이상이어야 한다.
        \end{enumerate}

      \item 볼트 및 너트
        \begin{enumerate}
          \item 결합된 볼트와 너트의 강도가 \cref{item:축전지박스 패스너}과 동등하거나 그 이상을 입증해야 한다.
          \item \cref{item:체결장치의 풀림 방지}을 만족해야 한다.
        \end{enumerate}
    \end{enumerate}
    
  \item 축전지 박스는 축전지 박스 자체, 축전지 박스와 세그먼트간의 체결을 고려하여, 축전지 박스가 샤시와 체결되었을 때 다음과 같은 가속도에서 안전하도록 설계 및 제작되어야 한다. \label{item:축전지박스 가속도}
    \begin{enumerate}
      \item 세로 방향으로 40g (앞/뒤)
      \item 가로 방향으로 40g (왼쪽/오른쪽)
      \item 수직 방향으로 20g (위/아래)
    \end{enumerate}
    
  \item 축전지 박스는 차량에서 분리하여 케이스를 열고 검사할 수 있도록 설치하여야 하며 장착할 때나 차량으로부터 분리할 때나 항상 커버가 닫혀 있어야 한다.
  \item 모든 축전지 박스에는 ISO 7010-W012 (노란색 바탕에 검은색 번개가 있는 삼각형, \cref{fig:고전압주의 스티커 (예)} 참조) 심볼 또는 흰색 바탕에 적색 번개 표시의 심볼 스티커를 부착해야 하고 스티커에는 "High Voltage" 또는 "고전압"의 글자가 포함되어야 한다.

  \fig{고전압주의 스티커 (예)}{formula}{0.3}

  \item 밀폐된 구조의 축전지 박스는 압력 해제장치를 가지고 있어야 한다.
  
  \item 축전지 박스의 외벽과 내벽의 구멍은 아래를 만족해야 한다.
    \begin{enumerate}
      \item 구멍은 원형이어야 하고, 슬롯 형상은 금지한다. 구멍의 직경은 10mm 이하여야 한다.
      \item 축전지 박스 외벽에는 배선, 체결 또는 환기를 위한 구멍만 허용된다. 축전지 박스 내 환기를 위한 구멍에는 흡습 또는 투습을 막는 필터 또는 투습을 막기 위한 기구적 덕트를 설치해야 한다. 또한 드라이버를 직접적으로 바라보는 방향으로 설치를 허용하지 않는다.
      \item 축전지 박스 내벽의 구멍은 허용하나, 내벽이 인접한 세그먼트 면적의 75\% 이상을 가릴 수 있어야 한다.
    \end{enumerate}
    
  \item 모든 팀은 안전검사에 앞서 축전지 박스의 기계적 시스템에 대해 명확하게 구성된 보고서를 제출하여야 하며 양식과 제출 기한은 대회 홈페이지에서 별도로 공지한다.
  
  \item 축전지 박스 기계적 시스템 보고서에는 다음 사항 등이 포함되어야 한다.
    \begin{enumerate}
      \item 축전지 박스의 외벽, 내벽, 방화벽에 대한 두께 및 물성 정보
      \item 축전지 박스의 차량 내 배치 및 부착 지점 정보
      \item \cref{item:축전지박스 하중}, \cref{item:축전지박스 마운트}, \cref{item:축전지박스 가속도}에 대한 계산/해석/시험 자료
    \end{enumerate}
    
  \item 축전지 박스 기계적 시스템 보고서를 기한 내에 제출하지 않거나 정해진 기한 내에 검토자로부터 최종 승인을 받지 못한 팀은 대회에 참가할 수 없다.
  \item 축전지 박스 내부의 회로 보드와 모든 LV시스템 구성요소들(배선 제외)은 볼트, 리벳 등의 기계적인 결합을 통해 견고하게 장착되어야 한다. 접착제나 케이블 타이를 이용한 고정은 허용하지 않는다.
\end{enumerate}

\section{HV 릴레이 - Accumulator Isolation Relay(s), AIR} \label{section:AIR}
\begin{enumerate}
  \item 모든 축전지박스에는 반드시 두개 이상의 HV 릴레이(AIR)가 장착되어야 한다.
  \item HV 릴레이는 반드시 축전지팩의 양쪽 단자를 개방할 수 있어야 한다. HV 릴레이가 개방되었을 때, 반드시 축전지박스 외부에 HV가 존재하면 안 된다. \label{item:AIR 개방}
  \item HV 릴레이는 반드시 "상시 개방(Normally Open)" 유형이어야 한다. 작동신호가 들어오지 않았을 때는 개방(Open)상태이고 작동신호가 들어왔을 때 닫힘(Close) 상태이어야 한다.
  \item 축전지 및 구동시스템 과전류 보호 장치는 HV 릴레이의 최대 차단전류보다 낮은 정격전류를 가져야 한다.
\end{enumerate}

\section{구동시스템 정비 및 충전 – Working on Tractive System and Accumulator Charging}
\begin{enumerate}
  \item 구동시스템 및 축전지박스 정비는 다음과 같은 규정을 준수하며 진행해야 한다.
    \begin{enumerate}
      \item 대회장 내에서 전기 기술 검사(Electrical Tech Inspection) 동안 별도로 지정된 구역에서만 축전지박스를 열거나 정비할 수 있다.
      \item 반드시 전기시스템 관리자(ESO)의 허가 이후에 정비를 진행할 수 있다.
      \item 반드시 적절한 절연 공구를 이용해야 한다.
      \item 구동시스템의 일부가 활성화된 상태로 노출되어 있거나 축전지 정비를 진행할 때마다, 참여하는 모든 팀원은 반드시 측면 보호물이 있는 보안경을 착용해야 한다.
    \end{enumerate}
    
  \item 축전지 충전은 다음과 같은 규정을 준수하며 진행해야 한다.
    \begin{enumerate}
      \item 대회장 내에서 별도로 지정된 구역에서만 구동시스템 축전지를 충전할 수 있다. 충전 구역에서 축전지를 충전하는 것은 언제나 허용된다.
      \item 충전을 위해서 축전지박스는 반드시 차량에서 분리되어야 하며 축전지박스 핸드 카트에 실어서 충전해야 한다. 축전지 충전을 위해서 축전지는 \cref{item:충전 차단 회로}의 충전 차단 회로와 연결되어야 한다. 축전지박스에 들어있는 상태로 차량 밖으로 꺼내서 충전해야 한다.
      \item 축전지박스에는 팀 이름, 충전에 걸리는 시간, 전기시스템 관리자(ESO)의 전화번호가 표시되어 있어야 한다.
      \item 충전 절차에 대해서 잘 알고 있는 팀원 한 명이 충전하는 동안 반드시 대기하고 있어야 한다.
    \end{enumerate}
    
  \item 축전지 충전을 위한 충전기는 다음과 같은 조건을 만족해야 한다.
    \begin{enumerate}
      \item 전기 기술 검사(Electrical Tech Inspection)에서 확인된 충전기만 사용할 수 있다.
      \item 충전기의 (AC) 입력과 (DC) 출력은 반드시 전기적 절연을(galvanic isolation) 이루어야 한다.
      \item 충전기의 하우징이 전도체일 경우 AC 입력의 접지에 반드시 연결되어야 한다.
      \item 충전기의 모든 연결 부분은 절연되어야 하고 덮개로 보호되어야 한다.
      \item 충전기의 고전압 배선은 \cref{item:구동시스템}에 정의된 구동시스템이며, 주황색이어야 한다. 단, \cref{item:롤오버}을 만족하지 않아도 된다.
      \item 충전기 커넥터는 축전지박스에 올바르게 연결된 경우에만 커넥터가 활성화되도록 하는 기능을 포함해야 한다.
      \item 충전 중에는 차량기술규정 \cref{section:TSMP}에 명시된 두 개의 구동시스템 측정 포인트와 접지된 저전압 시스템의 접지 측정 포인트가 반드시 설치되어야 한다. 구동시스템 측정 포인트는 \cref{item:TSMP 전류 제한 저항}의 전류 제한 저항으로 보호되어야 한다.
    \end{enumerate}
    
  \item 축전지 충전을 위한 충전 차단 회로(Charging Shutdown Circuit)는 다음 조건을 만족해야 한다. \label{item:충전 차단 회로}
    \begin{enumerate}
      \item 최소 1개의 충전기 차단 버튼(Shutdown Button), IMD, BMS로 구성되어 있어야 한다.
      \item 충전기 차단 버튼은 Push-rotate 비상 스위치 방식이어야 하고, 스위치 최소 직경은 25mm이어야 한다.
      \item BMS와 IMD에 의해 충전 차단 회로가 개방된 경우, 수동으로 재설정할 때까지 구동시스템은 비활성화 상태를 유지해야 한다.
      \item 충전 중 BMS와 IMD는 반드시 활성화되어야 하며 문제가 발생될 때 충전차단회로를 개방한다. BMS와 IMD의 오류가 발생했을 때, 각각의 오류에 대한 표시등을 포함해야 한다.
      
      \item 충전 차단 회로가 개방되었을 경우 다음을 만족해야 한다.
        \begin{enumerate}
          \item \cref{section:AIR}의 AIR와 초기 충전 회로를 포함하여 즉시 모든 축전지 전류 흐름이 중단되어야 한다.
          \item 5초 이내에 구동시스템의 전압이 저전압 기준 미만으로 낮아져야 한다.
          \item 충전기가 꺼져야 한다.
          \item 수동으로 재설정할 때까지 충전기는 비활성화 상태를 유지한다.
        \end{enumerate}
        
      \item \cref{item:차단 회로 전류}, \cref{item:차단 회로 전원 차단}, \cref{item:차단 회로 입증}을 만족해야 한다.
    \end{enumerate}
    
  \item 팀은 축전지박스를 운반할 수 있는 핸드카트를 반드시 가지고 있어야 한다. 핸드카트는 다음과 같은 조건을 만족해야 한다.
    \begin{enumerate}
      \item 대회장 내에서 축전지박스를 운반할 때는 반드시 4개 이상의 바퀴를 갖는 핸드카트를 사용해야 한다.
      \item 축전지박스의 하중을 견디고 운반할 수 있어야 하며 이동 중 축전지 박스가 카트 밖으로 이탈되지 않도록 축전지박스와 핸드카트는 기계적 결합으로 고정되어야 한다.
      \item 평상시 브레이크가 잡힌 상태여야 하며, 손으로 브레이크 장치를 해제한 상태에서만 움직일 수 있어야 한다. (예를 들면, 공항 카트) 축전지 박스가 완전히 적재된 상태에서 멈춘 상태를 유지할 수 있어야 한다. 브레이크 장치를 해제한 뒤 쉽게 움직일 수 있어야 한다.
      \item 핸드카트는 모터 등의 동력장치를 사용할 수 없다.
      \item 배터리박스와 직접적으로 접촉하는 부분은 절연 처리되어야 한다.
      \item 축전지박스가 핸드카트에 장착된 후 일반적으로 작동되는 상황에서 진동과 충격으로부터 보호되어야 한다. (예시: 핸드카트 바퀴에 공기 타이어 사용, 고무 혹은 우레탄 소재의 활용 등)
      \item 핸드카트를 움직이는 사람을 보호하기 위한 방화벽이 설치되어야 한다. 방화벽은 카트 바닥면과 동일한 폭을 가져야 하며 사람의 다리와 몸을 적절히 보호할 수 있어야 한다. 또한 사람의 손을 보호하기 위해 핸드카트의 손잡이 중심보다 30cm 이상의 높이를 가져야 한다. 방화벽은 \cref{item:e-formula 방화벽}을 만족해야 하나, 이 중 \cref{item:접지}은 만족하지 않아도 된다.
    \end{enumerate}
\end{enumerate}

\section{에너지 미터 - Energy Meter}
\begin{enumerate}
  \item 동적이벤트 참가 시 운영위원회에서 제공하는 에너지미터를 장착하여야 한다. 에너지미터에 대한 세부적인 사항은 별도 공지한다.
  \item 에너지미터는 축전지박스 출력단의 HV 전류와 전압을 각각 측정하여 저장하며 이벤트 중 허용된 출력 또는 전류를 초과하는지 확인하는 용도로 사용된다.
  \item 전류 측정은 축전지 출력단의 – 배선을 양쪽 단자에 직렬로 연결해야 하므로 에너지미터를 장착할 수 있도록 배선, 링 터미널 등을 규격에 맞게 준비하여야 한다.
  \item 에너지 미터는 외부에서 접근하기 쉬운 곳에 장착해야 하며 모든 축전지 장치에서 나오는 HV 배선은 한 점에 모여서 에너지 미터를 통과하여 모터 컨트롤러로 연결되도록 차량을 제작하여야 한다. 에너지미터는 축전지박스 내부에 장착할 수 없고 15분 이내에 장착 또는 제거, 교체할 수 있어야 한다.
  \item 주행 중 최대 출력 또는 최대 전압을 연속으로 100ms 이상 초과했거나 출력 또는 전압의 500ms 이동 평균이 최대 출력 또는 최대 전압을 초과하였을 때 출력과 전압 제한을 위반한 것으로 간주한다.
  \item 팀의 실수로 출력 에너지미터의 데이터가 잘못되었을 경우 DNF 처리한다.
  \item 데이터 조작 또는 조작 시도를 할 경우 DNF 처리한다.
  \item 가속, 스키드패드, 짐카나(오토크로스)에서 출력 제한을 위반할 경우 해당 이벤트는 DNF 처리한다.
  \item 내구레이스 경기에서 출력 제한을 위반할 경우 1회당 60초 페널티를 부과한다.
  
  \fig{에너지미터 연결 개략도}{formula}{0.8}
\end{enumerate}

\chapter{참가 차량의 식별}

\section{차량 출전번호} \label{section:차량 출전번호}
\begin{enumerate}
  \item 차량 출전번호(엔트리)는 조직위원회에서 제작하여 현장 등록 시 지급한다.
  \item 참가차량은 이에 대비하여 차량 제작 시 출전번호를 부착할 수 있는 공간을 확보해 두어야 한다. (좌, 우측 및 전방 각 1개, 총 3개이며 크기는 300mm × 300mm이다.)
  \item 측면에 부착하는 차량 출전 번호는 측면에서 확인 가능 하도록 하여야 한다.
\end{enumerate}

\section{차량검사 스티커 공간}
차량검사를 통과하였을 때 증명하는 스티커를 부착하기 위해 차량 외부에 가로 200mm×세로100mm 공간을 비워두어야 한다.

\chapter{기록계측장치}

\section{트랜스폰더}
\begin{enumerate}
  \item 차량 주행 중의 시간 계측 시스템으로 트랜스폰더를 사용한다.
  \item 트랜스폰더는 경기 전 조직위원회에서 제공하며 사용 후 반드시 반납하여야 한다.
\end{enumerate}

\section{트랜스폰더 부착 방법}
\begin{enumerate}
  \item 트랜스폰더는 트랜스폰더 홀더를 사용하여 거치하며 트랜스폰더 및 홀더가 경기 중 탈락되지 않도록 견고히 부착하여야 한다. 경기 중 트랜스폰더의 이탈로 인한 계측 불가 시 DNF처리 될 수 있다.
  \item 방향: 지면 방향으로 트랜스폰더가 세워지도록 홀더를 고정한다. (\cref{fig:트랜스폰더 설치 위치 및 형상} 참고)
  \item 위치: 트랜스폰더의 위치는 차량의 측면에 지면에서 최대한 가까운 위치에 장착한다. 트랜스폰더와 지면 사이에 무선 송수신을 방해하는 어떤 부품도 있어서는 안 된다.
  \item 트랜스폰더가 경기 중에 작동하지 않으면 흑기(Black Flag)가 해당 차량 번호와 함께 게시될 수 있으며 해당 차량은 즉시 피트로 돌아와 조치를 취하여야 한다.
  \item 트랜스폰더 홀더는 각 팀에서 준비하며, 주최 측에서 지급한 트랜스폰더를 분실하거나 파손하였을 경우 변상하여야 한다.
\end{enumerate}

\fig{트랜스폰더 및 홀더}{formula}{0.8}
\fig{트랜스폰더 설치 위치 및 형상}{formula}{0.7}

\chapter{기타 규정}

\section{공력장치 및 디퓨져 – Aero Dynamics and Ground Effect}
\begin{enumerate}
  \item 기본조건\\
    사고 시 드라이버의 탈출을 방해할 수 있는 어떠한 부품 또는 구조물도 있으면 안 된다.
  \item 스포일러 모서리\\
    스포일러 모서리는 날카로워서는 안 된다.
  \item 디퓨져 보조 장치\\
    팬 등의 사용으로 디퓨져로부터 강압적으로 공기를 빨아들이는 장치를 사용하여서는 안 된다.
    
  \item 공력장치 일반규정  
    \begin{enumerate}
      \item 공기저항의 감소 및 다운포스 생성을 목적으로 공기 흐름을 유도하는 차량의 부품 Wing, Undertray, Splitter, Endplates, Vanes 등을 예로 들었으나 이외의 장치도 가능하다.
      \item 냉각만을 목적으로 설치된 fan을 제외한 어떤 동력을 이용한 장치도 차량 하부의 공기를 빼내거나 흐름을 만들기 위해 사용되면 안 된다. Ground effect를 위해 동력을 사용하는 것을 금지한다.
      \item 모든 공력장치는 정적상태에서 충분히 튼튼하게 설계되어야 하며, 주행 중 공력장치가 필요이상으로 심하게 진동하거나 움직이면 안 된다.(\cref{item:공력장치의 안전성과 강도} 참고) 공력장치에 관한 모든 제한은 차량의 바퀴가 똑바로 앞을 향한 상태를 기준으로 한다.
    \end{enumerate}
    
  \fig{공력장치}{formula}{0.8}
  
  \item 전면부 설치\\
    평면도(Top view) 상에서 공력장치의 모든 부분은 아래 규정을 만족하여야 한다.
    \begin{enumerate}
      \item 앞 타이어의 앞 방향으로 700mm 이상 돌출되면 안 된다.
      \item 전륜 허브 중심 높이에서의 타이어 바깥 면에 접하고, 차량 중심선에 평행한 두 수직면 사이에 위치해야 한다. (전륜 허브 중심 높이에서의 차 폭 이내에 위치해야 한다.)
      \item 차량을 전방에서 보았을 때 공차상태에서 지면으로부터 250mm 초과하는 높이에서는 전륜 휠 타이어가 공력 장치에 의해 가려져서는 안 된다.
    \end{enumerate}
    
  \item 후면부 설치\\ \label{item:공력장치 후면부 설치}
    평면도(Top view) 상에서 공력장치의 모든 부분은 아래 규정을 만족하여야 한다.
    \begin{enumerate}
      \item 뒷 타이어의 뒷 방향으로 250mm 이상 돌출되면 안 된다.
      \item 패딩류를 제외한 머리충격 흡수장치(head rest)의 최후면 보다 전방으로 돌출되어서는 안 된다.(undertray 제외, 머리충격 흡수장치를 드라이버에 맞게 조절할 수 있다면 가장 뒤쪽 위치로 한다.)
      \item 후륜 허브 중심 높이에서의 타이어 안쪽 면에 접하고, 차량 중심선에 평행한 두 수직면 사이에 위치해야 한다. (후륜 허브 중심 높이에서의 후륜 타이어 안쪽 폭 이내에 위치해야 한다.)
    \end{enumerate}
    측면에서 보았을 때 공력장치의 어떠한 부분도 공차상태에서의 높이 1.2m를 초과할 수 없다.
    
  \item 바퀴 사이 설치\\
    평면도(Top view) 상에서 전륜과 후륜 중심선 이내에 있는 공력장치는 전, 후륜의 휠 중심 높이에서의 휠의 바깥 면까지만 설치할 수 있다. \cref{item:공력장치 후면부 설치}에 따라 허용되는 경우를 제외하고, 전륜과 후륜의 중심선을 통과하는 두 횡방향 수직면 사이에 위치한 공력장치와 차체의 어떠한 부분도 공차상태에서 지면으로부터의 높이 500mm를 초과할 수 없다.\\
    다만, 차량 중심선에 평행하고 좌우 각 400mm 떨어진 수직면 이내에 위치한 차체는 위 요구사항에서 제외한다.
    
  \item 공력장치의 안전성과 강도 \label{item:공력장치의 안전성과 강도}
    \begin{enumerate}
      \item 모든 공력장치의 모든 부분은 임의의 지점에서 임의의 방향으로 하중을 부여하여 검사할 수 있다. (아래의 내용은 주행 중 공력장치의 이탈을 방지하기 위한 최소한의 요구사항이므로, 각 경기에서 기술심사위원의 재량에 따라 다르게 판단될 수 있다.)
      \item 큰 변형이 발견되는 부분에는 약 200N의 하중을 부여하여 확인하며, 이때 최대 변형량은 25mm를 넘지 않아야 하며 영구 변형은 5mm 이내여야 한다.
      \item 주행 중 공력장치의 의도되지 않은 과도한 움직임이 발견될 시 오피셜에 의해 재검을 요청받을 수 있다.
    \end{enumerate}
\end{enumerate}

\section{볼트, 너트 등 체결장치 – Fasteners} \label{section:체결장치}
\begin{enumerate}
  \item 체결장치로 피스류의 사용은 금지한다.
  \item 볼트의 등급 요구조건\\
    조향, 브레이크, 안전벨트와 서스펜션에 사용되는 볼트는 SAE 등급5 (M8.8) 이상이어야 한다. 또한, 볼트 체결 후 볼트의 피치는 2개 이상이어야 한다.
  \item 볼트, 너트 등 체결장치의 풀림 방지 체결 방식\\ \label{item:체결장치의 풀림 방지}
    조향, 제동장치, 안전벨트, 동력계통, 서스펜션, 공력장치와 고전압시스템에 쓰이는 볼트와 너트는 풀림을 방지할 수 있는 아래 1 \string~ 4와 같은 체결방식, 또는 이와 동등한 성능 이상의 풀림 방지 체결 방식을 써야 한다.
    \begin{enumerate}
      \item 핀(Cotter Pins)과 홈붙이 너트(Castle Nut)
      \item 철사와 홈붙이 너트(Castle Nut)
      \item 나일론 록 너트 (엔진, 배기장치, 브레이크 디스크, 캘리퍼, 고온이 발생하는 전기적 연결부에 사용할 수 없다.)
      \item 볼트 와이어링(Safety Wiring) (\cref{fig:볼트 와이어링 (Safety Wiring) 예시} 참조)
      \item 그 외 풀림 방지 체결방식을 사용할 경우 동등한 성능 이상을 증빙할 수 있는 자료를 준비하여야 한다.
    \end{enumerate}
  \item 화학적 고정제(순간접착제 등)의 사용이 풀림 방지 체결 방식에 해당하지 않는다. 서스펜션, 랙 \& 피니언, 조향 너클암 등은 풀림 방지를 할 수 있는 방식이어야 한다.
\end{enumerate}

\fig{볼트 와이어링 (Safety Wiring) 예시}{formula}{0.5}

\section{검차 후 개조와 수리의 제한 – Modifications \& Repairs}
\begin{enumerate}
  \item 검차 후의 개조는 허가되지 않는다. 단, 검차를 통과 못해 재검차 받아야 할 부분을 수정하는 것은 허락된다.
  \item 경기 중에는 타이어 공기압, 브레이크 유압조절, 서스펜션 조절, 스포일러 각도 조절, 체인의 장력 조절 등은 가능하다.
\end{enumerate}

\section{압축가스의 사용 – Compressed Gas}
\begin{enumerate}
  \item LPG, 프로판, 니트로 등의 연소용 가스 사용을 금지한다.
  \item 변속을 위한 불연소 압축가스의 사용은 허용한다.
  \item 사용하는 압축가스의 최대 압력을 견딜 수 있는 용기를 사용하여야 한다.
\end{enumerate}

\section{퀵 잭과 푸시바 – Quick Jack \& Push Bar}
\begin{enumerate}
  \item 각 팀은 퀵 잭(Quick Jack)과 푸시바(Push Bar)를 반드시 지참하여 한다.
  \item 차량 이동시 푸시바를 사용하여야 하며, 패독에서 정비 시 퀵 잭을 사용해야 한다.
  \item E-Formula는 반드시 절연 처리된 퀵 잭과 푸시바를 사용해야 한다.
\end{enumerate}

\section{난연성 재료 – Non-flammable material} \label{section:난연성 재료}
난연성 재료는 UL94-V0, FAR25 혹은 동등 이상의 등급을 만족해야 한다.

\section{촬영장치 장착}
\begin{enumerate}
  \item 각 팀은 필요 시 참가차량에 촬영 장치를 장착할 수 있다.
  \item 촬영장치는 드라이버 시야를 방해하거나 드라이버 신체 및 움직이는 공간 내 장착할 수 없다.
  \item 촬영장치의 고정은 반드시 브래킷을 이용하여야 한다.
\end{enumerate}

\clearpage

\begin{center}
  부칙
\end{center}

\begin{enumerate}[label=\arabic*.]
  \item 이 규정은 제정일로부터 시행한다.
  \item 이 규정의 제정 및 개정 이력은 아래와 같다.
\end{enumerate}

\begin{table}[H]
  \centering
  \begin{tblr}{
    width = \linewidth,
    rows = {ht=1.0\baselineskip},
    colspec = {X[c,m] X[c,m] X[c,m] X[c,m]}
  }
    2011. 6.16 제정 & 2012. 3. 8 개정 & 2013. 3. 8 개정 & 2014. 3.13 개정   \\
    2015. 3.12 개정 & 2016. 1.14 개정 & 2017. 1.19 개정 & 2017.12.21 개정   \\
    2019. 2.21 개정 & 2020. 1.15 개정 & 2020.12.17 개정 & 2022. 3.17 개정   \\
    2022.10.13 개정 & 2023.11.16 개정 &                &                    
  \end{tblr}
\end{table}

\clearpage

\thispagestyle{star}

\begin{center}
  {\fontsize{16}{18}\selectfont\pretendardb Formula 구조 대응물 양식 (Structural Equivalency Form)}\\[5ex]
\end{center}

이 양식은 조직위원회에서 제시하는 기간 안에 반드시 제출해야 한다. 조직위원회는 모든 전복안전성, 측면 충돌 보호 구조 규정에 대한 제출물을 검토한다. 또한 이 양식은 안전 및 기술적 사항 검사(Technical Inspection)에도 포함된다. 대체 재료를 사용한 경우 구조 대응물 양식(Structural Equivalency Form: SEF)과 차량기술규정 제22조를 만족한다는 결과로 실험 또는 증명한 계산결과물 등은 출력물로 제출해야 한다.

\vspace{\baselineskip}

※ 제출자 인적 사항 

\begin{table}[H]
  \centering
  \begin{tblr}{
    width = 0.9\linewidth,
    colspec = {|Q[c,m]|X[c,m]|Q[c,m]|X[c,m]|},
    rows = {ht=1.0\baselineskip},
    row{1} = {bg=gray!30, font=\bfseries},
    column{1,3} = {bg=gray!30, font=\bfseries},
    hlines,
    vlines
  }
    구분      & 내용 & 구분            & 내용 \\
    학교명    &      & 팀명            &     \\
    팀장      &      & 연락처 및 E-mail &     \\
    지도교수   &      & 연락처 및 E-mail &                   
  \end{tblr}
\end{table}

\vspace{\baselineskip}

※ 규정 불일치 사항 (적용되는 것에 체크) 

\begin{table}[H]
  \centering
  {\footnotesize
    \begin{tblr}{
      width = 0.9\linewidth,
      colspec = {|l|X|Q[c,m]|Q[c,m]|},
      rows = {ht=1.0\baselineskip},
      row{1,2} = {bg=gray!30, font=\bfseries},
      column{1} = {bg=gray!30, font=\bfseries},
      hlines,
      vlines
    }
      \SetCell[r=2]{c} 규정
        & \SetCell[r=2]{c} 내용
        & \SetCell[c=2]{c} 확인
        & \\
        & 
        & 일치
        & 불일치 \\
      \cref{item:메인 롤 후프}           & 메인롤후프 Main Roll Hoop                                         & & \\
      \cref{item:모노코크 메인 롤 후프}    & 메인롤후프와 모노코크 접합부 Main Roll Hoop Attachment to Monocoque   & & \\
      \cref{item:전방 롤 후프}           & 전방롤후프 Front Roll Hoop Material                               & & \\
      \cref{item:메인 롤 후프 지지대}      & 메인롤후프 지지대 Main Roll Hoop Bracing                           & & \\
      \cref{item:전방 롤 후프 지지대}      & 전방롤후프 지지대 Front Roll Hoop Bracing                          & & \\
      \cref{item:모노코크 지지대 접합부}    & 모노코크 지지대 접합부 Monocoque Bracing Attachment                 & & \\
      \cref{item:벌크헤드}               & 벌크헤드 Bulkhead                                                 & & \\
      \cref{item:모노코크 벌크헤드}        & 모노코크 벌크헤드 Monocoque Bulkhead                                & & \\
      \cref{item:충격완화장치}            & 충격완화장치 Impact Attenuator                                     & & \\
      \cref{section:측면 충돌 보호 구조}   & 측면 충돌 보호 구조 Side Impact Protection                          & & \\
      \cref{item:모노코크 안전벨트 접합부}  & 모노코크 안전벨트 접합부 Monocoque Safety Harness Attachment          & & \\
    \end{tblr}
  }
\end{table}

\vspace{\baselineskip}

※ 증명 첨부 자료 목록

\vspace{\baselineskip}

\hrule

{\footnotesize
  \begin{enumerate}[label=\arabic*.]
    \item 일치 재료 사용: 사용된 재료의 물성치를 알 수 있는 재질증명자료(항복강도, 인장강도 등)
    \item 불일치 재료 사용: 재료의 유형, 재료의 증명서, 특성, 열처리, 직물의 무게, 수지의 유형, 내구력 적응, 겹친 층의 수, 핵심 물질, 레이-업 기술, 힘에 대한 동등함을 보여주는 실험 또는 계산 결과물
  \end{enumerate}
}

\vspace{\baselineskip}

\noindent\hfill 날짜 : \underline{\qquad\qquad.\qquad\qquad.\qquad\qquad.} / 검차위원 : \underline{\qquad\qquad\qquad\qquad} / 조직위원 : \underline{\qquad\qquad\qquad\qquad}

\end{document}